\documentclass{report}
\usepackage{url, amsmath, amsthm, amssymb}
\usepackage{amsmath}
\usepackage[top=10mm,bottom=15mm,right=10mm,left=10mm]{geometry}
\linespread{1}
\newtheorem{theorem}{Theorem}[section]
\newenvironment{defn}[1][]{\refstepcounter{theorem}\begin{trivlist}
\item[\hskip \labelsep {\bfseries Definition  \thetheorem  \, \def\temp{#1}\ifx\temp\empty  #1\else  (#1)\fi
}]}   {\end{trivlist}}
\newenvironment{example}[1][]{\refstepcounter{theorem}\begin{trivlist}
\item[\hskip \labelsep {\bfseries Example  \thetheorem  \, \def\temp{#1}\ifx\temp\empty  #1\else  (#1)\fi
}]}   {\end{trivlist}}
\usepackage{multicol}
\begin{document}
\vspace*{2cm}
\begin{huge}
\begin{center}
\textbf{The infinite symmetric groups}
\end{center}
\end{huge}
\begin{align*}
&\text{Student:} &\text{Luke Elliott}\\
&\text{Supervisor:} &\text{Dr James D Mitchell}
\end{align*}

\begingroup
\let\clearpage\relax
\tableofcontents
\endgroup
\pagebreak
Throughout this document we will be working in ZFC.\\
Throughout this document we will use the following notations:
\begin{enumerate}
\item We use $\mathbb{N}$ to denote the set $\{1,2, \ldots\}$ and $\mathbb{N}_0$ will be used to denote the set $\{0,1,2, \ldots\}$
\item We use $\subseteq$ to denote a subset and $\subset$ be be used to denote a strict subset.
\item We use $S^c$ to denote the complement of $S$, as the subset of some other set, if this other set is obvious from the context.
\item We use $\langle S \rangle$ to denote the group generated by the set $S$, $\langle S \rangle_s$ to denote the semigroup generated by the set $S$ and $\langle S \rangle_T$ to denote the topology generated by the set $S$.
\item If $\alpha$ is an ordinal, we use $\alpha^+$ to denote the least ordinal greater than $\alpha$.
\end{enumerate}
It will be assumed that the reader is familiar with the following:
\begin{enumerate}
\item Concepts concerning ordinals and cardinals (these can be found in \cite{ordinals}).
\item Definition of a semigroup.
\item Basic properties and definitions concerned with sets, groups, metric spaces and topological spaces (most of these are found in MT2505, MT3502, MT4003 and MT4526)
\end{enumerate}
In this document we aim to prove many theorems concerning infinite symmetric groups including the following.

\begin{enumerate}
\item All elements of $Sym(\Omega)$ can be written as a commutator of elements of $Sym(\Omega)$
\item The semigroup $Sym(\Omega)$ satisfies the semigroup Bergman property.
\item There are $2^{2^{\vert \Omega \vert}}$ pairwise non-conjugate maximal subgroups of $Sym(\Omega)$
\end{enumerate}
\chapter{Indroduction}
We will start with some background on infinite sets and some basic definitions.
\begin{defn}
Let $(P,<)$ be a partially ordered set then and let $C\subseteq P$. We call $C$ a \textit{chain} if $C$ is totally-ordered by $<$.
\end{defn}
\begin{defn}
Let \((W,<)\) be a well ordered set. We call \(S \subseteq W\) an \textit{initial segment} of \(W\) if \(S=\{x \in W:x< M\}\) for some \(M\) in \(W\).
\end{defn}
\begin{theorem}\label{zorn's lemma}
(Zorn's Lemma) Let \((P,<)\) be a partially ordered set such that every chain has an upper bound. Then \(P\) has a maximal element \(M\).
\end{theorem}
\begin{proof}
The following proof comes from \cite{good zorn}.\\
Suppose for a contradiction there is no maximal element. By the axiom of choice let \(c\) be a choice function for \(\mathcal{P}(P)\). For \(a \in P\) let \(G_a:=\{x \in P:x> a\}\). For \(C \subseteq P\) a chain let \(G_C:=\{x \in P: x\text{ is an upper bound for }C\}\). As \(P\) has no maximal element and every chain has an upper bound we have that these sets are non-empty.\\
We call a chain \(C\) a c-chain if it satisfies the following:
\begin{enumerate}
\item \(C\) is well-ordered by \(<\).
\item If \(C_M\subset C\) is an initial segment of \(C\) with maximal element \(M\) then \(min\{x \in C:x \notin C_{M}\} = c(G_M)\).
\item If \(C_u\subset C\) is an initial segment of \(C\) with no maximal element then \(min\{x \in C:x \notin C_{u}\} = c(G_{C_u})\).
\end{enumerate}
Let \(C_1 \neq C_2\) be c-chains, without loss of generality assume that \(C_1 \backslash C_2 \neq \emptyset\).\\
Let \(m = min(C_1 \backslash C_2)\), we have that \(A = \{x \in C_1:x<m\}(= \{x \in C_2:x<min\{y \in C_2:y\geq m\}\})\) is an initial segment of both \(C_1\) and \(C_2\) assuming \(C_2 \backslash A \neq \emptyset\). We have \(m=min(C_1 \backslash A)=min(C_2 \backslash A)\) the second equality following from the fact that \(C_1,C_2\) are c-chains and thus the element following any initial segment is uniquely determined. So \(m \in C_2\) which contradicts it's definition. Thus we have that \(C_2 \backslash A = \emptyset\) and therefore \(C_2 = A\) is an initial segment of \(C_1\).\\
Let \(S\) be the union of all c-chains of \(P\). We will show that S is a c-chain:
\begin{enumerate}
\item  Let \(s,t \in S\) and let \(C_s,C_t\) be c-chains with \(s,t\) maximal respectively, we have that either \(C_s \subseteq C_t\) or \(C_t \subseteq C_s\), so either \(s=t,s<t\) or \(t<s\) and therefore \(S\) is totally ordered. Let \(A \subseteq S\) be non-empty and let \(t \in A\), consider the set \(B:=\{x \in A: x < t\}\), if \(B\) has a least element then \(A\) has a least element and so \(S\) is well ordered. All \(x \in B\) are contained in a c-chain \(C_x\) with \(x\) as maximal element, if \(C_x \supset C_t\) it follows that \(t \leq x\) a contradiction so we have \(C_x \subseteq C_t\) and thus \(B \subseteq C_t\) and so \(B\) has a minimal element and \(S\) is well ordered.
\item Let \(S_M \subset S\) be an initial segment with maximal element \(M\). Let \(s \in S\backslash S_M\). It follows that there is a c-chain \(C_{sM}\) containing \(M\) with maximal element \(s\) thus \(c(G_M) = min\{y \in C_{sM}:y \notin\{x \in C_{sM}:x \leq M\} \} \in S\). It follows that \(c(G_M) \leq s\) but \(s\) was arbitrary so we have that \(min\{x \in S: x \notin S_M\} = c(G_M)\) as required.\\
\item Let \(S_u \subset S\) be an initial segment no maximal element. Let \(s \in S\backslash S_u\). Let \(C_s\) be a c-chain with \(s\) it's maximal element. If \(x \in S_u\) it follows there is a c-chain with \(x\) maximal which is an initial segment of \(C_s\) and thus \(x \in C_s\). So we have that \(S_u \subset C_s \subseteq S\), as \(C_s\) is a c-chain it follows that \(c(G_{S_u}) = min\{y \in C_{s}:y \notin S_{u} \}\in S\) and as \(s\) was arbitrary and \(s \geq c(G_{S_u})\) we have that \(min\{x \in S: x \notin C_u\}=G_{S_u}\)\\
\end{enumerate}
Now we have that \(S\) is a c-chain and there is no greater c-chain than \(S\) by definition. However if \(S\) has a maximal element \(M\) then \(S \cup \{c(G_{M})\}\) is a strictly greater c-chain and if \(S\) has no maximal element we have that \(S \cup \{c(G_{S})\}\) is a strictly greater c-chain so we have reached a contradiction.
\end{proof}
\begin{theorem}
Every set $A$ is well-orderable and thus there is a cardinal $\kappa$ such $\vert A\vert = \kappa$
\end{theorem}
\begin{proof}
This proof comes from \cite{well orderable}.\\
Let $A_o:=\{(S,R):S \subseteq A, R\text{ is a well ordering of S}\}$. Let $A_o$ be partially ordered by: $(S_1,R_1) \leq (S_2,R_2)$ iff $S_1 \subseteq S_2$, $R_2$ restricted to $S_1$ is equal to $R_1$ and for all $x_1\in S_1,x_2 \in S_2\backslash S_1$ we have $x_1<x_2$ with respect to $R_2$. Note that $A_o$ is non-empty as singletons are trivially well-orderable.\\
Let $((S_i,R_i))_{i<\kappa}$ be a chain. We can construct an upper bound for this chain as follows. Let the set be $\cup_{i<\kappa}S_i$ and the well-ordering be defined by: For $x_i \in S_i,x_j\in S_j$ we have $x_i\leq x_j \iff x_i\leq x_j$ with respect to $R_{max\{i,j\}}$. This is well-defined as the orderings agree whenever they are defined. It is a well-ordering as we can construct a least element for any subset as follows. Let $X\subseteq \cup_{i<\kappa}S_i$ be non-empty, it contains and element $x_i\in S_i$ for some $i$. It follows that $S_i\cap X$ is non-empty and thus has a least element, by the definition of our partial ordering this is also a minimum element for $X$ as required.\\
By theorem \ref{zorn's lemma} $A_o$ has a maximal element $(X,R_x)$. If $X \neq A$ then there is an element $a \in A \backslash X$ and therefore we have that $(X\cup\{a\},R_x \text{(except a is bigger than everything)}) > (X,R_x)$ a contradiction. Therefore we have that $A$ is well-ordered by $R_x$.
\end{proof}
\begin{theorem}\label{double cardinal}
Let $\Omega$ be an infinite set then $\vert \Omega \vert = 2\vert \Omega \vert$.
\end{theorem}
\begin{proof}
Without loss of generality we may assume that $\Omega$ is a cardinal. We will define an equivalence relation on $\Omega$ as follows: 
\[ a \sim b \iff a=b+k \text{ or } b=a+k \text{ for some }k \in \aleph_0\]
We can partition $\Omega$ into equivalence classes under this relation each of which is countably infinite. It therefore follows that we can construct a bijection for each of these into two disjoint copies of itself (as $\vert \aleph_0 \vert =\vert \aleph_0 \sqcup \aleph_0 \vert$). By adjointing these bijections we have a bijection from $\Omega$ into two disjoint copies of itself as required.
\end{proof}
\begin{theorem}
Let $\Omega$ be an infinite set then $\vert \Omega \vert = \vert \Omega \vert^2$.
\end{theorem}\par
Proof: Without loss of generality we can assume $\Omega$ is a cardinal. Let $\Omega=\{(S,\phi):S \subseteq \Omega, \phi:S\rightarrow S \times S \text{ is a bijection}\}$. Let $\Omega_o$ be partially ordered by: $(S_1,\phi_1) \leq (S_2,\phi_2)$ iff $S_1 \subseteq S_2$, $\phi_2$ restricted to $S_1$ is equal to $\phi_1$. Note that $\Omega_o$ is non-empty as there is a bijection from $\aleph_0$ to $\aleph_0 \times \aleph_0$.\\
Let $((S_i,\phi_i))_{i<\kappa}$ be a chain. We can construct an upper bound for this chain as follows. Let the set be $\cup_{i<\kappa}S_i$ and the bijection $\phi_U$ be defined by: For $x_i \in S_i$ we have $(x_i)\phi_U=(x_i)\phi_i$. This is well-defined as the bijections agree whenever they are defined.\\
By theorem \ref{zorn's lemma} $\Omega_o$ has a maximal element $(X,\phi_x)$. If $\vert X \vert = \vert \Omega \vert$ then we are done. Suppose for a contradiction that $\vert X \vert < \vert \Omega \vert$, then $\vert X \vert < \vert X^c \vert$ and so there exists $X'\subseteq X^c$ such that:
\[\vert X \vert = \vert X' \vert = \vert X' \times X' \vert=\vert X' \times X' \vert + \vert X' \times X \vert + \vert X \times X' \vert= \vert (X' \times X') \cup (X' \times X) \cup (X \times X') \vert\]
Let $\phi_x':X' \rightarrow (X' \times X') \cup (X' \times X) \cup (X \times X')$ be a bijection. By adjoining the functions $\phi_x$ and $\phi_x'$ we can construct a bijection $\phi_x'':X\cup X' \rightarrow (X\cup X') \times (X\cup X')$. We therefore have that $(X,\phi_x)<(X\cup X',\phi_x'')$ a contradiction. $\qed$
\begin{theorem}
Let $\Omega_1,\Omega_2,\Omega_3$ be sets then $\vert \Omega_1^{\Omega_2 \times \Omega_3} \vert=\vert {(\Omega_1^{\Omega_2})}^{\Omega_3} \vert$
\end{theorem}\par
Proof: Let $\phi: {(\Omega_1^{\Omega_2})}^{\Omega_3} \rightarrow \Omega_1^{\Omega_2 \times \Omega_3}$ be defined by:
\[(f)\phi := \left\{
    \begin{array}{lr}
      (a)((b)f)&  (a,b) \in  \Omega_2 \times \Omega_3 \\
    \end{array}
    \right\}\]
We have that $\phi$ is the required bijection. $\qed$
\begin{defn}
Let $\Omega$ be an infinite set and let $S$ be a subset of $\Omega$. We call $S$ a \textit{moiety} of $\Omega$ if $\vert S\vert = \vert S^c \vert = \vert \Omega \vert$. Notice that it follows from theorem \ref{double cardinal} that all infinite sets have moiety subsets.
\end{defn}
\begin{defn}
An \textit{infinite symmetric group} $Sym(\Omega)$ is defined to be the set of bijections $f:\Omega\rightarrow\Omega$ under composition, where $\Omega$ is an infinite set.
\end{defn}
\begin{defn}
Let $\Omega$ be an infinite set, let $G \leq Sym(\Omega)$ and let $\Sigma$ be a subset of $\Omega$. We say that $\Sigma$ is \textit{full} in $G$ or $G$ acts fully on $\Sigma$ if for all $f\in Sym(\Sigma)$ there is a $f'\in G$ such that $f'\vert_{\Sigma}=f$
\end{defn}
\begin{defn}
Let $f\in Sym(\Omega)$, then the fix of $f$ is defined by:
\[fix(f):= \{x \in \Omega : (x)f = x\}\]
\end{defn}
\begin{defn}
Let $\Omega$ be an infinite set and let $S$ be a set on which $Sym(\Omega)$ can act.\\
The \textit{pointwise stabilizer} of S and \textit{setwise stabilizer} of S are defined to be the following:
\[Pstab(S)=\{f \in Sym(\Omega):(x)f=x \ \ \text{ for all } x \in S\}\ \ \ \ Sstab(S)=\{f \in Sym(\Omega):(x)f \in S \iff x \in S\}\]
\end{defn}
\begin{theorem}
Let $\Omega$ be an infinite set and let $S$ be a set on which $Sym(\Omega)$ can act.\\ Then the pointwise and setwise stabilizers of $S$ are groups.
\end{theorem}
\begin{proof}
Pointwise: Let $f \in Pstab(S)$ and let $x \in S$.\\
$(x)f=x \implies (x)f^{-1}=x$ so $Pstab(S)$ is closed under inverses.\\
Let $f,g \in Pstab(S)$ and let $x \in S$. Then $(x)fg=((x)f)g=(x)g=x$. So we have that $fg\in Pstab(S)$ and therefore $Pstab(S)$ is a group.\\
Setwise: Let $f \in Sstab(S)$ and $(x)f^{-1} \in S$ then $(xf^{-1})f \in S$ so $x \in S$\\
Let $x \in S$ then $((x)f^{-1})f \in S$ so $(x)f^{-1} \in S$. So we have that $Sstab(S)$ is closed under inverses.\\
Let $f,g \in Sstab(S)$ and $x \in S$ then $(x)f \in S$ so $(x)fg \in S$. We therefore have that $Sstab(S)$ is a group.
\end{proof}
\begin{defn}
Let $\Omega$ be an infinite set and let $S \subseteq \Omega$.
\[Sym_{\Omega}(S):=Pstab(S^c)\]
Note that $Sym_{\Omega}(S)$ is isomorphic to $Sym(S)$.
\end{defn}
\begin{theorem}\label{big}
An infinite symmetric group $Sym(\Omega)$ has cardinality $2^{\vert \Omega \vert}$.
\end{theorem}\par
Proof: Let $S$ be a subset of $\Omega$. If $\vert S \vert$ is finite, then there exists a bijection $\phi:S \rightarrow \{0 \ldots (n-1)\}$ for some $n \in \mathbb{N}_0$. We can construct an element $f_S$ of $Sym(\Omega)$ which maps $x\in S$ by $(x)f_S=((x)\phi + 1 $ mod n$)\phi^{-1}$ and fixes all other elements. If $\vert S \vert$ is infinite then we can construct a partition $\{M_1,M_2\}$ of $S$ such that $M_1,M_2$ are moieties of $S$. There exists a bijection $\phi:M_1 \rightarrow M_2$ so we can construct an element $f_S$ of $Sym(\Omega)$ by:
\[(x)f_S =  \left\{
    \begin{array}{lr}
      (x)\phi&  x\in M_1  \\
      (x)\phi^{-1}& x\in M_2\\
      x& \text{otherwise}\\
    \end{array}
    \right\}\]
For $\vert S\vert \geq 2$ then these maps $f_S$ fix precisely the elements not in S and therefore they are all distinct.
\[2^{\vert \Omega \vert} = 2^{\vert \Omega \vert} - \vert\Omega\vert = \vert P(\Omega) \backslash \{\{\phi\}\cup \{\{x\}:x\in \Omega\}\} \vert \leq\vert Sym(\Omega)\vert \leq \vert \Omega^{\Omega}\vert \leq \vert (2^{\vert \Omega \vert})^{\vert \Omega \vert} \vert=\vert 2^{\vert \Omega \times \Omega \vert} \vert=2^{\Omega}\]
  Therefore we have that $\vert Sym(\Omega)\vert=2^{\vert \Omega \vert}.\qed$
\chapter{Commutators}
\section{Infinite permutations}
In this chapter we will prove that every element of $Sym(\mathbb{N})$ can be written as a commutator. This will then be generalised to $Sym(\Omega)$ for any infinite set $\Omega$. We start however by introducing some useful notions about infinite permutations.
\begin{defn}
Let $\Omega$ be an infinite set. Let $p \in \Omega$ and $S\subseteq Sym(\Omega)$ then the \textit{orbit} of $p$ with respect to $S$ is defined by:
\[orb_S(p):=\{x \in \Omega: x=(p)f \text{ for some }f \in \langle S \rangle\}\]
\end{defn}
\begin{defn}
Let $\Omega$ be an infinite set, and let $f \in Sym(\Omega)$. The term \textit{disjoint cycle shape} of $f$ is used to describe how many disjoint orbits of points, with respect to the set $\{f\}$ there are of each cardinality. The term \textit{$\kappa$-cycle} will be used to refer to $f$ restricted to one of these orbits with cardinality $\kappa$ (or sometimes the orbit itself).\\
Note that any cycle must be countable.\\
We will also use the notation $(\ldots a_{-1},a_0,a_1,a_2 \ldots)$ to denote an cycle which maps $a_i$ to $a_{i+1}$.
\end{defn}
\begin{theorem}\label{cycle conjugate}
Let $\Omega$ be an infinite set and let $f,g \in Sym(\Omega)$, then $f$ and $g$ are conjugate iff they have the same disjoint cycle shape.
\end{theorem}\par
Proof:$(\impliedby)$  Let $C_{f,i}$ be the set of i-cycles of $f$ and similarly $C_{g,i}$ be the set of i-cycles of $g$. We have that $\vert C{f,i}\vert=\vert C{g,i}\vert$ for all $i \in \aleph_0 \cup \{\aleph_0\}$.\\
Label their elements such that:
\[C_{f,i} = \{C_{f,i,j}:j < \vert C{f,i}\vert\}\]
\[C_{g,i} = \{C_{g,i,j}:j < \vert C{g,i} \vert\}\]
Note that $\vert C_{f,i,j}\vert = \vert C_{g,i,j}\vert = i$ for all j.\\
As disjoint cycles partition $\Omega$ we have that for all $p \in \Omega$   there exists unique $i_p,j_p$ such that $p \in C_{f,i_p,j_p}$.\\
For all $i$ and $j$ we have that $f\vert_{C_{f,i,j}}$ cycles the elements of $C_{f,i,j}$ and similarly $g\vert_{C_{g,i,j}}$ cycles the elements of $C_{g,i,j}$ so we can construct a bijection $\phi_{i,j}:C_{f,i,j} \rightarrow C_{g,i,j}$ such that for all $x \in C_{f,i,j}$ we have $(x)f\phi_{i,j} = (x)\phi_{i,j}g$. \\
Let $h$ be defined by:
$(p)h:(p)\phi_{i_p,j_p}$\\
It is clear that $h^{-1} f h = g$ as required.\\
$(\implies)$ Suppose that $h^{-1}gh = f$ for some $h \in Sym(\Omega)$.
Let $C_f=(\ldots c_{-1},c_0, c_1 \ldots c_k)$ be a cycle of $f$.\\
$h^{-1}C_fh = ((c_1)h,(c_2)h \ldots (c_k)h)$.\\
We therefore have that all disjoint cycles in $f$ have a unique corresponding cycle in $g$ and therefore $g$ has the same number of cycles of each length as $f$ and so $g$ has the same disjoint cycle shape as $f$. $\qed$\\
\\
The desired result for this chapter will take some time to prove, however there is a weaker result which can be show with relatively little effort.
\begin{theorem}
All functions $f\in Sym(\mathbb{N}_0)$ such that $(i)f = i$ for all $i \geq k$ for some $k\in \mathbb{N}_0$ can be written in the form:
\[f=[g,h]=ghg^{-1}h^{-1}\]
where $g,h\in Sym(\mathbb{N}_0)$.
\end{theorem}\par
Proof: let $g,h\in Sym(\mathbb{N}_0)$ be defined by:\\
Let $n\in\mathbb{N}_0$\\
Let $n = 2kq + r$ where $q\in \mathbb{N}_0$ and $0\leq r < 2k$
\[(n)g=
 \left\{
    \begin{array}{lr}
      2qk + (r)f&  r<k\\
      n& otherwise
    \end{array}
    \right\}
\ \ \ \ \ (n)h=
 \left\{
    \begin{array}{lr}
      n + k&  n<k \\
      n-2k&  r<k\ \ \text{and}\ \ q>0\\
      n+2k& r\geq k
    \end{array}
    \right\}
\]
Note that $g,h$ have the following inverses:
\[(n)g^{-1}=
 \left\{
    \begin{array}{lr}
      2qk + (r)f^{-1}& r<k\\
      n& otherwise
    \end{array}
    \right\}
\ \ \ \ \ (n)h^{-1}=
 \left\{
    \begin{array}{lr}
      n - k&  k\leq n<2k \\
      n+2k&  r<k\\
      n-2k&  r\geq k\ \ \text{and}\ \ q>0
    \end{array}
    \right\}
\]
We now show that $f=[g,h]$.
\begin{multicols}{3}
\begin{align*}
&\text{Case 1: } n<k\\
(n)[g,h]&=((((n)g)h)g^{-1})h^{-1}\\
&=((((r)g)h)g^{-1})h^{-1}\\
&=(((r)f+k)g^{-1})h^{-1}\\
&=((r)f+k)h^{-1}\\
&=(r)f=(n)f
\end{align*}


\begin{align*}
&\text{Case 2: }q>0\text{ and }r<k\\
(n)[g,h]&=((((n)g)h)g^{-1})h^{-1}\\
&=((((2qk+r)g)h)g^{-1})h^{-1}\\
&=(((2qk+(r)f)h)g^{-1})h^{-1}\\
&=((2(q-1)k+(r)f)g^{-1})h^{-1}\\
&=2qk+r=n=(n)f
\end{align*}


\begin{align*}
&\text{Case 3: }k\leq r\\
(n)[g,h]&=((((n)g)h)g^{-1})h^{-1}\\
&=(((n)h)g^{-1})h^{-1}\\
&=((n+2k)g^{-1})h^{-1}\\
&=(n+2k)h^{-1}\\
&=n=(n)f\qed
\end{align*}
\end{multicols}
\section{Constructing a topology on an infinite symmetric group}
In this section we will be constructing a completely metrizable topology on $Sym(\mathbb{N})$. We will later use this topology to prove the desired result of the chapter.
\begin{defn}
The countably infinite product topology of the discrete topology on $\mathbb{N}$ will be denoted by $T$. This topology is defined on the set $\mathbb{N}^\mathbb{N}$.
\begin{align*}
\mathbb{N}^{<\mathbb{N}} &:= \{\sigma : \{1,2 \ldots n\}\rightarrow\mathbb{N}: n\in \mathbb{N}\}\\
[\sigma] &:= \{f\in \mathbb{N}^{\mathbb{N}} : f\vert_{dom(\sigma)} = \sigma\}\\
B&:= \{[\sigma]:\sigma \in \mathbb{N}^{<\mathbb{N}}\}
\end{align*}
\end{defn}
\begin{theorem}
The set $B$ above is a basis for $T$.
\end{theorem}\par
Proof: We have that $B_o := \{U_1\times U_2 \ldots  \times U_k \times \mathbb{N}\times\mathbb{N}\times \ldots : k\in\mathbb{N},\ U_i\subseteq \mathbb{N}\ \ \text{for all } i\in \{1,2 \ldots k\}\}$ is a basis for T by the definition of an infinite product topology.\\
It suffices to show that for all $U \in B_o$ there exists $G_U \subseteq B$ such that $U = \cup G_U$.
Let $U := U_1\times U_2 \ldots \times U_k \times \mathbb{N}\times\mathbb{N}\times \ldots$\\
 Let $G_U :=\{[\sigma]:\sigma\in\mathbb{N}^{<\mathbb{N}}, dom(\sigma)=\{1,2 \ldots k\}, \text{for all } i \in dom(\sigma)\ (i)\sigma\in U_i\}$.
Clearly $G_U \subseteq B$.\\
It therefore suffices to show that $U = \cup G_U.$
\begin{multicols}{2}
\begin{center}
$(\subseteq)$ let $f\in U$\\
$(i)f\in U_i$ for all $i\in \{1,2\ldots k\}$\\
$let\ \sigma_f := f\vert_{\{1,2 \ldots k\}}\in\mathbb{N}^{<\mathbb{N}}$\\
$f\in [\sigma_f] \in G_U 
\implies f\in \cup G_U$\\
$(\supseteq)\ let\ f\in \cup G_U$\\
there exists $[\sigma_f]\in G_U \ \text{such that} \ f\in [\sigma_f]$\\
$(i)f = (i)\sigma_f \in U_i\ \text{ for all } i \in \{1,2 \ldots k\}$\\
$\implies f\in U \qed$\\
\end{center}
\end{multicols}
\begin{defn}
A topological space is called \textit{completely metrizable} if it is induced by a complete metric space.
\end{defn}
\begin{theorem}
Let $d:\mathbb{N}^\mathbb{N}\times\mathbb{N}^\mathbb{N}\rightarrow\mathbb{R}^+$ be defined by:
$$d(f,g)=
 \left\{
    \begin{array}{lr}
      0& f=g \\
      \frac{1}{min\{i\in\mathbb{N}:(i)f \neq (i)g\}}& f\neq g
    \end{array}
    \right\}
$$
$(\mathbb{N}^\mathbb{N}, d)$ is a complete metric space which induces the topology T, and thus T is completely metrizable.
\end{theorem}\par
Proof: We first show that d is a metric. The non-negativity, identity of indiscernibles and symmetry conditions are clearly true from the definition.\\
Triangle inequality: Let $f,g,h \in \mathbb{N}^\mathbb{N}:$\\
First notice that if $j=min\{i\in\mathbb{N}:(i)f\neq (i)h\}$ then either $(j)f\neq (j)g$ or $(j)g\neq (j)h$. It therefore follows:
\begin{align*}
min\{i\in\mathbb{N}:(i)f\neq (i)h\} &\geq min\{i\in\mathbb{N}:(i)f\neq (i)g\ or\ (i)g \neq (i)h\}\\
\implies min\{i\in\mathbb{N}:(i)f\neq (i)h\} &\geq min\{min\{i\in\mathbb{N}:(i)f\neq (i)g\},min\{i\in\mathbb{N}:(i)g\neq (i)h\}\}\\
\implies \frac{1}{d(f,h)} &\geq min\{\frac{1}{d(f,g)},\frac{1}{d(g,h)}\}\\
\implies \frac{1}{d(f,h)} &\geq \frac{1}{max\{d(f,g),d(g,h)\}}\\
\implies d(f,h) &\leq max\{d(f,g),d(g,h)\}\\
\implies d(f,h) &\leq d(f,g)+d(g,h)
\end{align*}
We next show that $(\mathbb{N}^\mathbb{N}, d)$ is complete.\\
let $S = (f_1, f_2, f_3\ldots)$ be a Cauchy sequence.\\\par
\noindent\fbox{%
    \parbox{\textwidth}{%
Claim: for all $i \in \mathbb{N}\ \text{there is a minimal }M_i\in \mathbb{N}\ \text{such that for all } j \in \{1,2\ldots i\}\text{ and all } n \geq M_i$
$$(j)f_{M_i} = (j)f_n$$
Proof of Claim: $let\ i\in \mathbb{N}$\\
as $S$ is Cauchy we have that for all $\varepsilon > 0$ there exists an $N \in \mathbb{N}\ \text{such that}\ \text{for all } n,m \geq N$
$$d(f_n, f_m) < \varepsilon$$
choose $\varepsilon = \frac{1}{i+1}$
$$\text{there exists } N_i \ \text{such that}\text{ for all } n,m \geq N_i\ \ \ \ d(f_n,f_m) < \frac{1}{i+1}$$
$$\implies d(f_{N_i},f_n)< \frac{1}{i+1}\ \ \text{ for all } n\geq N_i$$
$$\implies (j)f_{N_i}=(j)f_n\ \text{ for all } n\geq N_i \ \text{ for all } j \in \{1,2\ldots i\}$$
As there exists a natural number $N_i$ with the desired property there must exist a minimal $M_i$ with this property. $\qed$
    }%
}\\\par
define $l:\mathbb{N}\rightarrow\mathbb{N}$ by: $(i)l=(i)f_{M_i}$.\\
It suffices to show that $S$ converges to $l$.\\
It is clear from the definition of $M_i$ that for all $i\in \mathbb{N}\ \  M_i \leq M_{i+1}$\\
$($For all $i\in \mathbb{N}\ \  M_i \leq M_{i+1})\implies\ ($For all $i\in \mathbb{N}\ \ (i)f_{M_{i+1}}=(i)f_{M_{i}}) \implies\ ($For all $i,j\in \mathbb{N}\ \ (i)f_{M_{i}}=(i)f_{M_{i+j}})$\\
We have that for all $n \geq M_i$ and all $j \in \{1,2\ldots i\}$: $(j)l=(j)f_{M_j}=(j)f_{M_i}=(j)f_n$ and therefore $d(l,f_n) < \frac{1}{i}$.\\
We have just shown that for all $i \in \mathbb{N}$ there exists $M_i\in\mathbb{N}$ such that for all $n\geq M_i$ $d(l,f_n)< \frac{1}{i}$.\\
In particular for all $\varepsilon > 0$ there exists an $N\in\mathbb{N}$ such that for all $n\geq N$ $d(l,f_n)<\varepsilon$ and so $S$ converges to $l$ as required.\\\par
Finally we show that $T$ is induced by $d$.
\begin{align*}
\text{The topology induced by d} &=\langle\{\{g\in \mathbb{N}^\mathbb{N}:d(g,f)<\varepsilon\}:f\in \mathbb{N}^\mathbb{N},\varepsilon>0\}\rangle_T\\
&=\langle \{\{g\in \mathbb{N}^\mathbb{N}:(i)g=(i)f \text{ for all } i \in \{1,2 \ldots \lfloor \frac{1}{\varepsilon}\rfloor\}\}:f\in \mathbb{N}^\mathbb{N},\varepsilon>0\}\rangle_T\\
&=\langle\{\{g\in \mathbb{N}^\mathbb{N}:(j)g=(j)f \text{ for all } j \in \{1,2 \ldots i\}\}:f\in \mathbb{N}^\mathbb{N},i\in\mathbb{N}\}\rangle_T\\
&=\langle\{\{g\in \mathbb{N}^\mathbb{N}:(j)g=(j)\sigma \text{ for all } j \in dom(\sigma)\}:\sigma\in \mathbb{N}^{<\mathbb{N}}\}\rangle_T\\
&=\langle\{\{g\in \mathbb{N}^\mathbb{N}:g\vert_{dom(\sigma)}=\sigma\}:\sigma\in \mathbb{N}^{<\mathbb{N}}\}\rangle_T\\
&=\langle\{[\sigma]:\sigma\in \mathbb{N}^{<\mathbb{N}}\}\rangle_T= T \qed
\end{align*}
\begin{theorem} \label{metric product}
Let $((X_1,d_1),(X_2,d_2) \ldots)$ be complete metric spaces. Then $(X_\pi,d_\pi)$ where 
\[X_\pi := \prod_{i=1}^{\infty}{X_i}\ \ \ \ \ d_\pi ((x_{1,1},x_{1,2}\ldots ),(x_{2,1},x_{2,2}\ldots )) := \sum_{i\in \mathbb{N}}{\frac{d_i(x_{1,i},x_{2,i})}{2^i \times (1 + d_i(x_{1,i},x_{2,i}))}}\] is a complete metric space.
\end{theorem}\par
Proof: We first show that $d_\pi$ is a metric.\\
The non-negativity and symmetry conditions are clearly true from the definition.
\[x_1 = (x_{1,1},x_{1,2}\ldots)\ \ \ \ \ x_2 = (x_{2,1},x_{2,2}\ldots)\ \ \ \ \ x_3 = (x_{3,1},x_{3,2}\ldots)\]
identity of indiscernibles:
\[d(x_1,x_1) = \sum_{i\in \mathbb{N}}{\frac{d_i(x_{1,i},x_{1,i})}{2^i \times (1 + d_i(x_{1,i},x_{1,i}))}} = \sum_{i\in \mathbb{N}}{\frac{0}{2^i \times (1 + 0)}}= 0\]
\begin{align*}
&d(x_1,x_2)=0\\
&\implies \sum_{i\in \mathbb{N}}{\frac{d_i(x_{1,i},x_{2,i})}{2^i \times (1 + d_i(x_{1,i},x_{2,i}))}} = 0\\
&\implies \frac{d_i(x_{1,i},x_{2,i})}{2^i \times (1 + d_i(x_{1,i},x_{2,i}))} = 0\ \text{ for all } i \in \mathbb{N}\\
&\implies d_i(x_{1,i},x_{2,i}) = 0 \ \text{ for all }i \in \mathbb{N}\\
&\implies x_{1,i} = x_{2,i}\ \text{ for all }i \in \mathbb{N}\\
&\implies x_{1} = x_{2}
\end{align*}
triangle inequality:\\
note that $f(x):= \frac{x}{1+x}$ is an increasing function on $\mathbb{R}^+$ as $f' (x)= \frac{1}{(1+x)^2}$\\
\begin{align*}
d_\pi(x_1,x_3) &= \sum_{i\in \mathbb{N}}{\frac{d_i(x_{1,i},x_{3,i})}{2^i \times (1 + d_i(x_{1,i},x_{3,i}))}}\\
&= \sum_{i\in \mathbb{N}}{\frac{f(d_i(x_{1,i},x_{3,i}))}{2^i}}\\
&\leq \sum_{i\in \mathbb{N}}{\frac{f(d_i(x_{1,i},x_{2,i}) + d_i(x_{2,i},x_{3,i}))}{2^i}}\\
&= \sum_{i\in \mathbb{N}}{ \frac{d_i(x_{1,i},x_{2,i}) + d_i(x_{2,i},x_{3,i})}{2^i \times (1 + d_i(x_{1,i},x_{2,i})+d_i(x_{2,i},x_{3,i}))}}\\
&= \sum_{i\in \mathbb{N}}{\Big( \frac{d_i(x_{1,i},x_{2,i})}{2^i \times (1 + d_i(x_{1,i},x_{2,i})+d_i(x_{2,i},x_{3,i}))} + \frac{d_i(x_{2,i},x_{3,i})}{2^i \times (1 + d_i(x_{1,i},x_{2,i})+d_i(x_{2,i},x_{3,i}))}\Big)}\\
&= \sum_{i\in \mathbb{N}}{ \frac{d_i(x_{1,i},x_{2,i})}{2^i \times (1 + d_i(x_{1,i},x_{2,i})+d_i(x_{2,i},x_{3,i}))}} + \sum_{i\in \mathbb{N}}{\frac{d_i(x_{2,i},x_{3,i})}{2^i \times (1 + d_i(x_{1,i},x_{2,i})+d_i(x_{2,i},x_{3,i}))}}\\
&\leq \sum_{i\in \mathbb{N}}{\frac{d_i(x_{1,i},x_{2,i})}{2^i \times (1 + d_i(x_{1,i},x_{2,i}))}} + \sum_{i\in \mathbb{N}}{\frac{d_i(x_{2,i},x_{3,i})}{2^i \times (1 + d_i(x_{2,i},x_{3,i}))}}\\
&=d_\pi (x_{1,i},x_{2,i}) + d_\pi (x_{2,i},x_{3,i})
\end{align*}
We now show that $(X_\pi, d_\pi)$ is complete\\
Let $S = (x_1,x_2 \ldots )$ where $$x_i = (x_{i,1},x_{i,2}\ldots )\ \text{ for all }i \in \mathbb{N}$$
be a Cauchy sequence.\\
S is Cauchy\\
\begin{align*}
&\implies\text{ for all }\varepsilon > 0\text{ there exists an }N\in \mathbb{N}\ \text{ such that}\text{ for all }n,m \geq N\ d_\pi (x_{n}, x_{m}) \leq \varepsilon\\
&\implies\text{ for all }\varepsilon > 0\text{ there exists an }N\in \mathbb{N}\ \ \text{such that for all }n,m \geq N\ \sum_{i\in \mathbb{N}}{\frac{d_i(x_{n,i},x_{m,i})}{2^i \times (1 + d_i(x_{n,i},x_{m,i}))}} \leq \varepsilon\\
&\implies (\text{for all }\varepsilon > 0\text{ there exists an }N\in \mathbb{N}\ \ \text{such that for all }n,m \geq N\ \frac{d_i(x_{n,i},x_{m,i})}{2^i \times (1 + d_i(x_{n,i},x_{m,i}))} \leq \varepsilon)\text{ for all }i \in \mathbb{N}\\
&\implies (\text{for all }\varepsilon > 0\text{ there  exists an }N\in \mathbb{N}\ \ \text{such that for all }n,m \geq N\ \frac{d_i(x_{n,i},x_{m,i})}{1 + d_i(x_{n,i},x_{m,i})} \leq \varepsilon)\text{ for all }i \in \mathbb{N}\\
&\implies (\text{for all }\varepsilon > 0\text{ there exists an }N\in \mathbb{N}\text{ such that for all }n,m \geq N\ d_i(x_{n,i}, x_{m,i}) \leq \varepsilon)\text{ for all }i \in \mathbb{N}\\
&\implies (x_{1,i}, x_{2,i} \ldots )\text{ is Cauchy w.r.t }d_i\text{ for all }i \in \mathbb{N}\\
&\implies (x_{1,i}, x_{2,i}\ldots )\text{ is Convergent w.r.t }d_i\text{ for all }i \in \mathbb{N}\text{  (let }x_l = (x_{l,1},x_{l,2}\ldots )\text{ be the sequence of these limits)}
\end{align*}
It suffices to show that $S$ converges to $x_l$\\
let $\varepsilon > 0$\\
\begin{align*}
&\sum_{i\in \mathbb{N}}{\frac{d_i(x_{n,i},x_{l,i})}{2^i \times (1 + d_i(x_{n,i},x_{l,i}))}}\leq \sum_{i\in \mathbb{N}}\frac{1}{2^i}\text{ for all }n \in \mathbb{N}\\
&\implies\text{ there exists a }k\in \mathbb{N}\text{ such that }(\sum_{i=k+1}^{\infty}{\frac{d_i(x_{n,i},x_{l,i})}{2^i \times (1 + d_i(x_{n,i},x_{l,i}))}}\leq \frac{\varepsilon}{2}\text{ for all }n \in \mathbb{N})\\
&(x_{1,i}, x_{2,i} \ldots )\text{ converges to }x_{l,i}\text{ for all }i\in \mathbb{N}\\
&\implies\text{ there exists an }N \in \mathbb{N}\text{ such that 
}(\text{for all }n \geq N\ \ \ d_i (x_{n,i},x_{l,i}) \leq\frac{\varepsilon}{2k}\text{ for all }i \in \{1,2\ldots k\})\\
&\implies\text{ there exists an }N \in \mathbb{N}\text{ such that for all }n \geq N\ \ (\frac{d_i(x_{n,i},x_{l,i})}{2^i \times (1 + d_i(x_{n,i},x_{l,i}))} \leq \frac{\varepsilon}{2k}\text{ for all }i \in \{1,2\ldots k\})\\
&\implies\text{ there exists an }N \in \mathbb{N}\text{ such that for all }n \geq N
\end{align*}
\begin{align*}
d_\pi (x_n,x_l) &=\sum_{i\in \mathbb{N}}{\frac{d_i(x_{n,i},x_{l,i})}{2^i \times (1 + d_i(x_{n,i},x_{l,i}))}}\\
&=\sum_{i=1}^{k}{\frac{d_i(x_{n,i},x_{l,i})}{2^i \times (1 + d_i(x_{n,i},x_{l,i}))}}+\sum_{i = k+1}^{\infty}{\frac{d_i(x_{n,i},x_{l,i})}{2^i \times (1 + d_i(x_{n,i},x_{l,i}))}}\\
&\leq\Big( \sum_{i=1}^{k}{\frac{d_i(x_{n,i},x_{l,i})}{2^i \times (1 + d_i(x_{n,i},x_{l,i}))}}\Big) +\frac{\varepsilon}{2}\\
&\leq\Big( \sum_{i=1}^{k}{\frac{\varepsilon}{2k}}\Big) +\frac{\varepsilon}{2}\\
&= \varepsilon
\end{align*} 
So $S$ converges to $x_l$ as required. $\qed$
\begin{defn}
A set S is a $G_\delta$ subset of a topological space $(X,T)$ if S is a countable intersection of open sets.
\end{defn}
\begin{theorem}\label{Sym G-delta}
The infinite symmetric group $Sym(\mathbb{N})$ is a $G_\delta$ subset of $(\mathbb{N}^\mathbb{N},T)$ 
\end{theorem}\par
Proof: Let $U:\mathbb{N}\times P(\mathbb{N})\times P(\mathbb{N})\rightarrow P(\mathbb{N})$ be defined by:
$$U(i,S_1,S_2):=
 \left\{
    \begin{array}{lr}
      S_2& i\in S_1 \\
      \mathbb{N}& i\notin S_1
    \end{array}
    \right\}
$$
Note that if $\vert S_1\vert$ is finite then $\prod_{i=1}^{\infty}{U(i,S_1,S_2)}$ is open.\\
For $j,k,l \in \mathbb{N}$:
$$ \prod_{i=1}^{\infty}{U(i,\{j,k\},\{l\})}=\mathbb{N}^\mathbb{N}\backslash
\Big((\prod_{i=1}^{\infty}{U(i,\{j\},\{\mathbb{N}\backslash \{l\}\})})
\cup (\prod_{i=1}^{\infty}{U(i,\{k\},\{\mathbb{N}\backslash \{l\}\})})\Big)$$
Therefore $\prod_{i=1}^{\infty}{U(i,\{j,k\},\{l\})}$ is clopen for all $j,k,l \in \mathbb{N}$.\\
Let $I$ denote the set of all injective functions from $\mathbb{N}$ to $\mathbb{N}$.
$$I=(I^c)^c = \Big(\bigcup_{\substack{j,k,l \in \mathbb{N}\\j\neq k}} \prod_{i=1}^{\infty}{U(i,\{j,k\},\{l\})}\Big)^c= \bigcap_{\substack{j,k,l \in \mathbb{N}\\j\neq k}}\Big( \prod_{i=1}^{\infty}{U(i,\{j,k\},\{l\})}\Big)^c=\bigcap_{\substack{j,k,l \in \mathbb{N}\\j\neq k}}\Big((\prod_{i=1}^{\infty}{U(i,\{j\},\{\mathbb{N}\backslash  \{l\}\})})
\cup (\prod_{i=1}^{\infty}{U(i,\{k\},\{\mathbb{N}\backslash  \{l\}\})})\Big)$$
We have $I$ as a countable intersection of open sets.\\
Let $S$ denote the set of all surjective functions from $\mathbb{N}$ to $\mathbb{N}$.
$$S = \bigcap_{k\in \mathbb{N}}{\Big(\bigcup_{j\in \mathbb{N}}{\prod_{i=1}^{\infty}{U(i,\{j\},\{k\})}}\Big)}$$
We have $S$ as a countable intersection of open sets.\\
Therefore we have $Sym(\mathbb{N}) = S\cap I$ as a countable intersection of open sets 
as required. $\qed$
\begin{theorem} \label{G-delta complete}
A $G_\delta$ subset of a completely metrizable topological space with the subspace topology is completely metrizable. 
\end{theorem}\par
Proof: The following proof comes from \cite{Gdelta}.\\ Let $S= \bigcap_{i\in \mathbb{N}}{U_i}$ for open sets $U_i$, be a $G_\delta$ subset of a completely metrizable topological space $(X, T)$ induced by complete metric space $(X,d)$.\\ Define the map $\phi':S\rightarrow X \times
\mathbb{R} \times \mathbb{R} \ldots $ by
$$(x)\phi' = (x,\frac{1}{d(x,U_1^c)},\frac{1}{d(x,U_2^c)},\ldots )$$
By Theorem \ref{metric product} $(X \times \mathbb{R} \times \mathbb{R} \ldots ,d_\pi)$ is a complete metric space.\\
Define the map $\phi:S\rightarrow image(\phi')$ by
$$(x)\phi = (x)\phi'$$
$\underline{Claim:}\ \phi$ is a homeomorphism.\par
Proof of claim: By construction $\phi$ is surjective, and clearly $\phi$ is also injective. So $\phi$ is a bijection.\\
We need to show $\phi$ is continuous.
Let $\varepsilon>0\ \ x\in S$\\
Let $k\in\mathbb{N}$ be such that $(\sum_{i=k+1}^{\infty}{\frac{1}{2^{i+1}}\Big(\frac{\vert \frac{1}{d(x,U_i^c)}-\frac{1}{d(y,U_i^c)}\vert }{1+\vert \frac{1}{d(x,U_i^c)}-\frac{1}{d(y,U_i^c)}\vert }}\Big)\leq \sum_{i=k+1}^{\infty}{\frac{1}{2^{i+1}}} \leq \frac{\varepsilon}{3})$\\
Let $\delta = \min{\{\frac{\varepsilon}{3},\frac{\min\{d(x,U_i^c):i \in \{1,2 \ldots k\}\}}{2},\frac{\varepsilon \min\{d(x,U_i^c):i \in \{1,2 \ldots k\}\}^2}{6k}\}}$\\
Let $y\in X$
$$d(x,y)<\delta$$
\begin{align*}
\implies d_\pi ((x)\phi,(y)\phi) &=\frac{d(x,y)}{1+d(x,y)}+\sum_{i=1}^{\infty}{\frac{1}{2^{i+1}}\Big(\frac{\vert \frac{1}{d(x,U_i^c)}-\frac{1}{d(y,U_i^c)}\vert}{1+\vert \frac{1}{d(x,U_i^c)}-\frac{1}{d(y,U_i^c)}\vert}}\Big)\\
&\leq d(x,y)+\frac{\varepsilon}{3} + \sum_{i=1}^{k}{\frac{1}{2^{i+1}}\Big(\frac{\vert\frac{d(y,U_i^c)-d(x,U_i^c)}{d(x,U_i^c)d(y,U_i^c)}\vert}{1+\vert \frac{1}{d(x,U_i^c)}-\frac{1}{d(y,U_i^c)}\vert}}\Big)\\
&\leq \frac{\varepsilon}{3}+\frac{\varepsilon}{3} + \sum_{i=1}^{k}{\vert\frac{d(y,U_i^c)-d(x,U_i^c)}{d(x,U_i^c)d(y,U_i^c)}\vert}\\
&\leq \frac{\varepsilon}{3}+\frac{\varepsilon}{3} + \sum_{i=1}^{k}{\frac{d(x,y)}{d(x,U_i^c)d(y,U_i^c)}}\\
&\leq \frac{\varepsilon}{3}+\frac{\varepsilon}{3} + \sum_{i=1}^{k}{\frac{d(x,y)}{d(x,U_i^c)(d(x,U_i^c)-\delta)}}\\
&\leq \frac{\varepsilon}{3}+\frac{\varepsilon}{3} + \sum_{i=1}^{k}{\frac{2d(x,y)}{d(x,U_i^c)d(x,U_i^c)}}\\
&\leq \frac{\varepsilon}{3}+\frac{\varepsilon}{3} + \sum_{i=1}^{k}{\frac{2\delta}{d(x,U_i^c)d(x,U_i^c)}}\\
&\leq \frac{\varepsilon}{3}+\frac{\varepsilon}{3} + \sum_{i=1}^{k}{\frac{2\frac{\varepsilon \min\{d(x,U_i^c):i \in \{1,2 \ldots k\}\}^2}{6k}}{d(x,U_i^c)^2}}\\
&\leq \frac{\varepsilon}{3}+\frac{\varepsilon}{3} + \sum_{i=1}^{k}{\frac{\varepsilon}{3k}}\\
&\leq \varepsilon
\end{align*}
Finally we need to show $\phi^{-1}$ is continuous.
Let $\varepsilon>0\ \ x\in image(\phi)$\\
Let $\delta = \min \{\frac{\varepsilon}{2},\frac{1}{2}\}$\\
Let $y \in image (\phi)$.
\begin{align*}
&d_\pi (x,y) < \delta\\
&\implies \frac{d((x)\phi^{-1},(y)\phi^{-1})}{1+d((x)\phi^{-1},(y)\phi^{-1})}<\delta\\
&\implies d((x)\phi^{-1},(y)\phi^{-1})<\delta (1+d((x)\phi^{-1},(y)\phi^{-1}))\\
&\implies d((x)\phi^{-1},(y)\phi^{-1})(1-\delta)<\delta\\
&\implies d((x)\phi^{-1},(y)\phi^{-1})<\frac{\delta}{1-\delta}\leq \frac{\delta}{\frac{1}{2}}\leq\frac{\frac{\varepsilon}{2}}{\frac{1}{2}}= \varepsilon\qed
\end{align*}
We have that S is homeomorphic to $image(\phi)$ which is contained in a complete metric space. So to show S is completely metrizable it suffices to show that $image(\phi)$ is closed and therefore complete.\\
Let $S = ((x_1)\phi,(x_2)\phi, \ldots )$ be a sequence in $image(\phi)$ which converges in $X \times
\mathbb{R} \times \mathbb{R} \ldots $ to $(x,r_1,r_2 \ldots)$.\\
\begin{align*}
&\lim_{n\rightarrow \infty}{\frac{1}{d(x_n,U_i^c)}=r_i}\ \ \ \text{for all } i \in \mathbb{N} \text{  as if this were not the case S would not converge}\\
&\implies {\frac{1}{d( \lim_{n\rightarrow \infty}x_n,U_i^c)}=r_i}\ \ \ \text{for all } i \in \mathbb{N}\text{  as d and 1/x are continuous functions}\\
&\implies \frac{1}{d(x,U_i^c)}=r_i\ \ \ \text{for all } i \in \mathbb{N}\text{  as if }\lim_{n\rightarrow\infty}{x_n}\neq x \text{ then S wouldn't converge}\\
&\implies d(x,U_i^c) \neq 0\ \text{ for all } i \in \mathbb{N}\\
&\implies x \notin U_i^c\ \text{  for all } i \in \mathbb{N}\\
&\implies x \in U_i\ \text{  for all } i \in \mathbb{N}\\
&\implies x \in S
\end{align*}
As $x \in S$ and ${\frac{1}{d(x,U_i^c)}=r_i}$   for all $i \in \mathbb{N}$ we have that $(x,r_1,r_2,\ldots ) = (x)\phi \in image(\phi)$ as required. $\qed$
\begin{theorem}
Let $T_s$ be the subspace topology of $Sym(\mathbb{N})$ in $(\mathbb{N}^\mathbb{N},T)$. The topological space $(Sym(\mathbb{N}),T_s)$ is completely metrizable.
\end{theorem}\par
Proof: By theorems \ref{metric product}, \ref{Sym G-delta} and \ref{G-delta complete} we have that $(\mathbb{N}^\mathbb{N},T)$ is completely metrizable, $Sym(\mathbb{N})$ is $G_\delta$ in $(\mathbb{N}^\mathbb{N},T)$ and $G_\delta$ sets of a completely metrizable topology equipped with the subspace topology are completely metrizable. It therefore follows that $(Sym(\mathbb{N}),T_s)$ is completely metrizable.
\section{Topological groups and the baire category theorem}
Now that we have a topology on $Sym(\mathbb{N})$ we will show that we have a topological group and use the baire category theorem to prove the main result of this chapter.
\begin{defn}
A topological group $(X,*,T)$ is a group with a topology defined on it such that the functions: $\phi_1:X\times X \rightarrow X$ and $\phi_2:X \rightarrow X$ defined by:
\[\phi_1(x,y) := x*y \ \ \ \ (x)\phi_2 := x^{-1}\]
are both continuous.
\end{defn}
\begin{theorem}
The infinite symmetric group with the subspace topology $(Sym(\mathbb{N}),\circ,T_s)$ is a topological group.
\end{theorem}\par
Proof: Let $\sigma \in \mathbb{N}^{<\mathbb{N}}$
\[[\sigma]_s := \{f\in Sym(\mathbb{N}) : f\vert_{dom(\sigma)} = \sigma\}\]
As B is a basis for T we have that $B_s :=\{ b \cap Sym(\mathbb{N}):b \in B \}= \{[\sigma]_s:\sigma \in \mathbb{N}^{<\mathbb{N}}\}$ is a basis for $T_s$.
If therefore suffices to show that $(b)\phi_1^{-1}$ and $(b)\phi_2^{-1}=(b)\phi_2$ are open for all $b \in B_s$.\\
Let $b\in B_s$\\
Let $b=[\sigma]_s$\\
Let $dom(\sigma) = \{1,2\ldots k\}$
\[(f,g)\in (b)\phi_1^{-1} \implies fg\in b=[\sigma]_s \implies ((i)f)g = (i)\sigma\ \text{  for all } i\in \{1,2 \ldots k\}
\]
Let $U_{f,g} := [f\vert_{\{1,2 \ldots k\}}]_s\times[g\vert_{\{1,2 \ldots \max\{(i)f:i\in \{1,2 \ldots k\}\}\}}]_s$\\
Clearly $(f,g)\in U_{f,g}$ and $U_{f,g}$ is open.\\
Let $(f_2,g_2)\in U_{f,g}$
\begin{align*}
&((i)f_2)g_2 = ((i)f)g_2\ \text{  for all } i \in \{1,2 \ldots k\}\text{ as }f_2\vert_{\{1,2 \ldots k\}}=f\vert_{\{1,2 \ldots k\}}\\
&\implies ((i)f_2)g_2 = ((i)f)g\ \text{ for all } i \in \{1,2 \ldots k\} \text{ as }g_2\vert_{\{1,2 \ldots \max\{(i)f:i\in \{1,2 \ldots k\}\}\}}=g\vert_{\{1,2\ldots \max\{(i)f:i\in \{1,2 \ldots k\}\}\}}\\
&\implies ((i)f_2)g_2=(i)\sigma \text{  for all } i \in \{1,2 \ldots k\}=dom(\sigma)\\
&\implies (f_2,g_2)\in ([\sigma]_s)\phi_1^{-1}=(b)\phi_1^{-1}\\
&\implies U_{f,g}\subseteq (b)\phi_1^{-1}
\end{align*}
So we have that for all $(f,g)\in (b)\phi_1^{-1}$ there is an open set $U_{f,g}$ such that $(f,g)\in U_{f,g}\subseteq (b)\phi_1^{-1}$ and therefore $(b)\phi_1^{-1}$ is open as required.
\[f \in (b)\phi_2 \implies f^{-1} \in [\sigma]_s \implies (i)f^{-1}=(i)\sigma \ \text{ for all } i \in \{1,2\ldots k\} \implies i=((i)\sigma)f \ \text{ for all } i \in \{1,2 \ldots k\}\]
Let $U_f := [f\vert_{\{1,2 \ldots max(image(\sigma))\}}]_s$\\
Clearly $f\in U_f$ and $U_f$ is open.
It therefore suffices to show that $U_f\subseteq (b)\phi_2$.\\
Let $f_2\in U_f$\\
\begin{align*}
&((i)\sigma)f_2 = ((i)\sigma)f\ \text{ for all } i \in \{1,2 \ldots k\}\ \ \ \  \text{      as }f_2\vert_{\{1,2 \ldots max(image(\sigma))\}}=f\vert_{\{1,2 \ldots max(image(\sigma))\}}\\
&\implies ((i)\sigma)f_2 = i\ \text{ for all } i \in \{1,2 \ldots k\}\\
&\implies (i)\sigma = (i)f_2^{-1}\ \text{ for all } i \in \{1,2 \ldots k\}\\
&\implies f_2^{-1}\in [\sigma]_s\\
&\implies (f_2^{-1})\phi_2 \in ([\sigma]_s)\phi_2=(b)\phi_2\\
&\implies f_2 \in (b)\phi_2\\
&\implies U_{f}\subseteq (b)\phi_2 \text{ as required.}\qed
\end{align*}
\begin{theorem}\label{homeomorphic multiplication}
Let $(X,*,T)$ be a topological group, we have that for all $x \in X$ the maps given by:
\[(y)\phi_{x_r}:=yx \ \ \ \ (y)\phi_{x_l}:=xy\]
are homeomorphisms.
\end{theorem}\par
Proof: By standard properties of groups we have that $\phi_{x_r}$ and $\phi_{x_l}$ are bijections. In addition it is clear by symmetry that if $\phi_{x_r}$ is continuous then $\phi_{x_r}^{-1}, \phi_{x_l}$ and $\phi_{x_l}^{-1}$ are continuous.\\
Let $U$ be open.\\
We have that $(U)*^{-1}$ is open.\\
By definition of the product topology, this means that there exists a collection of open sets $B$ such that $\cup B = (U)*^{-1}$ and for all $b \in B$  we have $b=U_{b,1}\times U_{b,2}$ for open sets  $U_{b,1}$ and $U_{b,2}$. As $U_{b,1}$ is open for all $b \in B$, it suffices to show that $(U)\phi_{x_r}^{-1}=\cup\{U_{b,1}:x \in U_{b,2}\}$.
\begin{align*}
&y \in (U)\phi_{x_r}^{-1}\\
&\iff yx \in U\\
&\iff (y,x) \in (U)*^{-1}\\
&\iff (y,x) \in \cup B\\
&\iff \text{there exists } b \in B \text{ such that } (y,x) \in b\\
&\iff \text{there exists } b \in B \ \ \text{such that} \ \ y \in U_{b,1}\ \ and\ \  x\in U_{b,2}\\
&\iff y \in \cup\{U_{b,1}:x \in U_{b,2}\}
\end{align*}
We therefore have that $(U)\phi_1^{-1}=\cup\{U_{b,1}:x \in U_{b,2}\}$ as required. $\qed$
\begin{defn}
A subset $D$ of a topological space $(X,T)$ is called \textit{dense(in X)} if for all $U \subseteq X$ such that $U$ is open and non-empty $D\cap U \neq \emptyset$. Note this is equivalent to saying all points of $X$ are limit points of $D$.
\end{defn}
\begin{defn}
In a topological space, a \textit{nowhere-dense} set $N$, is one satisfying: $(\overline{N})^{\circ}=\emptyset$.
\end{defn}
\begin{theorem}
Let $(X,T)$ be a topological space, $N$ be nowhere-dense. Then $N^c$ is dense in $X$.
\end{theorem}\par
Proof: Suppose for a contradiction that $U$ be a non-empty open set satisfying $U \cap N^c = \emptyset$.
\begin{align*}
&U \cap N^c = \emptyset\\
&\implies U\backslash N = \emptyset\\
&\implies U \subseteq N\\
&\implies U \subseteq \overline{N}\\
&\implies U \subseteq (\overline{N})^\circ \ \ \text{as U is open}\\
&\implies (\overline{N})^\circ \neq \emptyset
\end{align*}
This contradicts the nowhere-denseness of N. $\qed$
\begin{theorem}
Let $(X,T)$ be a topology induced by a metric d. A set $N$ is nowhere-dense iff for all $x_1 \in X$ and $r_1>0$. There exists an $x_2\in X$ and an $r_2>0$ such that $B(x_2,r_2)\subseteq B(x_1,r_1)\backslash N$
\end{theorem}\par
Proof:
$(\implies)$ Let $N$ be nowhere-dense.\\ Suppose there exists an $x_1 \in X$ and an $r_1>0$ such that there are no $x_2\in X$ and $r_2>0$ such that $B(x_2,r_2)\subseteq B(x_1,r_1)\backslash N$\\
Let $x\in B(x_1,r_1)$.\\ As $B(x_1,r_1)$ is open there exists an $r>0$ such that $B(x,r)\subseteq B(x_1,r_1)$\\
Let $r>0$ be such that $B(x,r)\subseteq B(x_1,r_1)$\\
By assertion $B(x,r)\nsubseteq B(x_1,r_1)\backslash N$\\
So there is a $y\in N\cap B(x,r)$ but as $r$ can be made arbitrarily small we therefore have that $x$ is a limit point of $N$.\\
As all elements of $B(x_1,r_1)$ are limit points of $N$ it follows that $B(x_1,r_1)\subseteq \overline{N}$ and therefore $B(x_1,r_1)\subseteq (\overline{N})^\circ$.\\
So $x_1\in (\overline{N})^\circ$. This contradicts the nowhere-denseness of $N$\\
$(\impliedby)$ Suppose $(\overline{N})^\circ \neq \emptyset$. \\
Let $x\in (\overline{N})^\circ$.\\ As $(\overline{N})^\circ$ is open there exists an $r>0$ such that $B(x,r)\subseteq (\overline{N})^\circ$.\\
By assertion there exists an $x_2\in X$ and an $r_2>0$ such that $B(x_2,r_2)\subseteq B(x,r)\backslash N\subseteq (\overline{N})^\circ\backslash N\subseteq \overline{N}\backslash N$\\
Therefore $x_2 \in B(x_2,r_2)\subseteq  \overline{N}\backslash N$ is a limit point of $N$.\\
But $B(x_2,r_2)$ is open and contains $x_2$ and $B(x_2,r_2)\subseteq B(x,r)\backslash N$ so $B(x_2,r_2)$ contains no elements of $N$.\\This is a contradiction. $\qed$
\begin{defn}
In a topological space, a \textit{meagre} set $M$ is one which can be expressed as $M=\bigcup_{i\in \mathbb{N}}N_i$, where $(N_i)_{i\in \mathbb{N}}$ is a family of nowhere-dense sets.
\end{defn}
\begin{defn}
In a topological space, a \textit{comeagre} set is one who's complement is meagre.
\end{defn}
\begin{defn}
A \textit{Baire Space} is a topological space in which any countable collection of dense open sets $(U_n)_{n\in \mathbb{N}}$ has dense intersection.
\end{defn}
\bigskip
The next pairs of definitions and theorems are introduced only for the proof of the second part of the baire category theorem. This part of the baire category theorem is not used anywhere in this document but is left in for interest.
\begin{defn}
A \textit{neighbourhood} of point $x$, in a topological space $(X,T)$, is a set $N$ such that $x \in U \subseteq N$ for some open set $U$. 
\end{defn}
\begin{defn}
A topological space $(X,T)$ is called \textit{locally compact} if every point in $X$ has a compact neighbourhood.
\end{defn}
\begin{theorem}\label{non-empty closed intersection}
Let $(X,T)$ be a compact topological space. Let $C_1 \supseteq C_2 \supseteq C_3 \ldots$ be a decreasing sequence of non-empty closed sets. Then \(\bigcap_{i\in \mathbb{N}}C_i \neq
\emptyset\).
\end{theorem}\par
Proof: Suppose for a contradiction that:
$$\bigcap_{i\in \mathbb{N}}C_i =
\emptyset$$
We therefore have:
\begin{align*}
X\backslash \bigcap_{i\in \mathbb{N}}C_i =X &\implies\big(\bigcap_{i\in \mathbb{N}}C_i\big)^c =X\\
&\implies\bigcup_{i\in \mathbb{N}}C_i^c =X\\
&\implies \bigcup_{i\in \{1,2\ldots k\}}C_i^c =X\ \ \ \text{for some k (this follows by compactness)}\\
&\implies \bigcup_{i\in \{1,2\ldots k\}}C_i^c = \bigcup_{i\in \mathbb{N}}C_i^c\\
&\implies \bigcap_{i\in \{1,2\ldots k\}}C_i = \bigcap_{i\in \mathbb{N}}C_i\\
&\implies C_k = \bigcap_{i\in \mathbb{N}}C_i\\
&\implies \bigcap_{i\in \mathbb{N}}C_i \neq \emptyset
\end{align*}
A contradiction. $\qed$
\begin{theorem}\label{lch open closed open}
Let $(X,T)$ be a locally compact Hausdorff space. Let $U_1$ be a non-empty open set:
there exists a non-empty open set $U_2$, a closed set $C$ and a compact set $C_x$ such that $U_2\subseteq C\subseteq U_1\cap C_x$.
\end{theorem}\par
Proof: Let x be a point in $U_1$. As $(X,T)$ is locally compact there exists a compact neighbourhood $C_x$ of $x$.\\
We have that $x\in U_3 \subseteq C_x$ for some open set $U_3$. Let $U_4:= U_3\cap U_1$. We have that $U_4$ is a non-empty open set as it is the intersection of two open sets and it contains $x$.\\
As $(X,T)$ is Hausdorff, for all $y \in C_x\backslash U_4$ there exists open sets $V_{y,1}$ and $V_{y,2}$ such that $V_{y,1}\cap V_{y,2} = \emptyset$, $x\in V_{y,1}$ and $y\in V_{y,2}$.\\
We have that $\{U_4\}\cup\{V_{y,2}:y\in C_x\backslash U_4\}$ is an open cover of $C_x$. By compactness it has a finite subcover. So we have that $C_x \subseteq \bigcup\{U_4,V_{y_1,2},V_{y_2,2},\ldots V_{y_k,2}\}$ for some $y_1,y_2,\ldots ,y_k \in C_x\backslash U_4$.\\
We therefore have that $$C_x \backslash U_4 \subseteq \bigcup\{V_{y_1,2},V_{y_2,2},\ldots V_{y_k,2}\}$$
$$\implies U_4 \supseteq C_x \backslash \bigcup\{V_{y_1,2},V_{y_2,2},\ldots V_{y_k,2}\}=: C$$
As $C_x$ is compact and $(X,T)$ is Hausdorff it follows that $C_x$ is closed, so $C_x^c$ is open. Therefore C is also closed as:
$$C=\big(\bigcup\{C_x^c,V_{y_1,2},V_{y_2,2},\ldots V_{y_k,2}\}\big)^c$$
Let $U_2 := \bigcap\{U_4, V_{y_1,1},V_{y_2,1},\ldots V_{y_k,1}\}$. $U_2$ is open as the intersection of finitely many open sets and is non-empty as it contains $x$.\\
We now show that $U_2 \subseteq C$.\\ Let $z\in U_2$.\\
$z\in U_2\subseteq U_4 \subseteq U_3 \subseteq C_x$. So $z \in C_x$.\\
for all $i\in \{1,2\ldots k\}$ we have : $z \in V_{y_i,1} \implies z \notin V_{y_i,2}$.\\
We therefore have that $z \in C$.\\
So we have that $U_2\subseteq C$.
\[x\in U_2 \subseteq C \subseteq U_4 =U_3\cap U_1 \subseteq C_x \cap U_1\]
This gives the required result. $\qed$
\begin{theorem}
Baire category theorem\\
1. Every completely metrizable topological space is a Baire Space.\\
2. Every locally compact Hausdorff space is a Baire space.\\
3. A non-empty completely metrizable space cannot be expressed as a countable union of nowhere-dense sets, and is therefore not meagre.
\end{theorem}\par
Proof of part 1: The following proof is an adapted version of the proof of the metric version of the Baire category theorem found in \cite{4004}.\\ Let $(X,T)$ be a completely metrizable topological space with complete metric $d$.\\
Let $(U_n)_{n\in \mathbb{N}}$ be a countable collection of open sets.\\
WTS:  $I=\bigcap_{i\in \mathbb{N}}{U_i}$ is dense in X.\\
Let $x\in X\backslash I$.\\
WTS: $x$ is a limit point of $I$.\\
Let $U_x$ be an open neighbourhood of $x$.\\
It suffices to show that $I\cap U_x \neq \emptyset$\\
As $T$ is a metric topology there exists an $r_0>0$ such that $B(x,r_0)\subseteq U_x$\\
Let $B_0=V_0=B(x,r_0)$ and $V_n=U_n\cap B_{n-1}\text{ for all } n \in \mathbb{N}$\\ and $B_n=B(x_n,r_n)\ \text{for all } n \in \mathbb{N}$ be such that $B_n\subseteq B(x_n,2\times r_n) \subseteq V_n$ and $0<r_n<\frac{r_{n-1}}{2}$\\
Note that this construction is possible as each $V_i$ is constructed by intersecting an open set with an open set, so each $V_i$ is open.\\
In addition $V_i$ is non-empty as the intersection of a dense set and an open set, and by the axiom of choice the sequence $S = (x_1,x_2,x_3,x_4 \ldots)$ must be constructable.\\
We will now show that S in Cauchy.\\
Let $\varepsilon > 0$, choose $N\in \mathbb{N}$ such that $$\frac{r_0}{2^{N}} < \frac{\varepsilon}{2}$$
Let $n \geq N$:\\
We have $x_n\in B_n \subseteq V_n \subseteq B_{n-1} \ldots \ldots \subseteq B_{N}=B(x_{N},r_{N})\subseteq B(x_{N},\frac{r}{2^{N}})$. This final inclusion follows as each B has radius less than half the size of the previous one.\\
$\implies d(x_n,x_N) < \frac{r_0}{2^N} < \frac{\varepsilon}{2}$\\
So for $n,m\geq N$:
$$d(x_n,x_m)\leq d(x_n,x_N)+d(x_m,x_N)$$
$$< \frac{\varepsilon}{2}+\frac{\varepsilon}{2}=\varepsilon$$
We therefore have that S is Cauchy.\\
As $(X,d)$ is complete we have that S is convergent.\\
Let $y = \lim_{n\rightarrow \infty} x_n$.\\
It suffices to show that $y\in U_x$ and $y\in I$.\\
for all $i \in \mathbb{N}$:
$x_i\in B_i \subseteq B_{i-1} \ldots \subseteq B_1\subseteq \overline{B_1} = \{z \in X: d(x_1,z) \leq r_1\} \subseteq \{z \in X: d(x_1,z) < 2\times r_1\} \subseteq V_1\subseteq B_0$.\\
As for all $i \in \mathbb{N}\ \ \ x_i \in \overline{B_1}$ a closed set. We have that $y\in \overline{B_1}\subseteq B_0 \subseteq U_x$\\
We now show that $y\in I$\\
To do this we show that for all $k \in \mathbb{N}$  $y\in U_k$.\\
Let $k\in \mathbb{N}$. The sequence $(x_{k+1},x_{k+2}, \ldots)$ converges to $y$.\\
for all $i > k$:
$x_i\in B_i \subseteq B_{i-1} \ldots \subseteq B_{k+1}\subseteq \overline{B_{k+1}} = \{z \in X: d(x_{k+1},z) \leq {r_{k+1}}\} \subseteq \{z \in X: d(x,z) < 2\times r_{k+1}\} \subseteq V_{k+1} \subseteq B_k$.\\
As for all $i > k$ we have  $x_i \in \overline{B_{k+1}}$ a closed set. We have that $y\in \overline{B_{k+1}}\subseteq B_k \subseteq V_k \subseteq U_k$\\
We therefore have that $y\in \bigcap_{i\in\mathbb{N}}U_i = I$ as required. $\qed$\\
Proof of part 2: Let $(X,T)$ be a locally compact Hausdorff space.\\
Let $(U_n)_{n\in \mathbb{N}}$ be a countable collection of open sets.\\
WTS:  $I=\bigcap_{i\in \mathbb{N}}{U_i}$ is dense in X.\\
Let $x\in X\backslash I$.\\
WTS: $x$ is a limit point of $I$.\\
Let $U_x$ be an open neighbourhood of $x$.\\
It suffices to show that $I\cap U_x \neq \emptyset$\\
As $U_1$ is dense $U_x\cap U_1\neq \emptyset$\\
By the Theorem \ref{lch open closed open} construct a non-empty open set $V_1$, a closed set $C_1$ and a compact set $C_x$ such that $V_1 \subseteq C_1 \subseteq (U_x \cap U_1) \cap C_x$\\
Similarly for all $i>1$ construct $V_i$ and $C_i$ such that $V_i$ is open and non-empty, $C_i$ is closed and $$V_i\subseteq C_i \subseteq U_i \cap V_{i-1}$$
Note that $C_x\cap U_x \supseteq C_1\supseteq C_2 \supseteq C_3 \ldots$\\
By considering the subspace topology on $C_x$ and by applying Theorem \ref{non-empty closed intersection} we have that $\bigcap_{i\in \mathbb{N}} C_i \neq \emptyset$.\\
$$\emptyset \neq \bigcap_{i\in \mathbb{N}} C_i \subseteq \bigcap_{i\in \mathbb{N}} U_i\cap U_x = U_x\cap I$$
$$\implies \emptyset \neq U_x\cap I \qed$$
Proof of part 3: Suppose $(X,T)$ is a non-empty completely metrizable space such that $X=\bigcup_{i\in \mathbb{N}} N_i$($=\bigcup_{i\in \mathbb{N}} \overline{N_i}$) where $N_i$ is nowhere-dense for all $i \in \mathbb{N}$.\\
As $(X,T)$ is Completely metrizable we have that $(X,T)$ is a Baire space by part 1.\\
consider the sets $(\overline{N_i}^c)_{i\in \mathbb{N}}$.\\
As the complements of closed nowhere-dense sets, these sets are open and dense.\\
Therefore we have the following:
\[\bigcap_{i \in \mathbb{N}} \overline{N_i}^c \text{  is dense}\implies \Big(\Big(\bigcap_{i \in \mathbb{N}} \overline{N_i}^c\Big)^c\Big)^c \text{  is dense}\implies \Big(\bigcup_{i \in \mathbb{N}} \overline{N_i}\Big)^c \text{  is dense}\implies (X)^c \text{  is dense}\implies \emptyset \text{  is dense}\]
As $(X,T)$ is a non-empty topology. $X$ is a non-empty open set.\\
But $X\cap \emptyset=\emptyset$. This contradicts the denseness of $\emptyset$. $\qed$
\section{Comeagre conjugacy class}
For the rest of this chapter we will construct a comeagre conjugacy class of $Sym(\mathbb{N})$ and use it to show the desired result that all elements are commutators.
\begin{defn}
Let the set $F$ be defined by:
$$F := \{f \in Sym(\mathbb{N}): \text{f has infinitely many cycles of all finite lengths but none of infinite length}\}$$
Notice that by theorem \ref{cycle conjugate} $F$ is a conjugacy class.
\end{defn}
\begin{theorem}
Let $I=(i_1,i_2,\ldots i_k)$ and $N=(n_1,n_2,\ldots n_k)$ be finite sequences of natural numbers  with no repeats.\\
there exists a $\sigma \in Sym(\mathbb{N})$ and an $r>0$ such that $(n_j)f=i_j\ \text{ for all } f \in B(\sigma,r)\ \text{ for all } j\in \{1,2\ldots k\}$
\end{theorem}\par
Proof: As both $I$ and $N$ are finite. We have that: $$\vert\mathbb{N}\backslash I\vert=\vert \mathbb{N} \vert = \vert\mathbb{N}\backslash N\vert$$
Therefore there exists a bijection $\phi :\mathbb{N} \backslash N \rightarrow \mathbb{N}\backslash I$\\
let $\sigma$ be defined by:
$$(n)\sigma=
 \left\{
    \begin{array}{lr}
      i_j& \text{if }n = n_j\text{ for some }j\in \{1,2,\ldots k\} \\
      (n)\phi&  \text{otherwise}
    \end{array}
    \right\}
$$
By construction $\sigma \in Sym(\mathbb{N})$.\\
Let $r = \frac{1}{max(N)}$ and $f\in B(\sigma, r)$
\begin{align*}
&d(\sigma, f) < \frac{1}{max(N)}\\
&\implies \frac{1}{min(\{n\in \mathbb{N}:(n)\sigma \neq  (n)f\})} < \frac{1}{max(N)}\\
&\implies max(N) < min(\{n\in \mathbb{N}:(n)\sigma \neq  (n)f\})\\
&\implies (n_j)f = (n_j)\sigma \text{ for all } j \in \{1,2,\ldots k\}\\
&\implies (n_j)f = i_j \text{ for all } j \in \{1,2,\ldots k\} \qed
\end{align*}
\begin{theorem}\label{nowhere-dense inf}
Let $F_{1,i}$ be defined by:
$$F_{1,i}=\{f \in Sym(\mathbb{N}): \text{f has a cycle of infinite length involving the number i}\}$$
$F_{1,i}$ is nowhere-dense for all $i \in \mathbb{N}$
\end{theorem}\par
Proof: Let $x_1 \in Sym(\mathbb{N})$ and $r_1>0$.\\
Let $x_r:=x_1\vert_{\{1,2 \ldots \lfloor \frac{1}{r_1}\rfloor\}}$\\
$B(x_1,r_1)= \{f \in Sym(\mathbb{N}): f\vert_{\{1,2 \ldots  \lfloor \frac{1}{r_1}\rfloor\}}=x_r\}$\\
It suffices to find $x\in X$ and $r>0$ such that  $B(x,r)\subseteq B(x_1,r_1)\backslash F_{1,i}$\\
let $$m_1:= min(j\in \mathbb{N}_0:(i)x_1^j\notin dom(x_r))$$
$$m_2:= min(j\in \mathbb{N}_0:(i)x_1^{-j}\notin image(x_r))$$
Noting if $m_1$ and $m_2$ don't exist then  $B(x_1,r_1)\subseteq B(x_1,r_1)\backslash F_{1,i}$ and we are done.\\
Let $\sigma$ be a bijection satisfying:
$$(j)\sigma=(j)x_1\ \text{ for all } j \in dom(x_r)$$
$$((i)x_1^{m_1})\sigma=(i)x_1^{-m_2}$$
Let $r > 0$ be such that for all $f \in B(\sigma,r)$\\
$$(j)\sigma=(j)f\ \text{ for all } j \in dom(x_r)$$
$$((i)x_1^{m_1})\sigma=((i)x_1^{m_1})f$$
Note that for all $f \in B(\sigma,r)$ $orb_f(i)$ is finite and therefore $f \notin F_{1,i}$.\\
Therefore we have that $B(\sigma,r)\subseteq B(x_1,r_1)\backslash F_{1,i}$ as required.$\qed$
\begin{theorem} \label{nowhere-dense finite}
Let $F_{2,i,j}$ be defined by:
$$F_{2,i,j} = \{f \in Sym(\mathbb{N}): \textsl{f has i cycles of length j}\}$$
$F_{1,i,j}$ is nowhere-dense for all $i \in \mathbb{N}_0$ and $j \in \mathbb{N}$
\end{theorem}\par
Proof: Let $x_1 \in Sym(\mathbb{N})$ and $r_1>0$.\\
Let $x_r:=x_1\vert_{\{1,2 \ldots \lfloor \frac{1}{r_1}\rfloor\}}$\\
$B(x_1,r_1)= \{f \in Sym(\mathbb{N}): f\vert_{\{1,2 \ldots \lfloor \frac{1}{r_1}\rfloor\}}=x_r\}$\\
It suffices to find $x\in X$ and $r>0$ such that $B(x,r)\subseteq B(x_1,r_1)\backslash F_{2,i,j}$\\
Let $$m := max(dom(x_r)\cup image(x_r))$$
Let $\sigma\in Sym(\mathbb{N)}$ be such that:
\begin{align*}
(k)\sigma&=(k)x_1\ \textsl{ for all } k \in dom(x_r)\\
(m+1)\sigma &=(m+2), (m+2)\sigma = (m+3) \ldots (m+j)\sigma = (m+1),\\
(m+j+1)\sigma &= (m+j+2)\ldots,(m+2j)\sigma=(m+j+1),\\
\vdots\\
(m+ij+1)\sigma &= (m+ij+2)\ldots (m+(i+1)j)\sigma=(m+ij+1)
\end{align*}
Let $r>0$ be such that for all $f \in B(\sigma, r)$:
$$(k)f=(k)\sigma\ \ \textsl{ for all } k \in \{1,2\ldots (m+(i+1))\}$$
For all $f\in B(\sigma, r)$ $f$ has at least $(i+1)$ cycles of length $k$ and therefore $f\notin F_{2,i,k}$. However $B(\sigma, r) \subseteq B(x_1,r_1)$.
Therefore we have $B(\sigma,r)\subseteq B(x_1,r_1)\backslash F_{2,i,j}$ as required. $\qed$
\begin{theorem}\label{F is comeagre}
The set $F$ is comeagre.
\end{theorem}\par
Proof: Let $F_1$ and $F_2$ are defined by:
\[F_1 = \{f \in Sym(\mathbb{N}): \textsl{f has a cycle of infinite length}\}\ \ \ F_2 = \{f \in Sym(\mathbb{N}): \textsl{f has finitly many cycles of some finite length }\}\]
Let $F_{2,i}$ be defined by:
$$F_{2,i} = \{f \in Sym(\mathbb{N}): \textsl{f has i cycles of some finite length }\}$$
Note the following:
\[F^c = F_1 \cup F_2 \ \ \ \ F_1 = \bigcup_{i \in \mathbb{N}}F_{1,i}\ \ \ \ F_2 = \bigcup_{i \in \mathbb{N}_0}F_{2,i}\ \ \ \ F_{2,i}=\bigcup_{j \in \mathbb{N}}F_{2,i,j}\]
We therefore have that:
\[F^c = F_1 \cup F_2= \big(\bigcup_{i \in \mathbb{N}}F_{1,i}\big) \cup \big(\bigcup_{i \in \mathbb{N}_0}F_{2,i}\big)= \big(\bigcup_{i \in \mathbb{N}}F_{1,i}\big) \cup \big(\bigcup_{i \in \mathbb{N}_0}\big( \bigcup_{j \in \mathbb{N}}F_{2,i,j} \big)\big)\]
By Theorems \ref{nowhere-dense inf} and \ref{nowhere-dense finite} all the $F_{1,i}$ and $F_{2,i,j}$ are nowhere-dense. Therefore $F^c$ is a countable union of nowhere-dense sets and is therefore meagre. So we have that $(F^c)^c = F$ is comeagre as required.
$\qed$
\begin{theorem}\label{comeagre intersection}
Let $(X,T)$ be a topological space. Let $M_1^c, M_2^c \ldots$ be comeagre. Then $\cap_{i \in \mathbb{N}}M_i$ is comeagre.
\end{theorem}\par
Proof: for all $i \in \mathbb{N}$ we have:
$$M_i^c = (\bigcup_{j \in \mathbb{N}}{N_{i,j}})^c$$
for nowhere dense sets $N_{i,j}$.
We therefore have:
\[\bigcap_{i\in \mathbb{N}}{M_i^c}=\big(\big(\bigcap_{i\in \mathbb{N}}{M_i^c}\big)^c\big)^c=\big(\big(\bigcap_{i\in \mathbb{N}}{(\bigcup_{j \in \mathbb{N}}{N_{i,j}})^c}\big)^c\big)^c=\big(\bigcup_{i\in \mathbb{N}}{\bigcup_{j \in \mathbb{N}}{N_{i,j}}}\big)^c=\big(\bigcup_{i,j\in \mathbb{N}}{N_{i,j}}\big)^c\]
This set is clearly comeagre as required. $\qed$
\begin{theorem}\label{comeagre image}
The homeomorphic image of a comeagre set is comeagre.
\end{theorem}\par
Proof: Let $M^c=(\bigcup_{i\in \mathbb{N}}N_i)^c$ be a comeagre set in a topological space $(X_1,T_1)$ (where $N_i$ is nowhere dense for all $i$). Let $\phi:(X_1,T_1)\rightarrow (X_2,T_2)$ be a homeomorphism.
\[(M^c)\phi = (M\phi)^c= ((\bigcup_{i \in \mathbb{N}}N_i)\phi)^c= (\bigcup_{i \in \mathbb{N}}(N_i)\phi)^c\]
As homeomorphisms preserve closures and interiors we have that the image of a nowhere-dense set is nowhere-dense. Therefore this is the complement of a countable union of nowhere-dense sets as required. $\qed$
\begin{theorem}\label{temp final}
$FF =Sym(\mathbb{N})$ and in particular all functions $f\in Sym(\mathbb{N})$ can be written in the form:
$$f=[g,h]=ghg^{-1}h^{-1}$$
where $g,h\in Sym(\mathbb{N})$.
\end{theorem}\par
Proof: Let $f\in Sym(\mathbb{N})$. By Theorems \ref{F is comeagre}, \ref{comeagre image} and \ref{homeomorphic multiplication} we have that $F$ is comeagre and $(F)f$ is comeagre. By theorem \ref{comeagre intersection} $F\cap (F)f$ is comeagre. So by the baire catagory theorem part 3: $F\cap (F)f \neq \emptyset$.\\
Let $x \in F\cap (F)f$.\\
there exists $f_1,f_2 \in F$ such that $f_1=x=f_2f$.\\
$f_1 = f_2 f \implies f = f_2^{-1} f_1$\\
It is clear that $F$ is closed under taking inverses as the disjoint cycle structure is the same but with cycles reversed.\\
It therefore follows that $f_2^{-1}$ is conjugate to $f_2$ (by inversion), which is conjugate to $f_1$ (as they are both in $F$), which is conjugate to $f_1^{-1}$.\\
So $f_2^{-1}$ is conjugate to $f_1^{-1}$.\\
Let $g$ be such that $gf_1^{-1}g^{-1} = f_2^{-1}$ and let $h$ be $f_1^{-1}$.
\[[g,h]=ghg^{-1}h^{-1}=gf_1^{-1} g^{-1}(f_1^{-1})^{-1}=f_2^{-1}(f_1^{-1})^{-1}= f_2^{-1}f_1=f \qed\]
We now have that every element of $Sym(\mathbb{N})$ is a commutator. It remains to show for any infinite set $\Omega$ that any element of $Sym(\Omega)$ can be written as a commutator.
\begin{theorem}\label{commutator result}
Let $\Omega$ be an infinite set. Then all functions $f\in Sym(\Omega)$ can be written in the form:
$$f=[g,h]=ghg^{-1}h^{-1}$$
where $g,h\in Sym(\Omega)$.\\
In addition the conjugacy class 
\[F_\Omega:= \{f \in Sym(\Omega): f\textsl{ has }\vert \Omega \vert \textsl{ cycles of all finite lengths and none of infinite length}\}\] 
satisfies $F_{\Omega}F_\Omega =Sym(\Omega)$.
\end{theorem}\par
Proof: If $\vert \Omega \vert = \aleph_0$ then we are done by theorem \ref{temp final}. Otherwise let $f \in Sym(\Omega)$. For any point $p\in \Omega$ we have that $\vert orb_{\{f\}}(p) \vert$ is countable. Let $P_1$ be the partition of $\Omega$ into the orbits of points under $f$. We must have that $\vert P_1 \vert = \vert \Omega \vert$ as if $\vert P_1 \vert > \vert \Omega \vert$ we would have that $\vert \cup P_1 \vert > \vert \Omega \vert = \vert \cup P_1 \vert$ and if $\vert P_1 \vert < \vert \Omega \vert$ then $\vert \cup P_1 \vert \leq max\{ \vert P_1 \vert, \aleph_0\} < \vert \Omega \vert = \vert \cup P_1 \vert$. We can therefore index $P_1$ by $P_1 = \{S_i:i<\vert \Omega \vert\}$. Define an equivalence relation by: $S_i \sim S_j \iff i=j+k$ or $j=i+k$ for some $k \in \aleph_0$. Let $P_2$ be the partition of $P_1$ into equivalence classes by this relation. For the same reasons as $P_1$ we have that $\vert P_2 \vert = \vert \Omega \vert$. Let $P:= \{\cup x:x\in P_2\}$. We therefore have that $P:= \{P_i:i<\vert \Omega \vert\}$ is a partition of $\Omega$ into countably infinite sets such that for all $p\in \Omega$ we have that: $p \in P_i \iff (p)f \in P_i$. We can therefore consider $f$ as an element of $\prod_{i<\vert \Omega \vert} Sym_{\Omega}(P_i)$. As each $P_i$ is countably infinite it follows from theorem \ref{temp final} that $f\vert_{P_i}$ can be written as $[g_i,h_i]$ for some $g_i,h_i \in Sym(P_i)$ for all $i < \vert \Omega \vert$. From these we can define $g,h$ by:
$$(p)g:= (p)g_i \textsl{ were i is the index of the unique }P_i \textsl{ containing p}$$
$$(p)h:= (p)h_i \textsl{ were i is the index of the unique }P_i \textsl{ containing p}$$
Then we have that $f = [g,h]$ as required.\\
We have that $f = [g,h] = ghg^{-1}h^{-1}$ and therefore $f$ is a product of two elements in the conjugacy class of $h$. As \(|\Omega|=|\Omega|\aleph_0\) we have by the definition of $h$ that for any $f$ this conjugacy class is given by: $$F_{\Omega} := \{f \in Sym(\Omega): f\textsl{ has }\vert \Omega \vert \textsl{ cycles of all finite lengths and none of infinite length}\} \qed$$
\chapter{Generating the infinite symmetric groups}
\section{The Bergman property}
\begin{defn}
A semigroup $S$ is said to have the \textit{Bergman Property} if it satisfies the following:\\
For all $U$ such that $\langle U \rangle_s = S$ there exists a natural number $n$ such that $\bigcup_{i=1}^{n} U^i =S$.
\end{defn}
In this section we will show that given an infinite set $\Omega$ the group $Sym(\Omega)$ has the Bergman Property. This is an unusual property.
\begin{theorem}
For $G = (\mathbb{Z},+)$, $G = (\mathbb{Q},+)$, $G = (\mathbb{R},+)$, $G = (\mathbb{Z} \backslash \{0\},\times)$, $G = (\mathbb{Q}\backslash \{0\},\times)$ and $G = (\mathbb{R}\backslash \{0\},\times)$ there exists $U \subseteq G$ such that $\langle U \rangle_s = G$ and $\bigcup_{i=1}^{n} U^i \neq G$ for all $n \in \mathbb{N}$. Therefore these semigroups don't have the Bergman Property.
\end{theorem}\par
Proof: Let U be $[-2,2]\cap G$. $\qed$
\begin{theorem}\label{stabgens}
Let $\Omega$ be an infinite set and let $\{A,B,C\}$ be a partition of $\Omega$ into moieties of $\Omega$. Then we have: 
$$Sym(\Omega)=Pstab(A)Pstab(B)Pstab(A)\cup Pstab(B)Pstab(A)Pstab(B)$$
\end{theorem}\par
Proof: the following proof comes from the proof of Lemma 2.1 found in \cite{finitegen}\\
 Let $f \in Sym(\Omega)$.\\
Suppose that $\vert \Omega \vert=\vert C \backslash  Af^{-1} \vert$.\\
Let $S\subseteq C \backslash Af^{-1}$ be a moiety of $C \backslash Af^{-1}$.\\
let $b_1:B \rightarrow S$ be a bijection.\\
Let $D:=\cup \{C \backslash  S, A\cap B f^{-1},A\cap C f^{-1},\{x \in S:(x)b_1^{-1} \notin Af^{-1}\}\}$\\
Let $b_2: D \rightarrow C$ be a bijection\\
Note that $b_1$ and $b_2$ exist as:\\
$\vert B \vert =  \vert S \vert = \vert \Omega \vert$ and $\vert \Omega \vert= \vert (C \backslash  Af^{-1})\backslash  S \vert \leq \vert C\backslash  S \vert \leq \vert D \vert \leq \vert \Omega \vert \implies \vert D \vert = \vert C \vert = \vert \Omega \vert$\\
Let $\mu_1,\mu_2$ and $\mu_3$ be the bijections defined by the following (note that we know they are bijections as we have constructed their inverses):
\begin{multicols}{2}
\begin{align*}
(x)\mu_1&=
 \left\{
    \begin{array}{lr}
      (x)b_1& x \in B \\
      (x)b_1^{-1}& x \in S \\
            x &  x \in C \backslash S\\      x & x \in A\\
    \end{array}
    \right\}\\
    (x)\mu_2&=
 \left\{
    \begin{array}{lr}
      (x)f&  x \in A\cap A f^{-1} \\
            (x)b_2&  x \in A\cap B f^{-1} \\
                  (x)b_2&  x \in A\cap C f^{-1} \\
      (x)f&  x \in C \cap Af^{-1} \\
      (x)b_1^{-1}f&  x \in S \textsl{ and } (x)b_1^{-1} \in Af^{-1} \\
      (x)b_2&  x \in S \textsl{ and } (x)b_1^{-1} \notin Af^{-1} \\
            (x)b_2&  x \in C \backslash  S  \\
      x&  x \in B
    \end{array}
    \right\}\\
(x)\mu_3&=
 \left\{
    \begin{array}{lr}
      (x)b_1f&  x \in B \\      
            (x)b_2^{-1}b_1^{-1}f& x \in C \cap Sb_2\\
                        (x)b_2^{-1}f& x \in C \backslash  Sb_2\\
      x& x \in A
    \end{array}
    \right\}
\end{align*}
\begin{align*}
(x)\mu_1^{-1}&=
 \left\{
    \begin{array}{lr}
      (x)b_1^{-1}& x \in S \\
      (x)b_1& x \in B \\
            x &  x \in C \backslash  S\\      x & x \in A\\
    \end{array}
    \right\}\\
    (x)\mu_2^{-1}&=
 \left\{
    \begin{array}{lr}
      (x)f^{-1}&  x \in A\cap A f \\
            (x)b_2^{-1}&  x \in C\\
      (x)f^{-1}&  x \in A \cap Cf \\
      (x)f^{-1}b_1&  x \in A \cap Bf \\
      x&  x \in B
    \end{array}
    \right\}\\
    (x)\mu_3^{-1}&=
 \left\{
    \begin{array}{lr}
      (x)f^{-1}b_1^{-1}&  x \in Sf \\
                        (x)f^{-1}b_2& x \in Df \textsl{ and } x \notin Sf\\
      x&  x \in A\\
                  (x)f^{-1}b_1 b_2&  \textsl{otherwise}\\
    \end{array}
    \right\}
\end{align*}
\end{multicols}
By definition we have that $\mu_1,\mu_3 \in Pstab(A)$ and $\mu_2 \in Pstab(B)$\\
Finally we can see that $\mu_1\mu_2\mu_3 = f$\\
So we have that $f \in Pstab(A)Pstab(B)Pstab(A)$\\
Suppose that $\vert \Omega \vert=\vert C \backslash  Bf^{-1} \vert$ then  similarly we conclude that $f \in Pstab(B)Pstab(A)Pstab(B)$\\
As $C=(C \backslash  Af^{-1})\cup (C \backslash  Bf^{-1})$ and $\vert C \vert=\vert \Omega \vert$ we must be in one of these cases and we are done. $\qed$
\begin{defn}
A semigroup $S$ is said to have \textit{Property 1} if it satisfies the following:\\
Every function $\psi:S \rightarrow \mathbb{N}$ such that there is some constant  $C_\psi$ satisfying: $$(st)\psi \leq (s)\psi + (t)\psi + C_\psi \ \ \ \textsl{for all } s,t \in S$$ Is bounded above.
\end{defn}
\begin{theorem} \label{conditions}
Let $S$ be a semigroup, which satisfies Property 1. Then $S$ also satisfies the Bergman Property.
\end{theorem}\par
Proof: Suppose for a contradiction that $S$ doesn't have the Bergman property. Then there is a $U$ such that $\langle U \rangle_s = S$ and that $\bigcup_{i=1}^{n} U^i \neq S$ for all $n \in \mathbb{N}$.\\
Define $\psi : S \rightarrow \mathbb{N}$ by:
$$(s)\psi = min\{n \in \mathbb{N}:s \in \bigcup_{i=1}^{n} U^i\}$$
Let $C_\psi = 0$.\\
Let $s,t \in S$\\
Observe that if $(s)\psi=l_1$ and $(t)\psi=l_2$. Then we have that $s=u_{s_1}u_{s_2} \ldots u_{s_{l_1}}$ and $t=u_{t_1}u_{t_2} \ldots u_{t_{l_2}}$\\ for some $u_{s_1},u_{s_2} \ldots u_{s_{l_1}},u_{t_1},u_{t_2}\ldots u_{t_{l_2}} \in U$.\\
Therefore $st=u_{s_1}u_{s_2} \ldots u_{s_{l_1}}u_{t_1}u_{t_2} \ldots u_{t_{l_2}}$\\
So we have that $st \in \bigcup_{i=1}^{l_1+l_2} U^i \implies (st)\psi \leq l_1 + l_2 = (s)\psi + (t)\psi + C_\psi$\\
By Property 1 we therefore have that $\psi$ is bounded by some $N \in \mathbb{N}$ and therefore $(s)\psi \leq N$ for all $s \in S$\\
This implies that $s\in \bigcup_{i=1}^{N} U^i$ for all $s \in S$ so we have $S = \bigcup_{i=1}^{N} U^i$, a contradiction as required. $\qed$
\begin{theorem} \label{finite gens} 
Let $\Omega'$ be an infinite set. Let $S\subseteq Sym(\Omega')$ be countable. There exists a subgroup $G$ of $Sym(\Omega ')$ of rank at most 4 such that $S \subseteq G$.
\end{theorem}\par
Proof: The following proof comes from the proof of theorem 3.1 found in \cite{finitegen}\\
Let $\Omega = \mathbb{Z}\times\mathbb{Z}\times\Omega'$.\\
Clearly $Sym(\Omega)\cong Sym(\Omega')$ so it suffices to prove the result for $Sym(\Omega)$.\\
Let $M$ be a moiety of $\Omega'$.
\[\Omega_0 :=\{0\}\times \{0\} \times \Omega'\ \ \ \ I := \Omega \backslash  \Omega_0\ \ \ \ \Omega_0^+  :=\{(0,0,x): x \in M\}\ \ \ \ \Omega_0^-  :=\{(0,0,x): x \in M^c\}\]
The sets $I, \Omega_0^+$ and $\Omega_0^-$ are moieties of $\Omega$ which partition $\Omega$ so by theorem \ref{stabgens} all elements of $S$ can be written as a product of 3 elements of $Pstab(I)\cup Pstab(\Omega_0^+)$. Therefore $S$ is generated by a countable subset of $Pstab(I)\cup Pstab(\Omega_0^+)$ so we may assume that $S \subseteq Pstab(I)\cup Pstab(\Omega_0^+)$.\\
Let $\phi_1:\Omega \rightarrow \Omega$ be an involution satisfying $(\Omega_0^+)\phi_1=I$.\\
Let $\phi_2:\Omega \rightarrow \Omega$, $\phi_3:\Omega \rightarrow \Omega$ and $S'$ be defined by:
\begin{align*}
((a,b,c))\phi_2&=(a+1,b,c)\\
((a,b,c))\phi_3&=
 \left\{
    \begin{array}{lr}
      (a,b+1,c)&  a=0 \\
     (a,b,c)&  a \neq 0\\
    \end{array}
    \right\}\\
S' &:= (\phi_1(S \cap Pstab(\Omega_0^+))\phi_1)\cup (S \cap Pstab(I))
\end{align*}
Note that all elements of $S'$ stabilize $I$ pointwise and thus $S' \subseteq Pstab(I)$. In addition $S'$ is countable as the union of 2 countable sets.\\
Let $S' = \{f_i: i \in \mathbb{Z}\}$ be an enumeration of $S'$.\\
As $S'$ stabilizes $I$ pointwise for each $f_i$ we can construct $\hat{f_i}:\Omega' \rightarrow \Omega'$ such that $(0,0,p)f_i=(0,0,p\hat{f_i})$ for all $p \in \Omega'$.\\
Let $\phi_4: \Omega \rightarrow \Omega$ be defined by:
\[((a,b,c))\phi_4= \left\{
    \begin{array}{lr}
      (a,b,c\hat{f_a})&  b\geq 0 \\
     (a,b,c)&  b<0\\
    \end{array}
    \right\}\]
Let $z,a,b \in \mathbb{Z}$ and $c \in \Omega'$:
\begin{align*}
((a,b,c))\phi_2^z\phi_4\phi_2^{-z}\phi_3^{-1}\phi_2^z\phi_4^{-1}\phi_2^{-z}\phi_3
&=((a+z,b,c))\phi_4\phi_2^{-z}\phi_3^{-1}\phi_2^z\phi_4^{-1}\phi_2^{-z}\phi_3\\
&=\left\{
    \begin{array}{lr}
      (a+z,b,c\hat{f_{a+z}})\phi_2^{-z}\phi_3^{-1}\phi_2^z\phi_4^{-1}\phi_2^{-z}\phi_3&  b\geq 0 \\
     (a+z,b,c)\phi_2^{-z}\phi_3^{-1}\phi_2^z\phi_4^{-1}\phi_2^{-z}\phi_3&  b<0\\
    \end{array}
    \right\}\\
&=\left\{
    \begin{array}{lr}
      (a,b,c\hat{f_{a+z}})\phi_3^{-1}\phi_2^z\phi_4^{-1}\phi_2^{-z}\phi_3&  b\geq 0 \\
     (a,b,c)\phi_3^{-1}\phi_2^z\phi_4^{-1}\phi_2^{-z}\phi_3&  b<0\\
    \end{array}
    \right\}\\
&=\left\{
    \begin{array}{lr}
      (a,b-1,c\hat{f_{a+z}})\phi_2^z\phi_4^{-1}\phi_2^{-z}\phi_3&  b\geq 0 \textsl{ and } a=0\\
     (a,b-1,c)\phi_2^z\phi_4^{-1}\phi_2^{-z}\phi_3&  b<0  \textsl{ and } a=0\\
           (a,b,c\hat{f_{a+z}})\phi_2^z\phi_4^{-1}\phi_2^{-z}\phi_3&  b\geq 0 \textsl{ and } a\neq 0\\
     (a,b,c)\phi_2^z\phi_4^{-1}\phi_2^{-z}\phi_3&  b<0  \textsl{ and } a\neq 0\\
    \end{array}
    \right\}\\
&=\left\{
    \begin{array}{lr}
      (a+z,b-1,c\hat{f_{a+z}})\phi_4^{-1}\phi_2^{-z}\phi_3&  b\geq 0 \textsl{ and } a=0\\
     (a+z,b-1,c)\phi_4^{-1}\phi_2^{-z}\phi_3&  b<0  \textsl{ and } a=0\\
           (a+z,b,c\hat{f_{a+z}})\phi_4^{-1}\phi_2^{-z}\phi_3&  b\geq 0 \textsl{ and } a\neq 0\\
     (a+z,b,c)\phi_4^{-1}\phi_2^{-z}\phi_3&  b<0  \textsl{ and } a\neq 0\\
    \end{array}
    \right\}\\
&=\left\{
    \begin{array}{lr}
      (a+z,b-1,c\hat{f_{a+z}})\phi_2^{-z}\phi_3&  b = a=0\\
     (a+z,b-1,c)\phi_2^{-z}\phi_3&  b \neq 0  \textsl{ and } a=0\\
     (a+z,b,c)\phi_2^{-z}\phi_3&   a\neq 0\\
    \end{array}
    \right\}\\
&=\left\{
    \begin{array}{lr}
      (a,b-1,c\hat{f_{a+z}})\phi_3&  b = a=0\\
     (a,b-1,c)\phi_3&  b \neq 0  \textsl{ and } a=0\\
     (a,b,c)\phi_3&   a\neq 0\\
    \end{array}
    \right\}\\
&=\left\{
    \begin{array}{lr}
      (a,b,c\hat{f_{a+z}})&  b = a=0\\
     (a,b,c)& b \neq 0  \textsl{ and } a=0\\
     (a,b,c)&   a\neq 0\\
    \end{array}
    \right\}\\
&=\left\{
    \begin{array}{lr}
      (0,0,c\hat{f_{z}})&  b = a=0\\
     (a,b,c)& b \neq 0  \textsl{ or } a\neq0\\
    \end{array}
    \right\}\\
&=((a,b,c))f_z
\end{align*}
Therefore $f_z=\phi_2^z\phi_4\phi_2^{-z}\phi_3^{-1}\phi_2^z\phi_4^{-1}\phi_2^{-z}\phi_3 \in \langle \phi_2,\phi_3,\phi_4 \rangle$.\\
So we have that $S' \subseteq \langle \phi_2,\phi_3,\phi_4 \rangle$.\\
Let $f \in S$.\\
If $f \in S \cap Pstab(I)$ then $f\in S' \subseteq \langle \phi_2,\phi_3,\phi_4 \rangle \subseteq \langle \phi_1, \phi_2,\phi_3,\phi_4 \rangle $.\\
Otherwise $f \in S \cap Pstab(\Omega_0^+)$ so we have that $\phi_1 f \phi_1 \in S'\implies f\in \phi_1 S' \phi_1 \subseteq \phi_1 \langle \phi_2,\phi_3,\phi_4 \rangle \phi_1 \subseteq \langle \phi_1, \phi_2,\phi_3,\phi_4 \rangle $.\\
So we have that $f \in \langle \phi_1, \phi_2,\phi_3,\phi_4 \rangle$ for all $f \in S$ so we have that $S \subseteq \langle \phi_1, \phi_2,\phi_3,\phi_4 \rangle = G$ as required $\qed$
\begin{theorem}\label{an}
Let $\Omega$ be an infinite set. There is a sequence $(a_n)_{n \in \mathbb{N}}$ and $L\in \mathbb{N}$ such that for all $(s_n)_{n \in \mathbb{N}} \subset Sym(\Omega)$ there are $
 g_1, g_2 \ldots g_{L} \in Sym(\Omega)$ such that for all $n \in \mathbb{N}$ $s_n = g_{n_1},g_{n_2} \ldots g_{n_k}$ for some $k \leq a_n$.
\end{theorem}\par
Proof: Choose $(a_n)_{n \in \mathbb{N}}= (18 + 252n)_{n \in \mathbb{N}}, \ \ L = 8$\\
Let $(s_n)_{n \in \mathbb{N}}$ be a sequence in $Sym(\Omega)$.\\
By Theorem \ref{stabgens} build a sequence $(s'_n)_{n \in \mathbb{N}} \subseteq  Pstab(I) \cup Pstab(\Omega_0^+)$ such that $s_n = s'_{3n - 2}s'_{3n - 1}s'_{3n}$ for all $n \in \mathbb{N}$.\\
Let $\phi_1$ be defined as in Theorem \ref{finite gens}. Let $S'$ be defined as in Theorem \ref{finite gens} but choose an enumeration of $S'$ such that for all $n \in \mathbb{N}$ $s'_n= f_{z_n}$ or $s'_n= \phi_1 f_{z_n} \phi_1$ for some $z_n \in \mathbb{Z}$ satisfying $\vert z_n \vert \leq 7n+7$ (Note that clearly this bound can be improved, but doing so is unnecessary for this proof and I like the number 7). Let $G = (g_1,g_2,g_3,g_4,g_5,g_6,g_7,g_8)=(\phi_1,\phi_1^{-1},\phi_2,\phi_2^{-1},\phi_3,\phi_3^{-1},\phi_4,\phi_4^{-1})$ as defined in Theorem \ref{finite gens}. \\
Let $(s)L$ denote the minimum length of $s$ as a product of elements of $G$. It suffices to show that $(s_n)L \leq a_n$ for all $n \in \mathbb{N}$\\ 
Let $n \in \mathbb{N}$:
$(s'_n)L \leq (f_{z_n})L + 2 \leq 4\vert z_n \vert+6$ by proof of Theorem \ref{finite gens}.\\
Therefore $(s_n)L \leq (s'_{3n-2})L + (s'_{3n-1})L + (s'_{3n})L \leq 4\vert z_{3n} \vert+6 + 4\vert z_{3n-2} \vert+6 + 4\vert z_{3n} \vert+6=18 + 4(\vert z_{3n-2} \vert + \vert z_{3n-1} \vert + \vert z_{3n} \vert)$\\
Therefore $(s_n)L \leq 18 + 4(\vert z_{3n-2} \vert + \vert z_{3n-1} \vert + \vert z_{3n} \vert) \leq 18 + 4(7(3n-2)+7 + 7(3n-1)+7 + 7(3n)+7)=18 + 28(3n-2+1 + 3n-1+1 + 3n+1)=18 + 28(9n)=18 + 252n=a_n$.\\
So we have that for all $n \in \mathbb{N}$ $(s_n)L \leq a_n$ as required. $\qed$
\begin{theorem}\label{has conditions}
Let $\Omega$ be an infinite set. $Sym(\Omega)$ satisfies property 1 and the Bergman Property.
\end{theorem}\par
Proof: The following proof is taken from the proof of Lemma 2.4 in \cite{bergman property}.\\
By Theorem \ref{conditions} it suffices to show that $Sym(\Omega)$ satisfies property 1.\\ 
Let $(a_n)_{n\in \mathbb{N}}$ be as in Theorem \ref{an}.\\
We may assume $a_n$ is strictly increasing as $a_n$ can be replaced by $max\{a_m + 1:m \leq n\}$ and the required property clearly holds.\\
Suppose for a contradiction that there exists $\psi: Sym(\Omega) \rightarrow \mathbb{N}$ such that there exists $C_\psi$ such that $$(st)\psi \leq (s)\psi + (t)\psi + C_\psi\ \ \textsl{ for all } s,t \in Sym(\Omega)$$ and $\psi$ is unbounded.\\
As $\psi$ is unbounded for all $n\in \mathbb{N}$ there exists $s \in Sym(\Omega)$ such that $(s)\psi > n$.\\
Therefore we can construct a sequence $(s_n)_{n \in \mathbb{N}}$ such that $(s_n)\psi > a_n^2\ \textsl{ for all } n \in \mathbb{N}$.\\
We now construct a set of generators $G=(g_1,g_2,g_3,g_4,g_5,g_6,g_7,g_8)$ for $(s_n)_{n \in \mathbb{N}}$ as done in Theorem \ref{an}.\\
Let $M = max\{(g)\psi: g \in G \}$\\
Each $s_n$ can be written as a product of length at most $a_n$ in elements of $G$ it therefore follows from induction that: $$(s_n)\psi \leq a_nC_\psi + a_nM \ \textsl{ for all } n \in \mathbb{N}$$
For all sufficiently large n we have that $a_n>C_\psi + M$ and therefore: $$(s_n)\psi \leq a_nC_\psi + a_nM=a_n(C_\psi + M)< a_n^{2} < (s_n)\psi$$
So $(s_n)\psi < (s_n)\psi$. This is a contradiction. $\qed$
\section{Cofinality and strong cofinality}
In this section we will be exploring the cofinality and strong cofinality of infinite symmetric groups.
\begin{defn}
Let \(S\) be a semigroup. A \textit{cofinal chain} of $S$, is defined to be a chain of strict subsemigroups $(S_i)_{i<\kappa}$ of $S$ indexed by the ordinals less than some cardinal $\kappa$ such that: 
\[ S =\bigcup_{i<\kappa}S_i\]
and $S_i \subseteq S_j$ for $i \leq j$. 
\end{defn}
\begin{defn}
Let $S$ be a semigroup. A \textit{strong cofinal chain} of $S$, is defined to be a chain of strict subsets $(S_i)_{i<\kappa}$ of $S$ indexed by the ordinals less than some cardinal $\kappa$ such that:  
\[S =\bigcup_{i<\kappa}S_i\]
and $S_i \subseteq S_j$ for $i \leq j$
and for all $i<\kappa$ there exists $j<\kappa$ such that:
$$S_iS_i \subseteq S_j$$
\end{defn}
\begin{theorem}
Let $S$ be a non-finitely generated semigroup, then there exists cofinal and strong cofinal chains of $S$.
\end{theorem}\par
Proof: This proof comes from Note 3 of \cite{ultrafiltermax}.\\
Let $S= \{t_i:i<\vert S \vert\}$ be an enumeration of $S$. Let $S_i:=\langle \{t_j:j<i\}\rangle_s$ for $i<\vert S \vert$. We will show that $(S_i)_{i<\vert S \vert}$ is a cofinal chain.\\
It is clear that $ S =\bigcup_{i<\vert S \vert}S_i$ and $S_i \subseteq S_j$ for $i \leq j$. To see that the $S_i$ are strict subsemigroups observe that each $S_i$ is generated by a set indexed by an ordinal $i<\vert S \vert$ and therefore is generated by  a set of strictly smaller cardinality. If $S$ is countable it follows that $S \neq S_i$ as $S_i$ is finitely generated, and if $S$ is uncountable it follows that $S \neq S_i$ as $\vert S_i \vert < \vert S \vert$. We therefore have that $(S_i)_{i<\vert S \vert}$ is a cofinal chain. It therefore follows that $(S_i)_{i<\vert S \vert}$ is also a strong cofinal chain as semigroups are closed. $\qed$\\\par
Note that the validity of the following two definitions follows from the previous theorem together with the fact that the cardinals are a subclass of the ordinals and thus any set of cardinals has a least element.
\begin{defn}
Let $S$ be a non-finitely generated semigroup. The \textit{cofinality} of $S$, denoted $cf(S)$, is defined to be the smallest cardinal $\kappa$ such that there exists a cofinal chain of $S$ indexed by $\kappa$.
\end{defn}
\begin{defn}
Let $S$ be a non-finitely generated semigroup. The \textit{strong cofinality} of $S$, denoted $scf(S)$, is defined to be the smallest cardinal $\kappa$ such that there exists a strong cofinal chain of $S$ indexed by $\kappa$.
\end{defn}
Notice that $Sym(\Omega)$ is not finitely generated for an infinite $\Omega$ as it is uncountable by theorem \ref{big} and therefore  we can assign it a cofinality and a strong cofinality.
\begin{theorem}\label{big cf}
Let $\Omega$ be an infinite set, then $cf(Sym(\Omega))> \aleph_0$
\end{theorem}\par
\begin{proof}
Suppose for a contradiction that $cf(Sym(\Omega)) \leq \aleph_0$. Then there is a set of strict subsemigroups of $Sym(\Omega)$ $(S_i)_i$ indexed by natural numbers, who's union is equal to $Sym(\Omega)$.\\
Let $\psi:Sym(\Omega) \rightarrow \mathbb{N}$ be defined by:
$(f)\psi = min\{n \in \mathbb{N}: f \in S_n \}$.\\
Let $s,t \in Sym(\Omega)$:
Wlog we may assume that $S_{\psi(s)}\subseteq S_{\psi(t)}$\\
It follows that $st \in S_{\psi(t)}$.\\
Let $C_\psi = 0$\\
So we have $(st)\psi \leq  (t)\psi \leq (t)\psi + (s)\psi + C_\psi$.\\
By theorem \ref{has conditions} it follows that $Sym(\Omega)$ has property 1 and so this function is bounded by some natural number $N$.\\
It follows then that for all $f \in Sym(\Omega)$, $f\in S_N$ so $S_N \nless Sym(\Omega)$ this is a contradiction.
\end{proof} 
\begin{theorem}
Let $\Omega$ be an infinite set, then $scf(Sym(\Omega))> \aleph_0$
\end{theorem}\par
Proof: The following proof comes from the proof of proposition 2.2 in \cite{bergman property}.\\
Suppose for a contradiction that $scf(Sym(\Omega)) \leq \aleph_0$. Then there is a chain of strict subsets of $Sym(\Omega)$ $(S_i)_i$ indexed by natural numbers such that $$S =\bigcup_{i\in \mathbb{N}}S_i$$
and $S_i \subseteq S_j$ for $i \leq j$
and for all $i\in \mathbb{N}$ there exists $j \in \mathbb{N}$ such that $S_iS_i \subseteq S_j$.\\
It is clear that $$S =\bigcup_{i\in \mathbb{N}} \langle S_i \rangle_s$$
and $\langle S_i \rangle_s \subseteq \langle  S_j \rangle_s$ for $i \leq j$. From theorem \ref{big cf} we have that $cf(Sym(\Omega)) > \aleph_0$ and so we must have that $\langle S_j \rangle_s = Sym(\Omega)$ for some $j \in \mathbb{N}$.\\
By theorem \ref{has conditions} $Sym(\Omega)$ has the Bergman property so it follows that $$Sym(\Omega) = \bigcup _{k = 1}^{n} U_j^{k}$$ for some $n \in \mathbb{N}$.\\
To reach a contradiction it suffices to show that $Sym(\Omega) = \bigcup _{k = 1}^{n} U_j^{k} \subseteq U_N$ for some $N\in \mathbb{N}$.\\
We have that $U_jU_j \subseteq U_{j_1}$ for some $j_1 \in \mathbb{N}$, and $U_jU_jU_jU_j\subseteq U_{j_1}U_{j_1} \subseteq U_{j_2}$ for some $j_2 \in N$.\\
By repeating this we have that $U_j^{2^a}\subseteq U_{j_a}$ for some $j_a \in \mathbb{N}$.\\
As $j_1 \geq j$ it follows that $U_j \subseteq U_{j_1}$.\\
So we have that:
\begin{align*}
&U_{j_1}^{2^{a-1}} \subseteq U_{j_a} \ \ \ \textsl{for all } a \in \mathbb{N}\\
&\implies (U_{j} \cup U_j U_j)^{2^{a-1}}
 \subseteq U_{j_a} \ \ \ \textsl{for all } a \in \mathbb{N}\\
&\implies U_{j}^{k}
 \subseteq U_{j_a} \ \ \ \textsl{for all } k \leq 2^a \ \ \textsl{for all } a \in \mathbb{N}
\end{align*}
So we have that for all $k \in \mathbb{N}$, there exists $N\in \mathbb{N}$ such that $\bigcup_{k=1}^n U_j^k \subseteq U_N$ as required. $\qed$
\section{Shuffling the plane}
In this section we try to write the elements of \(Sym(\Omega)\) as the product of 'slides'.
Throughout the next two sections we will consider \(\Omega\) as \(A\times A\) where \(A\) is an abelian group of order \(\Omega\) (note that there are abelian groups of all infinite cardinalities for example a free group quotiented by it's derived subgroup). Let \(A\) be indexed by \(\{a_i:i<|A|\}\) with \(a_0=id_A\). We will use the functions \(\pi_1,\pi_2\) to be the projection of a tuple onto it's first and second coordinates respectively and if we have functions \(S_i,S_{i+1}\ldots S_k\) we will use the notation \(S_{i\rightarrow k}\) to mean  \(S_iS_{i+1}\ldots S_k\). 
\begin{defn}
Let \(f:A\rightarrow A\) be a function. Then
a vertical slide(by \(f\)) \(v_f:A\times A \rightarrow A\times A\) is defined by:
\[(x,y)v_f=(x,y+(x)f)\]
Similarly a horizontal slide(by \(f\)) \(h_f:A\times A \rightarrow A\times A\) is defined by:
\[(x,y)h_f=(x+ (y)f,y)\]
We will use the word slide to refer to either of these.
\end{defn}
Note that the set of vertical(or horizontal) slides forms an abelian group. We will start by showing any moiety can be mapped into the diagonal line \(\{(x,y):x=y\}\) and then show from this that we can construct any element of \(Sym(\Omega)\).
\begin{defn}
Let \(x \in A\). Then the vertical and horizonal lines of \(x\) are defined respectively by:
\[v_x=\{(x,y):y \in A\}\]
\[h_x=\{(y,x):y \in A\}\]
The world line will be used to describe either of these.
\end{defn}
\begin{defn}
Let \(L\subseteq \Omega\) be a line and let \(S\subseteq \Omega\). We say that \(L\) is S-contained if we have that \(L\subseteq S\), we say \(L\) is S-disjoint if \(L \cap S = \emptyset\) and we say that \(L\) is S-spotty if we have that neither \(L\subseteq S\) nor \(L \subseteq S^c\).
\end{defn}
\begin{theorem}\label{many spots}
Let \(M\) be a moiety of \(\Omega\) then either \(|\{x\in A:h_x \text{ is M-spotty }\}|=|A| \) or \(|\{x\in A:v_x \text{ is M-spotty }\}|=\vert A \vert\)
\end{theorem}
\begin{proof}
The following proof comes from lemma 1 in \cite{shuffle2}.\\
Suppose not, then we have \(|\{x\in A:v_x \text{ is M-spotty }\}|<|A| \) and \(|\{x\in A:h_x \text{ is M-spotty }\}|<|A| \) and thus also \(|\{x\in A: v_x \text{ is M-contained or M-disjoint}\}|=|A|\).\\
Case 1:\\
If \(|\{x\in A:v_x  \text{ is M-contained }\}|=|A|\) and \(|\{x\in A: v_x \text{ is M-disjoint }\}|=|A|\) then it follows that for all \(y \in A\) we have \(h_y\) is M-spotty a contradiction.\\
Case 2:\\
If \(|\{x\in A:v_x  \text{ is M-contained }\}|<|A|\) it follows that we have \(|\{x\in A:v_x  \text{ is M-contained or M-spotty }\}|<|A|\) and \(|\{x\in A:v_x  \text{ is M-disjoint }\}|=|A|\). It follows that \(\{x\in A: h_x \text{ is M-contained }\} = \emptyset\). As \(|\{x\in A:h_x \text{ is M-spotty }\}|<|A| \) we have that \(|\{x\in A:h_x \text{ is M-spotty or M-contained }\}|<|A| \). However:
 \[M \subseteq \{x\in A:v_x \text{ is M-spotty or M-contained }\}\times\{x\in A:h_x \text{ is M-spotty or M-contained }\}\]
Thus it follows that \(|M|<|A\times A|=|A|\) as \(M\) is a moiety this is a contradiction.\\
Case 3:\\
If \(|\{x\in A:v_x  \text{ is M-disjoint }\}|<|A|\) it follows that we have \(|\{x\in A:v_x  \text{ is M-disjoint or M-spotty }\}|<|A|\) and \(|\{x\in A:v_x  \text{ is M-contained }\}|=|A|\). It follows that \(\{x\in A: h_x \text{ is M-disjoint }\} = \emptyset\). As \(|\{x\in A:h_x \text{ is M-spotty }\}|<|A| \) we have that \(|\{x\in A:h_x \text{ is M-spotty or M-disjoint }\}|<|A| \). However:
 \[M^c \subseteq \{x\in A:v_x \text{ is M-spotty or M-disjoint }\}\times\{x\in A:h_x \text{ is M-spotty or M-disjoint }\}\]
Thus it follows that \(|M^c|<|A\times A|=|A|\) as \(M\) is a moiety this is a contradiction.\\
\end{proof}
\begin{theorem}\label{Soon to be all spots}
Let \(M\) be a moiety, then there exists a slide \(S\) such that either \(h_x\) is MS-spotty for all \(x\in A\) or \(v_x\) is MS-spotty for all \(x\in A\).
\end{theorem}
\begin{proof}
The following proof comes from lemma 2 in \cite{shuffle2}.\\
By theorem \ref{many spots} we may assume without loss of generality that \(|\{x\in A:h_x \text{ is M-spotty }\}|=|A| \).\\
Let \(\{M_1,M_2\}\) be a partition of \(\{x\in A:h_x \text{ is M-spotty }\}\) into moieties . Let \(\phi_1:M_1\rightarrow A\) and \(\phi_2:M_2\rightarrow A\) be bijections. Let \(\phi_3:\{x\in A:h_x \text{ is M-spotty }\} \rightarrow A\) and \(\phi_4:\{x\in A:h_x \text{ is M-spotty }\} \rightarrow A\) be such that for all \(x \in A\) such that \(h_x\) is M-spotty we have \(((x)\phi_3,x) \in M\) and \(((x)\phi_4,x) \notin M\).\\
 Let \(f:A \rightarrow A\) be defined by:
\[(a)f= \left\{
    \begin{array}{lr}
       -(a)\phi_3 + (a)\phi_1  & a \in M_1\\
       -(a)\phi_4 + (a)\phi_2  & a \in M_2\\
    a & \text{otherwise}\\
    \end{array}\right\}\]
 Let \(S=h_f\). For all \(x \in A\) we have \[(x,(x)\phi_1^{-1})=((x)\phi_1^{-1}\phi_3-(x)\phi_1^{-1}\phi_3+(x)\phi_1^{-1}\phi_1,(x)\phi_1^{-1})=((x)\phi_1^{-1}\phi_3+(x)\phi_1^{-1}f,(x)\phi_1^{-1})=((x)\phi_1^{-1}\phi_3,(x)\phi_1^{-1})h_f\in (M)S\] 
 \[(x,(x)\phi_2^{-1})=((x)\phi_2^{-1}\phi_4-(x)\phi_2^{-1}\phi_4+(x)\phi_2^{-1}\phi_2,(x)\phi_2^{-1})=((x)\phi_2^{-1}\phi_4+(x)\phi_2^{-1}f,(x)\phi_2^{-1})=((x)\phi_2^{-1}\phi_4,(x)\phi_2^{-1})h_f \notin (M)S\] and thus \(v_x\) is MS-spotty as required.
\end{proof}
\begin{defn}
Let \(P:=\{(x,X,Y_0,Y_1):x\in A, X \text{ is an initial segment of }A, Y_0,Y_1\subseteq A\backslash X \text{ such that }Y_0\cap Y_1=\emptyset,\text{ and }|Y_0|,|Y_1|\leq 2\}\)
\end{defn}
\begin{theorem}\label{disjoint initials}
The set \(P\) can be indexed as \(\{p_i=(x_i,X_i,Y_{0,i},Y_{1,i}):i<|P|\}\) such that for all initial segments \(\{a_j:j< a_m\}\) of \(A\) we have \(\{p_i:\{x_i\}\cup X_i \cup Y_{0,i}\cup Y_{1,i}\subseteq \{a_j:j< M\}\}\) is bounded above. In addition there exist \(t_i \in A\) such that \(\{t_i+X_i \cup Y_{0,i} \cup Y_{1,i}:i <|P|\}\) are pairwise disjoint.
\end{theorem}
\begin{proof}
This proof comes from \cite{shuffle2} lemma 3.\\
By the axiom of choice let \(c_p:\mathcal{P}(P) \rightarrow P\) and \(c_a:\mathcal{P}(A) \rightarrow A\) be choice functions.\\
By transfinite recursion let \(A_{i}=\cup_{j<i}(\{x_k\}\cup X_k \cup Y_{0,k} \cup Y_{1,k}\})\), \(p_i=(\{(x,X,Y_0,Y_1)\in P: \{x\}\cup X \cup Y_0 \cup Y_1\subseteq A_i\} \backslash \{p_j:j<i\})c_p\) unless this set is empty in which case \(p_i=(min{(A\backslash \cup_{j<i}A_j),\emptyset,\emptyset,\emptyset)}\) in addition let \(t_i:=(A\backslash \cup_{j<i}(t_j+X_j \cup Y_{0,j} \cup Y_{1,j}-X_i \cup Y_{0,i} \cup Y_{1,i}))c_a\). Note that \(A\backslash \cup_{j<i}{(t_j+X_j \cup Y_{0,j} \cup Y_{1,j}-X_i \cup Y_{0,i} \cup Y_{1,i})}\) is non-empty as we are removing less than \(|A|\) points from \(A\). Note that by construction we have \(\{p_i:\{x_i\}\cup X_i \cup Y_{0,i}\cup Y_{1,i}\subseteq \{a_j:j< M\}\}\) is bounded above by \((a_{m^+},\emptyset,\emptyset,\emptyset)\), and if \(\{t_i+X_i \cup Y_{0,i} \cup Y_{1,i}:i <|P|\}\) were not pairwise disjoint then we would have \((t_i+X_i \cup Y_{0,i} \cup Y_{1,i})\cap (t_j+X_j \cup Y_{0,j} \cup Y_{1,j})\neq \emptyset\) for some \(i>j\) and therefore \(t_i\in (t_j+X_j \cup Y_{0,j} \cup Y_{1,j}-X_i \cup Y_{0,i} \cup Y_{1,i})\) a contradiction.
\end{proof}
\begin{theorem}\label{disjoint segments slide}
Let \(M\) be a moiety of \(\Omega\). Then there exist slides \(S_1,S_2\) such that for all \(p_i \in P\) we have \(\{(x_i,t_i + b):b\in Y_{1,i}\}\subseteq MS_{1\rightarrow 2}\) and \(\{(x_i,t_i+c):c\in X_i\cup Y_{0,i}\}\subseteq (MS_{1\rightarrow 2})^c\) or we have \(\{(t_i + b,x_i):b\in Y_{1,i}\}\subseteq MS_{1\rightarrow 2}\) and \(\{(t_i+c,x_i):c\in X_i\cup Y_{0,i}\}\subseteq (MS_{1\rightarrow 2})^c\)
\end{theorem}
\begin{proof}
This proof comes from \cite{shuffle2} lemma 3.\\
Let \(S_1\) be as in theorem \ref{Soon to be all spots} and without loss of generality assume that for all \(x \in A\) we have \(h_x\) is \(MS_1\)-spotty. By theorem \ref{disjoint initials} we have \(\{t_i+X_i \cup Y_{0,i} \cup Y_{1,i}:i <|P|\}\) are pairwise disjoint. Let \(\phi_1:A\rightarrow A\) and \(\phi_2:A\rightarrow A\) be such that \(((x)\phi_1,x)\in MS_1\) and \(((x)\phi_2,x)\notin MS_1\). Let \(f:A\rightarrow A\) be defined by:
\[(a)f= \left\{ 
	\begin{array}{lr} 
			-(a)\phi_1 + x_i & a \in t_i + Y_{1,i} \\
		-(a)\phi_2 + x_i & a \in t_i+X_i\cup Y_{0,i} \\
		a & \text{ otherwise }
	\end{array}\right\}\]
Let \(S_2=h_f\). For all \(i < |P|\), \(b\in Y_{1,i}\) and \(c \in X_i \cup Y_{0,i}\) we have 
\[(x_i,t_i+b)=((t_i+b)\phi_1-(t_i+b)\phi_1 +x_i,t_i+b)=((t_i+b)\phi_1,t_i+b)h_f\in MS_{1\rightarrow 2}\] 
\[(x_i,t_i+c)=((t_i+c)\phi_2-(t_i+c)\phi_2 +x_i,t_i+c)=((t_i+c)\phi_2,t_i+c)h_f\notin MS_{1\rightarrow 2}\] 
\end{proof}
\begin{theorem}\label{moiety to diag}
Let \(M\) be a moiety of \(\Omega\), then there exist slides \(S_1,S_2,S_3,S_4,S_5\) such that \(MS_{1\rightarrow 5}\subseteq \{(a,a):a\in A\}\).
\end{theorem}
\begin{proof}
This proof comes from \cite{shuffle2} lemma 4.\\
 Let \(S_1,S_2\) be as in theorem \ref{disjoint segments slide}. Without loss of generality we have that for all \(p_i \in P\) we have \(\{(x_i,t_i + b):b\in Y_{1,i}\}\subseteq MS_{1\rightarrow 2}\) and \(\{(x_i,t_i+c):c\in X_i\cup Y_{0,i}\}\subseteq (MS_{1\rightarrow 2})^c\).\\\par
Case 1: \(A\) is countable.\\
Let \(k_0=0,X_0=Y_0=Z_0=\{a_0\},b_0=c_0=a_0\).
\begin{enumerate}
\item Let \(X_{n+1}:=\{a_i\in A:i\leq k_n \text{ and }(a_i,a_{n+1}+c_i)\in MS_{1\rightarrow 2}\}\).
\item Let \(k_{n+1}>k_n\) be such that \((X_{n+1}+min(A\backslash \cup_{i=0}^n{Z_i})-a_{k_{n+1}})\cap (\cup_{i=0}^n{Z_i})=\emptyset\). This must exist as 
\(|(X_{n+1}+min(A\backslash \cup_{i=0}^n{Z_i}))-(\cup_{i=0}^{n}{Z_i})|\) is finite.
\item Let \(b_{n+1}=min(A\backslash \cup_{i=0}^n{Z_i})-a_{k_{n+1}}\). Note this means that \((X_{n+1}+b_{n+1})\cap (\cup_{i=0}^n{Z_i})=\emptyset\) by the definition of \(k_{n+1}\).
\item Let \(c_i\) for \(k_n<i\leq k_{n+1}\) be defined by: \((a_{k_{n+1}},a_{n+1}+c_{k_{n+1}})\in MS_{1\rightarrow 2} \) and for all other \(k_n<i\leq k_{n+1}\) and \(j\leq n+1\) we have \((a_i,a_j+c_i)\notin MS_{1\rightarrow 2}\). This can be done by the definition of \(S_1,S_2\).
\item Let \(Y_{n+1} = X_{n+1}\cup \{a_{k_{n+1}}\}\) 
\item Let \(Z_{n+1}=Y_{n+1}+b_{n+1}\). Note that \(Z_{n+1}\) is disjoint from \(\cup_{i=0}^n Z_i\) (by 3).
\end{enumerate}
As each \(Z_n\) contains \(min(A\backslash \cup_{i=0}^{n-1}{Z_i})\) (by 3,5,6) we have that \(\cup_n{Z_n}=A\). In addition (by 6) we have that the \(Z_n\) are disjoint. So we have the \(Z_n\) partition \(A\).\\
Let \(S_3=v_{f_3}\) where \(f_3\) is defined by \((a_i)f_3=-c_i\). Let \(S_4=h_{f_4}\) where \(f_4\) is defined by \((a_i)f_4=b_i\). Let \(S_5=v_{f_5}\) where \(f_5\) is defined by \(f_5=a_i-a_n\) where \(n\) corresponds to the unique \(Z_n\) containing \(a_i\)\\
We have that \(Y_n=\{a_i \in A:(a_i,a_n)\in MS_{1\rightarrow 3}\}\) as if \(a_i\in Y_n\) we have that \((a_i,a_n+c_i)\in MS_{1\rightarrow 2}\) (by 1,4,5) and if \(a_i\notin Y_n\) we have that \((a_i,a_n+c_i)\notin MS_{1 \rightarrow 2}\) (by 1,5 if \(i \leq k_n\) and by 4,5 if \(i>k_n\)).\\
We therefore have (by 6) that \(Z_n=\{a_i\in A:(a_i-b_n,a_n)\in MS_{1 \rightarrow 3}\}=\{a_i\in A:(a_i,a_n)\in MS_{1 \rightarrow 4}\}\).\\
So we have that if \((a_i,a_n)\in MS_{1 \rightarrow 4}\) then \(a_i \in Z_n\) and thus \((a_i,a_n)S_5=(a_i,a_n-a_n+a_i)=(a_i,a_i)\).\\\par
Case 2: \(A\) is uncountable.\\
For the rest of this proof if we have an ordinal \(i=\alpha+k\) for some limit ordinal \(\alpha\) and \(k\in \aleph_0\) we use the notation \(2i\) to mean \(\alpha+2k\). We will also re-index \(A \backslash \{id_A\}=\{a_i:i<|A|\}\)\\ 
Let \((A_i)_{i<|A|}\) be a cofinal  chain for \(A\) such that \(A_0=\{id_A\}\) and for all \(i<|A|\) we have \(A_i\) is a group, \(a_{2i},a_{2i+1}\in A_{i+1}\) and \([A_{i+1}:A_{i}]\geq 4\). This can be done as follows:
\begin{enumerate}
\item if \(i=0\) then let \(A_i=\{id_A\}\)
\item if \(i\) is a successor ordinal let \(A_i:=\langle \{a_j:j\leq min\{k \geq 2i+1:[\langle \{a_\alpha:\alpha \leq k\}\rangle:A_i]\geq 4\}\}\rangle\)
\item if \(i\) is a limit ordinal then \(A_i:=\cup_{j<i} A_j\)
\end{enumerate}
For \(i<|A|\) we define \(c_i,d_i,m_i\) be transfinite recursion.
\begin{enumerate}
\item Let \(c_i\in A_{i+1}\backslash A_i\)
\item Let \(d_i \in A_{i+1}\backslash ((A_{i}+c_i)\cup (A_{i}+c_i^{-1})\cup A_i)\). Note that we can do this as \(|A_{i+1}/A_{i}|\geq 4\). It follows from the definition of \(d_i\) that \(a_0,c_i,d_i,c_i+d_i\) are in different cosets of \(A_i\) in \(A_{i+1}\).
\item For \(a_j \in A_{i+1}\backslash A_{i}\) by the definition of \(S_1,S_2\) we can find \(m_j\) be such that:
\begin{enumerate}
\item \((a_j,a_k-m_j)\notin MS_{1 \rightarrow 2}\) for \(k<2i\) and \((a_j,-m_j)\notin MS_{1\rightarrow 2}\)
\item If \(a_j\in (A_i-c_i)\) then \((a_j,a_{2i}-m_j)\notin MS_{1\rightarrow 2}\)
\item If \(a_j= a_k-d_i-c_i\) for some \(a_k\in A_i\) and \((a_k,a_{2i+1}-m_k)\notin MS_{1\rightarrow 2}\) then \((a_j,a_{2i}-m_{j})\in MS_{1\rightarrow 2}\)
\item If \(a_j= a_k-d_i-c_i\) for some \(a_k\in A_i\) and \((a_k,a_{2i+1}-m_k)\in MS_{1\rightarrow 2}\) then \((a_j,a_{2i}-m_{j})\notin MS_{1\rightarrow 2}\)
\item If \(a_j\in A_{i+1}\backslash (A_i \cup (A_i-c_i)\cup (A_i-(d_i+c_i)))\) then \((a_j,a_{2i}-m_j)\in MS_{1\rightarrow 2}\)
\item If \(a_j= a_k+c_i+d_i\) for some \(a_k\in A_i\) and \((a_k,a_{2i}-m_k)\notin MS_{1\rightarrow 2}\) then \((a_j,a_{2i+1}-m_{j})\in MS_{1\rightarrow 2}\)
\item If \(a_j= a_k+c_i+d_i\) for some \(a_k\in A_i\) and \((a_k,a_{2i}-m_k)\in MS_{1\rightarrow 2}\) then \((a_j,a_{2i+1}-m_{j})\notin MS_{1\rightarrow 2}\)
\item If \(a_j\in A_{i+1}\backslash (A_i \cup (A_i+(c_i+d_i)))\) then \((a_j,a_{2i+1}-m_j)\notin MS_{1\rightarrow 2}\)
\end{enumerate}
\end{enumerate}
Let \(S_3=v_{f_3}\) where \(f_3\) is defined by \((a_i)f_3=m_i\) and \((id_A)f_3\) is such that \((id_A,-(id_A)f_3)\in MS_{1\rightarrow 2}\). Let \(S_4=h_{f_4}\) where \(f_4\) is defined by \((a_{2i})f_4=c_i\), \((a_{2i+i})f_4=-d_i\) and \((id_A)f_4=id_A\). Note that for all non-identity elements  \(x\in A\) there is a unique \(i<|A|\) such that \(x\in A_{i+1}\backslash A_i\). 
We will now show that for all \(x\in A\) there is at most one point in \(MS_{1\rightarrow 4}\cap v_x\). For \(i<|A|\) we have (by 3b and 3h) that:
\[(-c_i,a_{2i}-(-c_i)f_3)\notin MS_{1\rightarrow 2}\implies (-c_i,a_{2i})\notin MS_{1\rightarrow 3} \implies (id_A,a_{2i})\notin MS_{1\rightarrow 4}\]
\[(d_i,a_{2i+1}-(d_i)f_3)\notin MS_{1\rightarrow 2}\implies (d_i,a_{2i+1})\notin MS_{1\rightarrow 3} \implies (id_A,a_{2i+1})\notin MS_{1\rightarrow 4}\]
Let \(a_{j}\in A_{i+1}\backslash A_{i}\).\\
For \(k<i\) we have \( a_j -c_k\in A_{i+1}\backslash A_{i}\) and \( a_j +d_k\in A_{i+1}\backslash A_{i}\) therefore (by 3a) we have \((a_j-c_k,a_{2k}-(a_j-c_k)f_3)\notin MS_{1\rightarrow 2}\) and \((a_j+d_k,a_{2k+1}-(a_j+d_k)f_3)\notin MS_{1\rightarrow 2}\).  For \(k>i\) we also have these two conditions (by 3b and 3h).\\
Therefore for \(i \neq k\) we have:
\[ (a_j-c_k,a_{2k}-(a_j-c_k)f_3)\notin MS_{1\rightarrow 2} \implies (a_j-c_k,a_{2k})\notin MS_{1\rightarrow 3}\implies (a_j,a_{2k})\notin MS_{1\rightarrow 4}\implies (a_j,a_j+a_{2k}-a_{2i})\]
\[ (a_j+d_k,a_{2k+1}-(a_j+d_k)f_3)\notin MS_{1\rightarrow 2} \implies (a_j+d_k,a_{2k+1})\notin MS_{1\rightarrow 3}\implies (a_j,a_{2k+1})\notin MS_{1\rightarrow 4}\]
If \(a_j= a_k-d_i\) for some \(a_k\in A_i\) then (by 3d) we have one of:
\[(a_k,a_{2i+1}-m_k)\notin MS_{1\rightarrow 2}\implies (a_k,a_{2i+1}) \notin MS_{1\rightarrow 3} \implies (a_j,a_{2i+1})\notin MS_{1\rightarrow S_4}\]
\[(a_j-c_{i},a_{2i}-(a_j-c_i)f_3)\notin MS_{1\rightarrow 2}\implies (a_j-c_i,a_{2i})\notin MS_{1\rightarrow 3} \implies (a_j,a_{2i})\notin MS_{1\rightarrow 4} \]
If \(a_j= a_k+c_i\) for some \(a_k\in A_i\). then (by 3g) we have one of:
\[(a_k,a_{2i}-m_k)\notin MS_{1\rightarrow 2}\implies (a_k,a_{2i})\notin MS_{1\rightarrow 3} \implies (a_j,a_{2i})\notin MS_{1\rightarrow 4}\]
\[(a_j+d_i,a_{2i+1}-(a_j+d_i)f_3)\notin MS_{1\rightarrow 2}\implies (a_j+d_i,a_{2i+1})\notin MS_{1\rightarrow 3}\implies (a_j,a_{2i+1})\notin MS_{1\rightarrow 4}\]\\
Otherwise by (3h) we have:
\[(a_j+d_i,a_{2i+1}-(a_j+d_i)f_3)\notin MS_{1\rightarrow 2} \implies (a_j+d_i,a_{2i+1})\notin MS_{1\rightarrow 3}\implies (a_j,a_{2i+1})\notin MS_{1\rightarrow 4}\]
We therefore have that \(v_{id_A}\cap MS_{1\rightarrow 5}\) contains at most \((id_A,id_A)\) and for all \(a_j\) we have at most one of \((a_j,a_{2i}),(a_j,a_{2i+1})\) in  \(v_{a_j}\cap M_{S_{1\rightarrow 4}}\) and no other points. Thus we can construct a vertical slide \(S_5\) such that \(MS_{1\rightarrow 5} \subseteq \{(a,a):a\in A\}\).
\end{proof}
\begin{theorem}\label{realise moiety perm}
Let \(M\) be a moiety and let \(p\in Sym_{\Omega}(M)\) then there exist 11 slides \(S_1,S_2\ldots S_{11}\) such that \(S_{1\rightarrow 11}\vert_{M}=p\vert_{M}\).
\end{theorem}
\begin{proof}
This proof comes from \cite{shuffle1} claim 11.\\
Let \(S_1,S_2,S_3,S_4,S_5\) be as in theorem \ref{moiety to diag} and let \(S_n:=S_{12-n}^{-1}\) for \(n>6\). Without loss of generality assume that \(S_5\) is a vertical slide.
Let \(I:M \rightarrow A\) be defined by \((x)S_{1\rightarrow 5}=((x)I,(x)I)\).\\
 Let \(f_1:A\rightarrow A\) and \(f_2:A\rightarrow A\) be defined by:
\[(a)f_1=\left\{\begin{array}{lr}
(a)I^{-1}pI-a& a\in image(I)\\
a_0 & \text{otherwise}\\
\end{array}\right\} \ \ \ \ (a)f_2=\left\{\begin{array}{lr}
a-(a)I^{-1}p^{-1}I& a\in image(I)\\
a_0 & \text{otherwise}\\
\end{array}\right\}\]
Let \(S_5':=v_{f_1}\) and \(S_6:=h_{f_2}\).\\
We now have for \(m \in M\):
\begin{align*}
(m)S_{1\rightarrow 5}S_5'S_{6\rightarrow 11}&=((m)I,(m)I)S_5'S_{6\rightarrow 11}\\
&=((m)I,(m)I+(m)II^{-1}pI-(m)I)S_{6\rightarrow 11}\\
&=((m)I,(m)pI)S_{6\rightarrow 11}\\
&=((m)I+(m)pI-(m)pII^{-1}p^{-1}I,(m)pI)S_{7\rightarrow 11}\\
&=((m)I+(m)pI-(m)I,(m)pI)S_{7\rightarrow 11}\\
&=((m)pI,(m)pI)S_{7\rightarrow 11}\\
&=(m)p\\
\end{align*}
Thus we have achieved the desired result with 12 slides but as \(S_5\) and \(S_5'\) are both vertical slides it follows that \(S_5S_5'\) is also a single vertical slide ans thus we have the required result.
\end{proof}
\begin{theorem}\label{moiety realised}
Let \(M\) be a moiety of \(\Omega\) and let \(p\in Sym_{\Omega}(M)\). Then there are slides \(S_1,S_2\ldots S_{44}\) such that \(p=S_{1\rightarrow 44}\).
\end{theorem}
\begin{proof}
This proof comes from \cite{shuffle1} claim 12.\\
 Let \(M_1,M_2\) be a partition of \(M^c\) into moieties. By theorem \ref{commutator result} let \(p_1,p_2\in Sym_{\Omega}(M)\) be such that \(p_1p_2p_1^{-1}p_2^{-1}=p\). By theorem \ref{realise moiety perm} let \(S_1,S_2 \ldots S_{22}\) be such that \(S_{1\rightarrow 11}\vert_{M\cup M_1}=p_1\vert_{M\cup M_1}\) and \(S_{12\rightarrow 22}\vert_{M\cup M_2}=p_2\vert_{M\cup M_2}\). Let \(S_n=S_{34-n}^{-1}\) for \(23 \leq n \leq 33\) and \(S_n=S_{56-n}^{-1}\) for \(34 \leq n \leq 44\).\\
Let \((x,y)\in \Omega\).
\begin{align*}
(x,y)S_{1\rightarrow 44}&=\left\{
	\begin{array}{lr}
	(x,y)p_1S_{12\rightarrow 44}& (x,y)\in M\\
	(x,y)S_{12\rightarrow 44}& (x,y)\in M_1\\
	((x,y)S_{1\rightarrow 11})S_{12\rightarrow 44}& (x,y)\in M_2
	\end{array}
	\right\}\\
&=\left\{
	\begin{array}{lr}
	(x,y)p_1p_2S_{23\rightarrow 44}& (x,y)\in M\\
	((x,y)S_{12\rightarrow 22})S_{23\rightarrow 44}& (x,y)\in M_1\\
	((x,y)S_{1\rightarrow 11})S_{23\rightarrow 44}& (x,y)\in M_2
	\end{array}
	\right\}\\
	&=\left\{
	\begin{array}{lr}
	(x,y)p_1p_2p_1^{-1}S_{34\rightarrow 44}& (x,y)\in M\\
	((x,y)S_{12\rightarrow 22})S_{34\rightarrow 44}& (x,y)\in M_1\\
	(x,y)S_{34\rightarrow 44}& (x,y)\in M_2
	\end{array}
	\right\}\\
		&=\left\{
	\begin{array}{lr}
	(x,y)p_1p_2p_1^{-1}p_2^{-1}& (x,y)\in M\\
	(x,y) & (x,y)\in M_1\\
	(x,y)& (x,y)\in M_2
	\end{array}
	\right\}\\
&=(x,y)p
\end{align*}
\end{proof}
\begin{theorem}
Let \(p \in Sym(\Omega)\), if \(\Omega\) is countable then there exist 88 slides \(S_1,S_2\ldots S_{88}\) such that \(p=S_{1\rightarrow 88}\)
\end{theorem}
\begin{proof}
This proof comes from \cite{shuffle1} claim 13.\\
By theorem \ref{temp final} we can write \(p\) as \(p_1p_2\) where \(p_1\) and \(p_2\) fix a moiety of points (1-cycles are fixed points and 2-cycles are non-fixed points). So by theorem \ref{moiety realised} we can find \(S_1,S_2\ldots S_{88}\) such that \(p_1=S_{1\rightarrow 44}\) and \(p_2=S_{45\rightarrow 88}\) so we have \(p=p_1p_2=S_{1\rightarrow 88}\).
\end{proof}
\begin{theorem}
Let \(p \in Sym(\Omega)\), if \(\Omega\) is uncountable then there exist 55 slides \(S_1,S_2\ldots S_{55}\) such that \(p=S_{1\rightarrow 55}\).
\end{theorem}
\begin{proof}
Let \(P\) be the partition of \(\Omega\) into disjoint cycles of \(p\). We have that \(|P|=|\Omega|\) as each cycle is countable and \(\Omega\) is uncountable. Let \(M_p\) be a moiety of \(P\) and let \(M=\cup M_p\). We have that \(p = p\vert_{M}p\vert_{M^c}\). By theorem \ref{realise moiety perm} let \(S_1,S_2 \ldots S_{11}\) be such that \(p\vert_{M}=S_{1\rightarrow 11}\vert_{M}\). By theorem \ref{moiety realised} we can find \(S_{12\rightarrow 55}\) such that \((S_{1\rightarrow 11})^{-1}(p\vert_{M})(p\vert_{M^c}) =S_{12\rightarrow 55}\) as this product fixes the moiety \(M\) pointwise. We therefore have that 
\[S_{1\rightarrow 55}=(S_{1\rightarrow  11})S_{12\rightarrow 55}=(S_{1\rightarrow 11})(S_{1\rightarrow11})^{-1}(p\vert_{M})(p\vert_{M^c}) =(p\vert_{M})(p\vert_{M^c})=p\]
\end{proof}
\section{Products of Abelian Groups}
In the previous section on shuffling the plane the works of Miklos Abert, Tamas Keleti and Peter Komjath gave us a means for expressing any element of an infinite symmetric group as a product of 'slides'. This gives us a way of writing any infinite symmetric group as a product of less than 100 abelian groups. In this section we add utilise the tecniques of Akos Seress found in \cite{abelian prod} attempt to decrease this bound further. 
\begin{defn}
Let \(V\) be used to denote the group of all vertical slides of \(\Omega\), and similarly let \(H\) denote the group of all horizontal slides of \(\Omega\). Note that these groups are abelian as \(A\) is abelian.
\end{defn}
\begin{defn}
Let \(b\in A\), we call a set \(D\subseteq \{(a,t+a):a\in A\}\) a t-diagonal segment. 
\end{defn}
Observe that if \(a\neq b\) then all a-diagonal segments are disjoint from all b-diagonal segments.
\begin{theorem}\label{full diagonal}
Let \(t\in A\) and \(D\) be a t-diagonal segment, then the group \(HV\) acts fully on \(D\).
\end{theorem}
\begin{proof}
The following proof is spiritually the same as the proof of Lemma 3 in \cite{abelian prod}:\\
Let \(g\in Sym(D)\) and
let \(f_1:A\rightarrow A\) and \(f_2:A\rightarrow A\) be defined by:
\[(a)f_1=(a-t,a)g\pi_1-a\ \ \ \ (a)f_2=a+t-(a,a+t)g^{-1}\pi_2\]
Let \((p,p+t)\in D\), we have that:
\begin{align*}
(p,p+t)h_{f_1}v_{f_2}&=((p,p+t)g\pi_1,p+t)v_{f_2}\\
&=((p,p+t)g\pi_1,p+t+(p,p+t)g\pi_1+t-((p,p+t)g\pi_1,(p,p+t)g\pi_1+t)g^{-1}\pi_2)\\
&=((p,p+t)g\pi_1,p+t+(p,p+t)g\pi_1+t-((p,p+t)g\pi_1,(p,p+t)g\pi_2)g^{-1}\pi_2)\\
&=((p,p+t)g\pi_1,p+t+(p,p+t)g\pi_1+t-(p,p+t)gg^{-1}\pi_2)\\
&=((p,p+t)g\pi_1,p+t+(p,p+t)g\pi_1+t-(p+t))\\
&=((p,p+t)g\pi_1,(p,p+t)g\pi_1+t)\\
&=((p,p+t)g\pi_1,(p,p+t)g\pi_2)\\
&=(p,p+t)g\\
\end{align*}
So we have that \(h_{f_1}v_{f_2}\in HV\) satisfies that \((h_{f_1}v_{f_2})|_{D}=g\) as required.
\end{proof}
\begin{theorem} \label{miss moiety}
Let \(M\) be a moiety of \(\Omega\) and \(t\in A\), then there exists an element of \(h_1v_1h_2\in HVH\) such that \(Mh_1v_1h_2 \cap D_t= \emptyset\) for all t-diagonal segments \(D_t\).
\end{theorem}
\begin{proof}
By theorem \ref{Soon to be all spots} we have that there is an element \(S\in H\cup V\) such that \(h_x\) is MS-spotty for all \(x\in A\) or \(v_x\) is MS-spotty for all \(x\in A\).\\
Case 1: If \(S \in H\) and we have \(v_x\) is MS-spotty for all \(x\in A\) then let \(p_x\in v_x \backslash MS\) for all \(x\in A\). Let \(f:A\rightarrow A\) be defined by \((a)f=a+t-p_a\). It follows that for all \((x,y)\in M\) we have \((x,y)Sv_f\notin D_t\) and thus we can choose \(h_1=S,v_1=v_f\) and \(h_2=id_A\).\\
Case 2: If \(S \in H\) and we have \(h_x\) is MS-spotty for all \(x\in A\) then we can make a similar argument by using only an element of \(H\) (viewed as a product of two elements of \(H\))\\
Case 3: If \(S \in V\) and we have \(h_x\) is MS-spotty for all \(x\in A\) then we can make a similar argument by letting \(h_1=id_A\) and \(v_1=S\).\\
Case 4: If \(S \in V\) and we have \(v_x\) is MS-spotty for all \(x\in A\) then we can make a similar argument by using only an element of \(V\) (viewed as a product of two elements of \(V\))
\end{proof}
\begin{theorem}\label{full diagonal complement}
Let \(M\) be a moiety of \(\Omega\) whose complement is a t-diagonal segment. Then there exist abelian groups \(H_M,V_M\) such that \(H_MV_M\) acts fully on \(M\).
\end{theorem}
\begin{proof}
As \(M^c\) is a t-diagonal segment we have by theorem \ref{full diagonal} that \(HV\) acts fully on \(M^c\). Let \(\phi:M\rightarrow M^c\) be a bijection. Let \(I_M:\Omega \rightarrow \Omega\) be the involution defined by:
\[(x)I_M=\left\{\begin{array}{lr}
(x)\phi & x\in M\\
(x)\phi^{-1} & x \in M^c 
\end{array}\right\}\]
As \(HV\) acts fully on \(M^c\) it follows that \(I_M(HV)I_M^{-1}\) acts fully on \(M\). So we also have that \(I_MHI_M^{-1}I_MVI_M^{-1}=(I_MHI_M^{-1})(I_MVI_M^{-1})\) acts fully on \(M\). Let \(H_M:=I_MHI_M^{-1}\) and \(V_M:=I_MVI_M^{-1}\). As the conjugates of abelian groups these are abelian groups and we have the required result.
\end{proof}
Now that we have a few theorems and definitions we will prove the main result of this section. This prove will makes use of a new group \(C\) very similar to the group \(D\) in \cite{abelian prod} in the same way as the group \(D\) is used in lemma 5 of \cite{abelian prod}.
\begin{theorem}
The group \(Sym(\Omega)\) can be expressed as the product of 10 abelian groups.
\end{theorem}
\begin{proof}
Let \(L\) be a moiety of \(A\), \(t\in A\backslash \{id_A\}\), \(D_L:=\{(a,a):a\in L\}\),  \(D_L':=\{(a,a+t):a\in L\}\). Let \(P=\{p_i:i<|\Omega|\}\) be a partition of \(D_L^c\) into countable sets such that there are \(|\Omega|\) sets of all cardinalities less that or equal to \(\aleph_0\). For all \(i<|\Omega|\) let \(c_i\) be the group generated by \(|p_i|\)-cycle on \(p_i\). Let \(C=\prod_{i<|\Omega|}c_i\). Finally let \(H_{D_L^c},V_{D_L^c},H_{D_L'^c},V_{D_L'^c}\) be as in theorem \ref{full diagonal complement}.\\
Note that all the above groups are abelian as \(C\) is a product of cyclic groups acting on disjoint sets and the others are abelian by construction.
We will show that \(Sym(\Omega)=HVHH_{D_L'^c}V_{D_L'^c}H_{D_L^c}V_{D_L^c}CV_{D_L^c}H_{D_L^c}\).\\
Let \(g\in Sym(\Omega)\), let \(M=\{(a,a):a\in A\}g^{-1}\). As the image of a moiety under a permutation we have that \(M\) is a moiety. By theorem \ref{miss moiety} we have that there is an element \(g'\in HVH\) such that \(Mg'\cap \{(a,a+t):a\in A\}=\emptyset\).\\
Observe that \(\{(a,a+t):a\in L^c\}\subseteq (Mg')^c\backslash D_{L}'\) and \(|\{(a,a+t):a\in L^c\}|=|\Omega|=|(\{(a,a):a\in A\}\cup D_L')^c|\).\\
Let \(\phi: (Mg')^c\backslash D_{L}' \rightarrow (\{(a,a):a\in A\}\cup D_L')^c\) be a bijection.\\
Let \(f\in Sym(D_L'^c)\) be defined by:
\[(x)f=\left\{\begin{array}{lr}
(x)g'^{-1}g& x\in Mg'\\
(x)\phi & otherwise
\end{array}\right\}\]
As \(H_{D_L'^c}V_{D_L'^c}\) acts fully on \(D_L'^c\) there exists \(g''\in H_{D_L'^c}V_{D_L'^c}\) such that \(g''|_{D_L'^c}=f\). We now have that \((g'g'')|_M=g|_M\) and \(g'g''\in HVHH_{D_L'^c}V_{D_L'^c}\).\\
It therefore follows that \(s:=(g'g'')^{-1}g \in PStab(\{(a,a):a\in A\})= Sym_{\Omega}(\{(a,a):a\in A\}^c)\leq Sym_{\Omega}(D_L^c)\). By definition of \(C\) there is an element \(s'\in C\) such that \(s'\) has the same disjoint cycle shape as \(s\). As we have \(s,s'\in Sym_{\Omega}(D_L^c)\) have the same disjoint cycle shape there exists \(w\in Sym_{\Omega}(D_L^c)\) such that \(w s' w^{-1}=s\). As \(H_{D_L^c}V_{D_L^c}\) acts fully on \(D_L^c\) we have that there exists \(w'\in H_{D_L^c}V_{D_L^c}\) such that \(w'|_{D_L^c}=w|_{D_L^c}\).\\
Let \(p\in \Omega\):
\[(p)w's'w'^{-1}=\left\{\begin{array}{lr}
(p)w'w'^{-1} & p\in D_L\\
(p)s & otherwise
\end{array}\right\}=(p)s\]
It follows that \(s=w's'w'^{-1}\in H_{D_L^c}V_{D_L^c}CV_{D_L^c}H_{D_L^c}\). As therefore \(g=g'g''(g'g'')^{-1}g=g'g''s\in HVHH_{D_L'^c}V_{D_L'^c}H_{D_L^c}V_{D_L^c}CV_{D_L^c}H_{D_L^c}\) as required.
\end{proof}
\chapter{Maximal subgroups of infinite symmetric groups}
In this chapter we aim to construct large families of maximal subgroups of $Sym(\Omega)$ has $2^{2^{\vert \Omega \vert}}$ maximal subgroups. To do this we will be building groups from ultrafilters.
\section{Ultrafilters}
\begin{defn}
Given a set $S$, a \textit{filter} on $S$ is defined to be a collection of subsets $F$ of $S$ satisfying:
\begin{enumerate}
\item \(S\in F\)
\item For all \(A,B \in F\ A\cap B \in F\)
\item If \(B\supseteq A \in F\) then \(B\in F\)
\end{enumerate}
\end{defn}
\begin{defn}
Given a set $S$, an \textit{ultrafilter} on $S$ is defined to be a collection of subsets $\mathcal{U}$ of $S$ satisfying:\\
1. $\mathcal{U}$ is a filter\\
2. $\emptyset \notin \mathcal{U}$\\
3. There is no filter $\mathcal{U}'$ such that  $\mathcal{U} \subset \mathcal{U}' \subset P(S)$
\end{defn}
\begin{theorem}\label{Ultrafilter complement}
Let $X$ be a set, let $S\subseteq X$ and let $\mathcal{U}$ be an ultrafilter on $X$. Then precisely one of $S$ and $S^c$ is in $\mathcal{U}$.
\end{theorem}\par
Proof: Suppose that both $S$ and $S^c$ are in $\mathcal{U}$ then $S\cap S^c= \emptyset$ is also in $\mathcal{U}$, a contradiction.\\
Suppose that neither $S$ nor $S^c$ are in $\mathcal{U}$, then let $\mathcal{V}$ be defined as follows:
$$\mathcal{V}:= \{V:V \supseteq S \cap U \textsl{ for some } U \in \mathcal{U} \}$$
We will now show that $\mathcal{V}$ is a filter contradicting condition 3 for ultrafilters.\\
1. As $\mathcal{U}$ is a filer $X \in \mathcal{U}$ and therefore as $X \supseteq X \cap S$ it follows that $X \in \mathcal{V}$.\\
2. If $A \supseteq A'\cap S$ for some $A'\in \mathcal{U}$ and $B \supseteq B'\cap S$ for some $B'\in \mathcal{U}$, then $A\cap B \supseteq A' \cap B'\cap S$ and as $A' \cap B' \in \mathcal{U}$ it follows that $A\cap B \in \mathcal{V}$.\\
3. If $B\supseteq A$ for some $A \in \mathcal{V}$ then $A \supseteq A'\cap S$ for some $A' \in U$. It is clear that $B \supseteq A' \cap S$ as well and therefore  $B \in \mathcal{V}$.\\
We therefore have that $\mathcal{V}$ is a filter.\\
We have that $M_1 \supseteq X\cap S$ so $S \in \mathcal{V}$. In addition for all $U\in \mathcal{U}$ we have $U\supseteq U\cap S$. So $\mathcal{U} \supset \mathcal{V}$.\\
It remains to show that $\mathcal{V} \subset P(X)$. Suppose that $\emptyset \in \mathcal{V}$ then $\emptyset \supseteq U\cap S$ for some $U\in \mathcal{U}$. It follows that $\emptyset = U\cap S$ and therefore $U\subseteq S^c$, so we have that $S^c\in \mathcal{U}$ a contradiction.
\begin{theorem}\label{moiety determined}
Ultrafilters on an infinite set $X$ are uniquely determined by their moiety elements.
\end{theorem}\par
Let $\mathcal{U}_1$ and $\mathcal{U}_2$ be ultrafilters with the same moiety elements.\\
Let $U \in \mathcal{U}_1$ if there is some moiety contained in $U$ then $U$ is also in $\mathcal{U}_2$. If not then $\vert X \backslash U \vert = \vert X \vert$ and we can therefore construct disjoint moieties $M_1$ and $M_2$ such that $M_1 \cup M_2 = X \backslash U $. It therefore follows that $M_1 \cup U$ and $M_2 \cup U$ are moieties and elements of $\mathcal{U}_1$. It therefore follows that $M_1 \cup U$ and $M_2 \cup U$ are elements of $\mathcal{U}_2$ and therefore their intersection $U$ is in $\mathcal{U}_2$ as required. We therefore have that $\mathcal{U}_1 \subseteq \mathcal{U}_2$ and by symmetry $\mathcal{U}_2 \subseteq \mathcal{U}_1$. So $\mathcal{U}_1 = \mathcal{U}_2$ as required. $\qed$
\begin{defn}
Let $X$ be a set and let $S \subseteq P(X)$. We say that $S$ has the \textit{finite intersection property} if any finite subset of $S$ has non-empty intersection.
\end{defn}
\begin{theorem}\label{ultrafilter lemma}
(Ultrafilter Lemma) let $X$ be a set and let $S \subseteq P(X)$ be such that $S$ has the finite intersection property. Then there is an ultrafilter on $X$ containing $S$.
\end{theorem}\par
Proof: The following proof comes from theorem 1.8 in \cite{no ultrafilters}.\\ Let $S$ be a set with the finite intersection property. Let $F(S)$ denote the set of all filters which contain $S$ and don't have $\emptyset$ as an element. The set $F(S)$ is partially ordered by $\subseteq$. Let $(F_i)_{i<\alpha}$ for some ordinal $\alpha$ be a chain. Then $\bigcup_{i<\alpha} F_i$ is an upper bound for $(F_i)_{i<\alpha}$. So if $\bigcup_{i<\alpha} F_i \in F(S)$ it follows by theorem \ref{zorn's lemma} that $F(S)$ has a maximal element $\mathcal{U}$.\\
It is clear that $\bigcup_{i<\alpha} F_i \in F(S)$ doesn't contain $\emptyset$. Consider the set $\{U:U\supseteq s \textsl{ for some }s\in S\}$, clearly this set is a filter as $S$ has the finite intersection property. It follows that $\bigcup_{i<\alpha} F_i$ is non-empty and therefore $S \subseteq \bigcup_{i<\alpha} F_i $\\
We will now show that $\bigcup_{i<\alpha} F_i$ is a filter.\\
1. As the non-empty union of filters $X \in \bigcup_{i<\alpha} F_i$.\\
2. Let $U_1,U_2 \in \bigcup_{i<\alpha} F_i \in F(S)$, then $U_1\in F_{i_1}$ for some $i_1$ and $U_2\in F_{i_2}$ for some $i_2$ and therefore $U_1,U_2 \in F_{max\{i_1,i_2\}}$. So $U_1 \cap U_2 \in F_{max\{i_1,i_2\}} \subseteq \bigcup_{i<\alpha} F_i $.\\
3. Let $U_1 \in \bigcup_{i<\alpha} F_i \in F(S)$ and let $U_2 \subseteq U_1$, then $U_1\in F_i$ for some $i$ and therefore $U_2 \in F_i$ and therefore $U_2 \in \bigcup_{i<\alpha} F_i$.\\
As $\mathcal{U} \in F(S)$ it follows that $\mathcal{U}$ is a filter and $\emptyset \notin \mathcal{U}$. If $F$ is a filter such that $F \supseteq \mathcal{U}$ then $F\in F(S)$ and therefore $F \subseteq \mathcal{U}$. So we have that $\mathcal{U}$ is an ultrafilter as required. $\qed$
\begin{defn}
Let $\alpha, \beta$ be cardinals. Let $L \subseteq \beta^\alpha$, we say that $L$ has \textit{large oscillation} if the following is satisfied:\\
If $n\in \mathbb{N}$ and $\{f_i: i <n \}\subseteq L$ are distinct and $\{b_i:i < n\}\subseteq \beta$ then there is an $a^*\in \alpha$ such that $(a^*)f_i = b_i$ for all $i< n$.
\end{defn}
\begin{theorem}\label{large osc}
Let $\kappa$ be an infinite cardinal. Then there exists $L \subseteq 2^\kappa$ such that $\vert L \vert = 2^\kappa$ and $L$ has large oscillation. Note that here we are viewing 2 as the set $\{0,1\}$ in addition to a cardinal number.
\end{theorem}\par
Proof: The following proof comes from the proof of theorem 2.2 in \cite{no ultrafilters}.\\
Let $\mathcal{T}$ be defined by:
$$\mathcal{T} := \{(s,S,\phi): s\textsl{ is a finite subset of }\kappa, S\in P(P(s)), \phi \in 2^{S}\}$$
As $\vert s \vert$ is finite it follows that $\vert S \vert$ is finite and therefore $\vert 2^S \vert$ is finite. It follows that $\vert \mathcal{T} \vert =\vert \kappa \vert$ as we have $\kappa$ choices for $s$ and finitely many choices for $S$ and $\phi$. It is therefore sufficient to find $L \subseteq 2^\mathcal{T}$ with the required properties.\\
Let $f:P(\kappa) \rightarrow 2^\mathcal{T}$ be defined by:
$$(\Sigma)f = f_\Sigma$$
where $f_\Sigma:\mathcal{T} \rightarrow 2$ is defined by:
$$f_\Sigma(s,S,\phi):=\left\{
    \begin{array}{lr}
      (\Sigma \cap s)\phi &  \Sigma \cap s \in S\\
      0 &  \Sigma \cap s \notin S\\
    \end{array}
    \right\}$$
Choose $L = image(f)$\\
To show that $\vert L \vert = 2^\kappa$ it suffices to show that $f$ is injective as $\vert P(\kappa) \vert = 2^\kappa$.\\
Let $\Sigma_1, \Sigma_2 \in P(\kappa)$ be distinct. Wlog $\Sigma_1 \text{\textbackslash} \Sigma_2 \neq \emptyset $.\\
Let $x \in \Sigma_1 \text{\textbackslash} \Sigma_2$. Let $s= \{x\}$, $S= \{s\}$, $\phi$ be such that $(s)\phi = 1$ then:
$$((\Sigma_1)f)(s,S,\phi)=(\Sigma_1 \cap s)\phi=(s)\phi= 1 \neq 0 = (\emptyset)\phi=(s\cap \Sigma_2)\phi=((\Sigma_2)f)(s,S,\phi)$$
It follows that $(\Sigma_1)f \neq (\Sigma_2)f$ and thus $f$ is injective.\\
It remains to show that $L$ has large oscillation. Let $n \in \mathbb{N}$, $\{f_{A_m}:m<n\}\subseteq L$ be distinct and $\{k_m:m<n\}\subseteq 2(=\{0,1\})$. For each $(m_1,m_2)$ such that $m_1<m_2<n$. Let $a_{m_1,m_2}\in A_{m_1} \triangle A_{m_2}$.\\
 Let $s:=\{a_{m_1,m_2}:m_1 < m_2 < n\}$\\
 Let $S:=\{A_m \cap s: m<n\}$\\
 Let $\phi:S \rightarrow \kappa$ be defined by: $(A_m \cap s)\phi = k_m$\\
 To see that $\phi$ is well-defined (that no  elements of $S$ can be represented as above in two ways) observe that if $A_{m_1} \neq A_{m_2}$ then $a_{m_1,m_2}\in A_{m_1} \triangle A_{m_2}$ and so is in precisely one of $s\cap A_{m_1}$ and $s\cap A_{m_2}$.\\
 It is clear that $(s,S,\phi)\in \mathcal{T}$.\\
Finally we have that for all $m<n$:
$f_{A_m}(s,S,\phi)= (s\cap A_m)\phi = k_m$
as required. $\qed$
\begin{theorem}\label{lots of ultrafilters}
Let $\kappa$ be an infinite cardinal. There are $2^{2^\kappa}$ ultrafilters on $\kappa$.
\end{theorem}\par
Proof: This proof comes from the proof of theorem 2.5 in \cite{no ultrafilters}.\\
It is clear that any ultrafilter on $\kappa$ is an element of $P(P(\kappa))$ and therefore there are at most $\vert P(P(\kappa)) \vert = 2^{2^{\kappa}}$ of them.\\
To show there are at least $2^{2^{\kappa}}$ we will construct such a family of ultrafilters.\\
Let $L\subseteq 2^\kappa$ be a family of large oscillation such that $\vert L \vert = \vert 2^\kappa \vert$ (the existance of $L$ follows from theorem \ref{large osc}).\\
For $S \in P(L)$ we define $(S)B$ by:
$$(S)B:=\{(\{0\})f^{-1}:f \in S\} \cup \{(\{1\})f^{-1}: f \in S^c \}$$
Let $B_1,B_2 \ldots B_k \in (S)B$ then it follows that for $i \in \{1,2,\ldots k\}$ we have $B_i = (b_i)f_i^{-1}$ for some $f_1,f_2 \ldots f_k \in L$ and $b_1,b_2 \ldots b_k\in \{0,1\}$. As $L$ has large oscillation it follows that for some $k\in \kappa$, $(k)f_i = b_i$ for all $i \in \{1,2 \ldots k\}$, and therefore $k \in \bigcap_{i \in \{1,2,\ldots k\}} B_i$. Therefore $(S)B$ has the finite intersection property. It therefore follows  by theorem \ref{ultrafilter lemma} that for every $S\in P(L)$ we can extend $(S)B$ to an ultrafilter $(S)\mathcal{U}$.\\
Suppose for a contradiction that there exists distinct $S_1, S_2,\in P(L)$ such that $(S_1)\mathcal{U} = (S_2)\mathcal{U}$. Then without loss of generality there exists $f \in S_1 \backslash S_2$. We have that $(\{0\})f^{-1} \in (S_1)\mathcal{U}$ is the complement of $(\{1\})f^{-1} \in (S_2)\mathcal{U}$. So because $(S_1)\mathcal{U} = (S_2)\mathcal{U}$ we have a set in an ultrafilter who's complement is also in that ultrafilter, this contradicts theroem \ref{Ultrafilter complement}. It follows that $(S)\mathcal{U}$ for $S \in P(L)$ are distinct ultrafilters and there is $\vert P(L) \vert = 2^{2^{\kappa}}$ of them as required. $\qed$
\section{Building groups from ultrafilters}

\begin{theorem}\label{full moiety}
Let $\Omega$ be an infinite set and let $G\leq Sym(\Omega)$, then if all moieties of $\Omega$ are full in $G$ then $G=Sym(\Omega)$
\end{theorem}\par
Proof: The following proof comes from Note 3 of section 4 in \cite{notes}\\
Let G be a group such that all moieties of $\Omega$ are full in $G$.\\
Let $M$ be a moiety of $\Omega$. We have that $M^c = \Omega \backslash M$ is also a moiety and therefore $G$ acts on fully on $M$ and $M^c$.
\[G_s:= \{f \in G: (x)f \in M  \iff x \in M\}\]
 It is clear that $G_s \leq Sym_{\Omega}(M)Sym_{\Omega}(M^c)$.\\
 Consider the map $\phi:G_s \rightarrow Sym(M^c)$ given by:
 \[(f)\phi = f\vert_{M^c}\]
 It is clear that $\phi$ is a homomorphism and therefore $ker(\phi) \unlhd G_s$, and in particular $\{f \vert_{M}:f \in ker(\phi)\}\unlhd \{ f \vert_{M} :f \in G_s\}$\\
It therefore follows that $ker(\phi) \unlhd Sym_{\Omega}(M)$, as $ker(\phi) \cong \{f \vert_{M}:f \in ker(\phi)\}$ and $Sym_{\Omega}(M) \cong \{ f \vert_{M} :f \in G_s\}$\\
 Construct two disjoint moieties $\Sigma_1$ and $\Sigma_2$ such that $\Sigma_1 \cup \Sigma_2 = M$.\\
 Construct an element $g \in Sym_{\Omega}(\Sigma_1)$ such that $g$ has $\vert \Omega \vert $ cycles of all finite lengths.\\
As $\Sigma_1 \cup M$ is a moiety there is an element $g'$ of G such that $g'\vert_{\Sigma_1 \cup M} = g$.\\
Clearly $g' \in ker(\phi)$ and therefore it's normal closure with respect to $Sym_{\Omega}(M)$ satisfies: $\langle \langle g'\rangle \rangle \unlhd ker(\phi) \unlhd Sym_{\Omega}(M)$.\\
$\langle \langle g'\rangle \rangle$ is closed under conjugation and therefore any element with the same disjoint cycle structure is an element.\\
We have that $g'$ has $\vert \Omega \vert$ cycles of all finite lengths.\\ We now construct an new element $g^*$ by reversing all cycles of odd or infinite length of $g'$ and preserving the others. In the product $g^* g'$ all cycles of odd or infinite length die and all cycles of even length are squared.\\ Notice that the square of a cycle of length $2n$ gives two cycles of length $n$. It follows that $g^* g'$ has $2*\vert \Omega \vert=\vert \Omega \vert$ cycles of all finite lengths and none of infinite length.\\
Therefore $\langle \langle g'\rangle \rangle$ contains all elements of $Sym_{\Omega}(M)$ which have $\vert \Omega \vert$ cycles of all finite lengths and none of infinite length and therefore by theorem \ref{commutator result}
$\langle \langle g'\rangle \rangle = Sym_{\Omega}(M)$. It follows that $ker(\phi)=Sym_{\Omega}(M)$ and thus $Sym_{\Omega}(M) \leq G$ for all moieties $M$. We therefore have by theorem \ref{stabgens} that $G = Sym(\Omega)$.
\begin{defn}
Given an ultrafilter $\mathcal{U}$ on an infinite set $\Omega$.
$$F_{\mathcal{U}}:= \{f \in Sym(\Omega):fix(f) \in \mathcal{U} \}$$
\end{defn}
\begin{theorem}
Given an ultrafilter $\mathcal{U}$ on an infinite set $\Omega$. Then $F_{\mathcal{U}}$ is a subgroup of $Sym(\Omega)$.
\end{theorem}\par
Proof: Clearly $F_{\mathcal{U}} \subseteq Sym(\Omega)$.\\
Let $f \in F_{\mathcal{U}}$ then $fix(f) = fix(f^{-1})$ so $f\in F_{\mathcal{U}}$.
So we have that $F_{\mathcal{U}}$ is closed under inverses.\\
Let $f,g \in G_{S}$ we have that $fix(f)$ and $fix(g)$ are in $\mathcal{U}$ therefore $fix(f)\cap fix(g) \in \mathcal{U}$. Clearly $fix(fg) \supseteq fix(f) \cap fix(g)$ and therefore $fix(fg)\in \mathcal{U}$ and $fg \in F_{\mathcal{U}}$ as required.\\
We therefore have that $F_{\mathcal{U}}$ is a group. $\qed$
\begin{defn}
Let $S$ be a set and let $G$ be a group which acts on $S$. Then we say $G$ is \textit{transitive} on $S$ if for all $x,y \in S$ there is an $f\in G$ such that $(x)f = y$.
\end{defn}
\begin{theorem}\label{transitive moieties 1}
Let $\mathcal{U}$ be an ultrafilter on a set $X$, then $F_{\mathcal{U}}$ is transitive on the moieties of $X$ in $\mathcal{U}$.
\end{theorem}\par
Proof: The following proof comes from the proof of theorem 6.4 in \cite{ultrafiltermax}.\\
Let $M_1, M_2 \in \mathcal{U}$ be moieties of $\mathbb{N}$.\\
We will show that there is an element of $F_{\mathcal{U}}$ mapping $M_1$ to $M_2$.\\
Case 1: If $M_1 \backslash M_2$ and $M_2 \backslash M_1$ are both moieties then it follows that $\vert M_1 \backslash  M_2 \vert =  \vert M_2 \backslash  M_1\vert$.\\
Let $\phi: M_1 \backslash  M_2 \rightarrow M_2 \backslash  M_1$ be a bijection.\\
Let $f\in Sym(\Omega)$ be defined by:
$$(x)f = \left\{
    \begin{array}{lr}
      x &  x \in M_1 \cap M_2\\
      x &  x \in X \backslash (M_1 \cup M_2)\\
      (x)\phi^{-1} &  x \in M_2 \backslash  M_1\\
      (x)\phi &  x \in M_1 \backslash  M_2\\
    \end{array}
    \right\}$$
as $M_1 \cap M_2 \subseteq fix(f)$ it follows that $fix(f) \in \mathcal{U}$ and therefore $f \in F_{\mathcal{U}}$.\\
We have that $(M_1)f = M_2$ as required.\\
Case 2: Suppose that $M_1 \backslash  M_2$ or $M_2 \backslash  M_1$ is not a moiety.\\
Then wlog $M_1 \backslash  M_2$ is not a moiety. Then it follows that $\aleph_0 > \vert M_1 \backslash  M_2 \vert$.\\ 
Therefore we have the following:\\
 $\vert M_1 \cap M_2 \vert = \vert M_1 \backslash (M_1 \backslash  M_2)\vert = \vert M_1 \vert - \vert M_1 \backslash  M_2 \vert = \aleph_0$\\
  $\vert X \backslash  (M_1 \cap M_2) \vert =  \vert (X \backslash  M_1) \cup (X \backslash  M_2) \vert  = \aleph_0$\\
 $\vert M_1 \cup M_2 \vert = \vert M_2 \cup (M_1 \backslash  M_2)\vert = \vert M_2 \vert + \vert M_1 \backslash  M_2 \vert = \aleph_0$\\
  $\vert X \backslash  (M_1 \cup M_2) \vert =  \vert X \backslash  (M_2 \cup (M_1 \backslash M_2)) \vert =  \vert (X \backslash  M_2) \cap (X \backslash (M_1 \backslash  M_2)) \vert=  \vert (X \backslash  M_2)  \backslash (M_1 \backslash M_2) \vert= \aleph_0$
So both $M_1 \cap M_2$ and $\Sigma_1 \cup M_2$ are moieties. \\
Let $M_3$ and $M_4$ be disjoint moieties such that $M_3 \cup M_4 = M_1 \cap M_2$. By theorem \ref{Ultrafilter complement} either $M_3$ or $X \backslash  M_3$ is in $\mathcal{U}$. If $X \backslash  M_3$ is in $\mathcal{U}$ then $M_4 = (X \backslash  M_3)\cap M_1 \cap M_2 \in \mathcal{U}$. So we have that either $M_3$ or $M_4$ is in $\mathcal{U}$.\\
Consider the set $M_4 \cup (X \backslash (M_1 \cup M_2))$.
\begin{align*}
(M_4 \cup (X \backslash (M_1 \cup M_2))) \backslash M_1&= X \backslash (M_1 \cup M_2) \text{ a moiety}.\\
M_1 \backslash(M_4 \cup (X \backslash (M_1 \cup M_2)))&= M_1 \backslash M_4\text{ a moiety (as it contains }M_3\text{ and it's complement contains the complement of }M_1\text{).}\\
(M_4 \cup (X \backslash (M_1 \cup M_2)))\backslash M_2&= X \backslash (M_1 \cup M_2) \text{ a moiety.}\\
M_2 \backslash(M_4 \cup (X \backslash (M_1 \cup M_2)))&= M_2 \backslash M_4\text{ a moiety (as it contains }M_3\text{ and it's complement contains the complement of }M_2\text{).}
\end{align*}
    So by Case 1 there are elements of $F_{\mathcal{U}}$ $f_1$ and $f_2$ such that:
    $$(M_1)f_1 = M_4 \cup (X \backslash  (M_1 \cup M_2))$$
    $$( M_4 \cup (X \backslash  (M_1 \cup M_2)))f_2 = M_2$$
    It follows that $(M_1)f_1f_2 = M_2$ and we have the required result.
     $\qed$
\begin{theorem}\label{transitive moieties 2}
Let $\mathcal{U}$ be an ultrafilter on a set $X$, then $F_{\mathcal{U}}$ is transitive on the moieties of $X$ not in $\mathcal{U}$.
\end{theorem}\par
Proof: Let $M_1$ and $M_2$ be moieties not in $\mathcal{U}$ then both $X \backslash  M_1$ and  $X \backslash  M_2$ are in $\mathcal{U}$ by theorem \ref{Ultrafilter complement}.\\ By theorem \ref{transitive moieties 1} there is an element of $F_{\mathcal{U}}$ such that:
$$(X \backslash  M_1)f = X \backslash  M_2$$
It therefore follows that $(M_1)f = M_2$ as required. $\qed$
\begin{theorem}\label{Ultrasubgroup}
Let $\mathcal{U}$ be an ultrafilter, then $F_{\mathcal{U}} \leq Sstab(\mathcal{U})$.
\end{theorem}
Let $f \in F_{\mathcal{U}}$ then $fix(f)\in \mathcal{U}$.\\
Let $U \in \mathcal{U}$. $U\cap fix(f) \in \mathcal{U}$ and $(U)f =(U\cap fix(f))\cup (U \backslash  fix(f))f \supseteq U\cap fix(f)$ so we have that $(U)f \in \mathcal{U}$.\\
Similarly let $(U)f \in \mathcal{U}$. $(U)f\cap fix(f) \in \mathcal{U}$ and $U =((U)f)f^{-1}=((U)f\cap fix(f))\cup ((U)f \backslash  fix(f))f^{-1} \supseteq (U)f\cap fix(f)$ so we have that $U \in \mathcal{U}$.\\
Therefore $f \in Sstab(\mathcal{U})$ as required. $\qed$
\begin{theorem}\label{ultrafilter movement}
Let $\mathcal{U}$ be an ultrafilter on a set $X$ and let $g$ be a permutation on $X$. Let $\mathcal{V}$ be defined by:
$$\mathcal{V} = (\mathcal{U})g := \{(U)g:U \in \mathcal{U}\}$$
then $\mathcal{V}$ is an ultrafilter on $X$.
\end{theorem}\par
Proof: We first show that $\mathcal{V}$ is a filter.\\
1. As $X \in \mathcal{U}$ we have that $X = (X)g \in \mathcal{V}$.\\
2. Let $A,B \in \mathcal{V}$ then $A = (A')g$ and $B= (B')g$ for some $A',B'\in \mathcal{U}$. As $\mathcal{U}$ is a filter it follows that $A'\cap B' \in \mathcal{U}$ and therefore $A\cap B= (A')g \cap (B')g = (A'\cap B')g \in \mathcal{V}$.\\
3. Let $A\in \mathcal{V}$ and let $B \supseteq A$. Then $A=(A')g$ for some $A' \in \mathcal{U}$ and  $A' \subseteq Bg^{-1}$ . So $Bg^{-1} \in \mathcal{U}$ and therefore $B \i \mathcal{V}$.\\
We now show that $\mathcal{V}$ is an ultrafilter. Clearly as $\emptyset \notin \mathcal{U}$ it follows that $\emptyset \notin \mathcal{U}$.\\
It therefore suffices to show that there is no filter $\mathcal{V}'$ such that $\mathcal{V} \subset \mathcal{V}' \subset P(X)$. Suppose for a contradiction that  there is such a $\mathcal{V}'$. Let $\mathcal{U}'$ be defined by:
$$\mathcal{U}' = \{(V)g^{-1}:V \in \mathcal{V}'\}$$
By using $g^{-1}$ with the previous part of the proof we have that $\mathcal{U}'$ is a filter and clearly we have that $\mathcal{U} \subset \mathcal{U}' \subset P(X)$.\\
This contradicts the fact that $\mathcal{U}$ is an ultrafilter. $\qed$
\begin{theorem}\label{max ultragroup}
Let $\mathcal{U}$ be an ultrafilter on an infinite set $\Omega$. The group $Sstab(\mathcal{U})$ is a maximal subgroup of $Sym(\Omega)$.
\end{theorem}\par
Proof: This proof comes from the proof of theorem 6.4 in \cite{ultrafiltermax}.\\
Let $g \in Sym(\Omega) \backslash  Sstab(\mathcal{U})$. Suppose that for all moieties of $\Omega$, $M \in \mathcal{U}$ we have $(M)g \in \mathcal{U}$. We have by theorem \ref{ultrafilter movement} that $(\mathcal{U})g$ is an ultrafilter who's moieties are all contained in $\mathcal{U}$, it follows that $\mathcal{U}$ and $(\mathcal{U})g$ have that same moieties and therefore by theorem \ref{moiety determined} we have that $(\mathcal{U})g = \mathcal{U}$ a contradiction. So we have that there is a moiety $M_1 \in \mathcal{U}$ such that $(M_1)g \notin \mathcal{U}$. Let $M_2$ be a moiety then precisely one of $M_2$ and $\Omega \backslash  M_2$ is a moiety in $\mathcal{U}$ by theorem \ref{Ultrafilter complement}. Wlog suppose $\Omega \backslash  M_2 \in \mathcal{U}$. Then $F_{\mathcal{U}}$ acts fully on $M_2$ as these elements can clearly be constructed using elements who fix elements $\Omega \backslash  M_2 \in \mathcal{U}$. By theorem \ref{Ultrasubgroup} $F_{\mathcal{U}} \leq Sstab(\mathcal{U})$. So $Sstab(\mathcal{U})$ acts fully on $M_2$. Let $M$ be a moiety then if $M \notin \mathcal{U}$ it follows that $Sstab(\mathcal{U})$ acts fully on $M$. As by theorem \ref{transitive moieties 2} we can map $M$ to $M_2$, act on $M_2$ and then map $M_2$ to $M$.\\
If $M \in \mathcal{U}$ then in $\langle Sstab(\mathcal{U}), g \rangle$ 
we can map $M$ to
 $M_1$,
 $M_1$
to $(M_1)g$ and $(M_1)g$ to $M_2$. It follows that $\langle Sstab(\mathcal{U}), g \rangle$ acts fully on $M$ by mapping $M$ to $M_2$, acting on $M_2$ and then mapping back.\\
We now have that $\langle Sstab(\mathcal{U}), g \rangle$ acts fully on all moieties of $\Omega$ and therefore by theroem \ref{full moiety} $\langle Sstab(\mathcal{U}), g \rangle = Sym(\Omega)$ and thus $ Sstab(\mathcal{U})$ is maximal.\\
To complete the proof we show that $Sstab(\mathcal{U}) \neq Sym(\Omega)$. Above we have that $\Omega\backslash M_2 \in \mathcal{U}$ and $M_2 \notin \mathcal{U}$ it follows that no element of $Sstab(\mathcal{U})$ maps $\Omega\backslash M_2 \in \mathcal{U}$ to $M_2$. But in $Sym(\Omega)$ this is clearly not the case so it follows that $Sstab(\mathcal{U}) \neq Sym(\Omega)$. $\qed$
\begin{theorem}\label{ultrastab equiv}
Let $\mathcal{U}$ be an ultrafilter on an infinite set $\Omega$, then $Sstab(\mathcal{U})= F_{\mathcal{U}}$
\end{theorem}\par
Proof: This proof comes from the proof of theorem 6.4 in \cite{ultrafiltermax}.\\
As by theorem \ref{Ultrasubgroup} we have $F_{\mathcal{U}} \leq Sstab(\mathcal{U})$, it suffices to show that $Sstab(\mathcal{U}) \subseteq F_{\mathcal{U}}$.
Let $f \in Sstab(\mathcal{U})$. We will construct two moieties $M_1$ and $M_2$ which partition $\mathbb{N}$. For each non-trival cycle of $f$ choose an element $x$ then we place all elements of the form $(x)f^{k}$ for even k in $M_1$, and all elements of the form $(x)f^{k}$ for odd k in $M_2$. Then $fix(f)$ is distributed between $M_1$ and $M_2$ such that they are moieties.\
It is clear that $ fix(f)^c \subseteq M_1 \cup (M_1)f$ and similarly $\ fix(f)^c \subseteq M_2 \cup (M_2)f$\\
By theorem \ref{Ultrafilter complement} precisely one of $M_1$ or $M_2$ is in $\mathcal{U}$. Wlog suppose that $M_1 \notin \mathcal{U}$, then as $f \in Sstab(\mathcal{U})$ it follows that $(M_1)f \notin \mathcal{U}$.\\
We have that $M_1^c, ((M_1)f)^c \in \mathcal{U}$ so $M_1^c \cap ((M_1)f)^c \in \mathcal{U}$ and therefore $(M_1^c \cap ((M_1)f)^c)^c \notin \mathcal{U}$.\\
So $M_1 \cup (M_1)f \notin \mathcal{U}$. If $fix(f)^c$ were in $\mathcal{U}$ that all it's supersets would be in $\mathcal{U}$ and we have just show that this is not the case so it follows that $fix(f)^c \notin \mathcal{U}$ so by applying theorem \ref{Ultrafilter complement} a final time we have that $fix(f) \in \mathcal{U}$ and therefore $f \in F_{\mathcal{U}}$ and $Sstab(\mathcal{U}) \subseteq F_{\mathcal{U}}$ as required. $\qed$
\begin{theorem} \label{distinct ultragroups}
Let $\mathcal{U}_1$ and $\mathcal{U}_2$ be distinct ultrafilters on an infinite set $\Omega$. Then $Sstab(\mathcal{U}_1) \neq Sstab(\mathcal{U}_2)$.
\end{theorem}\par
Proof: Let $U\in \mathcal{U}_1 \backslash  \mathcal{U}_2$ be a moiety (this must exist by theorem \ref{moiety determined}). By theorem \ref{ultrastab equiv} it suffices to prove $F_{\mathcal{U}_1} \neq F_{\mathcal{U}_2}$. Choose $f \in Sym(\Omega)$ such that $fix(f) = U$ then  $f \in F_{\mathcal{U}_1} \backslash  F_{\mathcal{U}_2}$ and therefore $F_{\mathcal{U}_1} \neq F_{\mathcal{U}_2}$ as required. $\qed$
\begin{theorem}
Let $\Omega$ be an infinite set, there are $2^{2^{\vert \Omega \vert}}$ non-conjugate maximal subgroups of $Sym(\Omega)$
\end{theorem}
\begin{proof}
This proof comes from the proof of corollary 6.5 in \cite{ultrafiltermax}.\\
By theorem \ref{lots of ultrafilters} there are $2^{2^{\vert \Omega \vert}}$ ultrafilters on $\Omega$. It follows from theorems \ref{distinct ultragroups} and \ref{Ultrasubgroup} that there are $2^{2^{\aleph_{0}}}$ maximal subgroups of $Sym(\Omega)$. As there are only $2^{\vert \Omega \vert}$ elements of $Sym(\Omega)$ it follows that there exists a family of $2^{2^{\aleph_{0}}}$ non-conjugate maximal subgroups, as one can be chosen from each of the $2^{2^{\aleph_{0}}}$ conjugacy classes.
\end{proof}
\section{Finite partition stabilizers}
In this section we will explore more examples of maximal subgroups of infinite symmetric groups in the form of partition stabilizers.
\begin{theorem}\label{finstab}
Let $\Omega$ be an infinite set, and let $F \subseteq \Omega$ be finite. We then have that $Sstab(\{F,F^c\})$ is a maximal subgroup of $Sym(\Omega)$.
\end{theorem}\par
Proof: Let $f\in Sym(\Omega)\backslash Sstab(F)$ and $g \in Sym(\Omega)$. Note that clearly $Sstab(\{F,F^c\})=Sstab(F)$, it therefore suffices to show that $g \in \langle Sstab(F), f \rangle$.\\
As $f \notin Sstab(F)$ there must exist a point $p \in F$ such that $(p)f \in F^c$.\\
Let $p_2 \in (F^c)f$. It is clear that $((p)f, p_2) \in Sstab(F)$. It therefore follows that $(p , (p_2)f^{-1})=f ((p)f,  p_2) f^{-1} \in \langle Sstab(F),f\rangle$.\\
Let $a\in F$, $b\in F^c$. We have that $(a ,b)=(a, p)(p, (p_2)f^{-1})((p_2)f^{-1}, b) \in \langle Sstab(F),f\rangle$.\\
Label $F$ by $F = \{p_i:i<k\}$.\\
It is clear that $g(p_1, (p_1)g)(p_2, (p_2)g) \ldots (p_k, (p_k)g)\in  Sstab(F)$.\\
It therefore follows that $g \in \langle Sstab(F), f \rangle$ as required. $\qed$
\begin{theorem}\label{symgroupintergen}
Let $\Omega$ be an infinite set. Let $\Sigma_1$ and $\Sigma_2$ be infinite subsets of $\Omega$ such that $\vert \Sigma_1 \cap\Sigma_2 \vert = \vert \Sigma_1 \cup\Sigma_2 \vert$.
\[Sym_{\Omega}(\Sigma_1 \cup \Sigma_2) = \langle Sym_{\Omega}(\Sigma_1) ,Sym_{\Omega}(\Sigma_2)  \rangle\]
\end{theorem}\par
Proof: This proof comes from \cite{symgroupintergen}.\\
It is clear that $Sym_{\Omega}(\Sigma_1 \cup \Sigma_2) \geq \langle Sym_{\Omega}(\Sigma_1) ,Sym_{\Omega}(\Sigma_2)  \rangle$.\\
Let $f \in Sym_{\Omega}(\Sigma_1 \cup \Sigma_2)$. We have that either $\vert (\Sigma_1 \cap \Sigma_2)f \cap \Sigma_1 \vert = \vert \Sigma_1 \cap \Sigma_2 \vert$ or $\vert (\Sigma_1 \cap \Sigma_2)f \cap \Sigma_2 \vert = \vert \Sigma_1 \cap \Sigma_2 \vert$. Without loss of generality say that $\vert (\Sigma_1 \cap \Sigma_2)f \cap \Sigma_1 \vert = \vert \Sigma_1 \cap \Sigma_2 \vert$. Let $M$ be a moiety of $(\Sigma_1 \cap \Sigma_2)f \cap \Sigma_1$. It follows that $M$ is a moiety of $\Sigma_1$ and $(M)f^{-1}$ is a moiety of $\Sigma_1 \cap \Sigma_2$, $\Sigma_1$ and $\Sigma_2$. Let $f' \in Sym_{\Omega}(\Sigma_1)$ be such that $(p)ff'=p$ for all $p \in Mf^{-1}$. As $\vert \Sigma_1 \cap \Sigma_2 \vert = \vert \Sigma_1 \vert$ we have that $\Sigma_1 \backslash \Sigma_2$ is contained in a moiety of $\Sigma_1$ and thus there exists an element $g\in \in Sym_{\Omega}(\Sigma_1)$ such that $\Sigma_1 \backslash \Sigma_2 \subseteq Mf^{-1}g$. As $ff'$ fixes $Mf^{-1}$ it follows that $g^{-1}ff'g \in Sym_{\Omega}(\Sigma_1 \cup \Sigma_2) $ fixes $\Sigma_1 \backslash \Sigma_2$ and thus $g^{-1}ff'g \in Sym_{\Omega}(\Sigma_2) $. So we have that $f \in g Sym_{\Omega}(\Sigma_2)g^{-1}f'^{-1} \subseteq \langle Sym_{\Omega}(\Sigma_1) ,Sym_{\Omega}( \Sigma_2)  \rangle$ as required. $\qed$
\begin{defn}
Let $\Omega$ be an infinite set and $\kappa$ be a cardinal then the $\kappa$-bounded symmetric group is defined by:
\[Sym(\Omega)_{<\kappa}:=\{f \in Sym(\Omega): \vert fix(f)^c \vert < \kappa\}\]
\end{defn}
\begin{theorem}\label{moistab}
Let $\Omega$ be an infinite set, and let $P=\{M_0,M_1 \ldots M_k\}$ be a partition of $\Omega$ into finitely many moieties. We then have that $Sstab(P)$ is not a maximal subgroup of $Sym(\Omega)$. In particular $Sstab(P)< \langle Sym(\Omega)_{<\vert \Omega \vert},Sstab(P)\rangle$ which is maximal.
\end{theorem}\par
Proof: This proof comes from \cite{ultrafiltermax}.\\
It is clear that $Sstab(P) \leq \langle Sym(\Omega)_{<\vert \Omega \vert},Sstab(P)\rangle$. Let $x_1 \in M_1$ and $x_2 \in M_2$ it is clear that $(x_1, x_2) \in Sym(\Omega)_{<\vert \Omega \vert}\backslash Sstab(P)$ and thus $Sstab(P) < \langle Sym(\Omega)_{<\vert \Omega \vert},Sstab(P)\rangle$.\\
We now show that $\langle Sym(\Omega)_{<\vert \Omega \vert},Sstab(P)\rangle$ is maximal. We first show that $\langle Sym(\Omega)_{<\vert \Omega \vert},Sstab(P)\rangle \neq Sym(\Omega)$.
Let $\{M_{1,1} , M_{1,2}\}$ be a partition of $M_1$ into moieties of $M_1$ and $\{M_{2,1},M_{2,2}\}$ be a partition of $M_2$ into moieties of $M_2$. Then we have that $\vert M_{1,1} \vert = \vert M_{2,1}\vert$ and thus there is a bijection $\phi:M_{1,1} \rightarrow M_{2,1}$. It is clear that the element of $Sym(\Omega)$ which swaps $M_{1,1}$ and $M_{2,1}$ with $\phi$ is not in $\langle Sym(\Omega)_{<\vert \Omega \vert},Sstab(P)\rangle$.\\
Let $f \in Sym(\Omega)\backslash \langle Sym(\Omega)_{<\vert \Omega \vert},Sstab(P)\rangle$ then $\vert fix(f)^c \vert = \vert \Omega \vert$ and $(M_i)f \notin P$ for some $i \leq k$.\\
It suffices to show that $\langle f, \langle Sym(\Omega)_{<\vert \Omega \vert},Sstab(P)\rangle \rangle=\langle f,  Sym(\Omega)_{<\vert \Omega \vert},Sstab(P) \rangle = Sym(\Omega)$.\\
Observe that there must exist $j_1,j_2$ such that $(M_i)f \cap M_{j_1}$ and $(M_i)f \cap M_{j_2}$ are moieties of $\Omega$, as if either of them have cardinality less than $\vert \Omega \vert$ then we can construct $f$ as the product of elements of $Sym(\Omega)_{<\vert \Omega \vert}$ and $Sstab(P)$.\\
As $Sym_{\Omega}((M_i)f \cap (M_{j_1}\cup M_{j_2}))$ is a subgroup of
 $fPstab(M_i) \leq \langle f,  Sym(\Omega)_{<\vert \Omega \vert},Sstab(P) \rangle$, and
  $Sym_{\Omega}(M_{j_1})$ is a subgroup of 
$Pstab(M_{j_1}^c) \leq Sstab(P) \leq \langle f,  Sym(\Omega)_{<\vert \Omega \vert},Sstab(P) \rangle$.\\
 It follows by theorem \ref{symgroupintergen} that $Sym_{\Omega}(M_{j_1}\cup ((M_i)f \cap (M_{j_1}\cup M_{j_2})))\leq \langle f,  Sym(\Omega)_{<\vert \Omega \vert},Sstab(P) \rangle$.
As $Sym_{\Omega}(M_{j_2})$ is a subgroup of 
$Pstab(M_{j_2}^c) \leq Sstab(P) \leq \langle f,  Sym(\Omega)_{<\vert \Omega \vert},Sstab(P) \rangle$ it follows by using theorem \ref{symgroupintergen} again that $Sym_{\Omega}((M_{j_1}\cup ((M_i)f \cap (M_{j_1}\cup M_{j_2}))))\cup(M_{j_2})=Sym_{\Omega}(M_{j_1}\cup M_{j_2}) \leq \langle f,  Sym(\Omega)_{<\vert \Omega \vert},Sstab(P) \rangle$.\\
As the elements of $P$ all have the same cardinality we can swap them using elements of $Sstab(P)$. If follows that for all $i,j \leq k$ we have $Sym_{\Omega}(M_{i}\cup M_{j}) \leq \langle f,  Sym(\Omega)_{<\vert \Omega \vert},Sstab(P) \rangle$. By repeadedly applying theorem \ref{symgroupintergen} we will get the required result as follows:
\begin{align*}
&Sym_{\Omega}(M_{0}\cup M_{1}) \leq \langle f,  Sym(\Omega)_{<\vert \Omega \vert},Sstab(P) \rangle \text{ and } Sym_{\Omega}(M_{1}\cup M_{2}) \leq \langle f,  Sym(\Omega)_{<\vert \Omega \vert},Sstab(P) \rangle \\
&\implies Sym_{\Omega}(M_{0}\cup M_{1} \cup M_{2}) \leq \langle f,  Sym(\Omega)_{<\vert \Omega \vert},Sstab(P) \rangle \text{ and } Sym_{\Omega}(M_{2}\cup M_{3}) \leq \langle f,  Sym(\Omega)_{<\vert \Omega \vert},Sstab(P) \rangle\\
&\implies Sym_{\Omega}(M_{0}\cup M_{1} \cup M_{2} \cup M_{3}) \leq \langle f,  Sym(\Omega)_{<\vert \Omega \vert},Sstab(P) \rangle \text{ and } Sym_{\Omega}(M_{3}\cup M_{4}) \leq \langle f,  Sym(\Omega)_{<\vert \Omega \vert},Sstab(P) \rangle \\
& \ \ \ \ \vdots\\
&\implies Sym_{\Omega}(M_{0} \cup M_1 \ldots \cup M_{k}) \leq \langle f,  Sym(\Omega)_{<\vert \Omega \vert},Sstab(P) \rangle \\
&\implies Sym_{\Omega}(\cup P) \leq \langle f,  Sym(\Omega)_{<\vert \Omega \vert},Sstab(P) \rangle \\
&\implies Sym(\Omega) \leq \langle f,  Sym(\Omega)_{<\vert \Omega \vert},Sstab(P) \rangle \qed
\end{align*}
\begin{theorem}\label{infstab}
Let $\Omega$ be an infinite set, and let $P=\{N,N^c\}$ be a partition of $\Omega$ where $\aleph_0 \leq \vert N \vert < \vert \Omega \vert$. We then have that $Sstab(P)$ is not a maximal subgroup of $Sym(\Omega)$. In particular $Sstab(P)< \langle Sym(\Omega)_{<\vert N \vert},Sstab(P)\rangle$ which is maximal.
\end{theorem}\par
Proof: It is clear that $Sstab(P) \leq \langle Sym(\Omega)_{<\vert N \vert},Sstab(P)\rangle$. Let $x_1 \in N$ and $x_2 \in N^c$ it is clear that $(x_1, x_2) \in Sym(\Omega)_{<\vert N \vert}\backslash Sstab(P)$ and thus $Sstab(P) < \langle Sym(\Omega)_{<\vert N \vert},Sstab(P)\rangle$.\\
We now show that $\langle Sym(\Omega)_{<\vert N \vert},Sstab(P)\rangle$ is maximal. We first show that $\langle Sym(\Omega)_{<\vert N \vert},Sstab(P)\rangle \neq Sym(\Omega)$.
Let $N' \subseteq N^c$ be such that $\vert N \vert = \vert N' \vert$. Then we have that there is a bijection $\phi:N \rightarrow N'$.\\
Let $\phi'$ be defined by:
\[(x)\phi' =  \left\{
    \begin{array}{lr}
      (x)\phi&  x\in N  \\
      (x)\phi^{-1}& x\in N'\\
      x& \text{otherwise}\\
    \end{array}
    \right\}\]
We have that $\phi'$ can't in $\langle Sym(\Omega)_{<\vert N \vert},Sstab(P)\rangle$ as it moves $\vert N \vert$ points.\\
Let $f \in Sym(\Omega)\backslash \langle Sym(\Omega)_{<\vert N \vert},Sstab(P)\rangle$. If $\vert Nf \cap N^c \vert < \vert N \vert$ and $\vert N^cf \cap N \vert < \vert N \vert$ then we could construct $f$ using elements of $Sym(\Omega)_{<\vert N \vert}$ and $Sstab(P)$ thus either $f$ or $f^{-1}$ must move $\vert N \vert$ points from $N$ into $N^c$. Without loss of generality say $f$ does this.\\
Let $N''$ be such that $N'' \subseteq N^c$, $\vert Nf \cap N^c \vert = \vert N'' \vert$ and $N'' \cap Nf = \emptyset$. Let $Nf \cap N^c=\{n_i:i<\vert N'' \vert\}$, $N''=\{n_i'':i<\vert N'' \vert\}$. Let $\phi:Nf \cap N^c \rightarrow N''$ be a bijection.\\
Let $\phi'$ be defined by:
\[(x)\phi' =  \left\{
    \begin{array}{lr}
      (x)\phi&  x\in Nf \cap N^c  \\
      (x)\phi^{-1}& x\in N''\\
      x& \text{otherwise}\\
    \end{array}
    \right\}\]
It follows that:
\[(x)f\phi' f^{-1} =  \left\{
    \begin{array}{lr}
      (n_i'')f^{-1}&  x = n_if^{-1}  \\
      (n_i)f^{-1}& x = n_i''f^{-1}\\
      x& \text{otherwise}\\
    \end{array}
    \right\}\]
We have that $f\phi'f^{-1}  \in \langle Sym(\Omega)_{<\vert N \vert},Sstab(P), f\rangle$ is an involution in $\langle Sym(\Omega)_{<\vert N \vert},Sstab(P), f\rangle$ which swaps $\vert N \vert$ points of $N$ with $\vert N \vert $ points of $N^c$. As $\langle Sym_{\Omega}(N), Sym_{\Omega}(N^c) \rangle \leq Sstab(P)$, We have that we can construct an involution which swaps any subset of $N$ with any subset of $N^c$ of the same cardinality.\\
Let $g\in Sym(\Omega)$, it suffices to show that $g \in \langle Sym(\Omega)_{<\vert N \vert},Sstab(P), f\rangle$.\\ 
Let $g' \in \langle Sym(\Omega)_{<\vert N \vert},Sstab(P), f\rangle$ be an involution which swaps $Ng$ with $N$. We have that $gg' \in Sstab(P)$ and therefore $g \in Sstab(P)g'^{-1} \leq \langle Sym(\Omega)_{<\vert N \vert},Sstab(P)\rangle $ as required. $\qed$
\begin{theorem}
Let $\Omega$ be an infinite set and let $P=\{\Omega_0, \Omega_1 \ldots \Omega_k\}$ be a partition of $\Omega$ into finitely many sets. Then $Sstab(P)$ is a maximal subgroup of $Sym(\Omega)$ iff $P=\{F, F^c\}$ where $F$ is finite or $P=\{S_1,S_2 \ldots S_{k-1}, \Omega \backslash (\cup_{i<k}S_i)\}$ where $S_1,S_2 \ldots S_{k-1}$ are singletons.
\end{theorem}\par
Proof: We will consider all the ways to partition $\Omega$ into finitely many sets.\\
Case 1: $P=\{S_0,S_1 \ldots S_{k-1}, \Omega \backslash (\cup_{i<k}S_i)\}$ where $S_0,S_1 \ldots S_{k-1}$ are singletons.\\
It follows that $Sstab(P)=Sstab(\{\cup_{i<k}S_i,\Omega \backslash (\cup_{i<k}S_i)\})$ and it therefore follows by theorem \ref{finstab} that $Sstab(P)$ is maximal.\\
Case 2: $P=\{S_0,S_1 \ldots S_{k_1}, \Omega_0, \Omega_1 \ldots \Omega_{k_2}\}$ where $S_0,S_1 \ldots S_{k_1}$ are singletons, $\Omega_0,\Omega_1 \ldots \Omega_{k_2}$ are not singletons, $k_1,k_2 \in \mathbb{N}$ and $k_2>1$.\\
Let $x_1 \in \Omega_1$ and $x_2 \in \Omega_2$. It is clear that $(x_1, x_2) \in Sstab(\{\cup_{i\leq k_1}S_i, \cup_{i\leq k_2}\Omega_i\} )\backslash Sstab(P)$ and therefore  $Sstab(P) < Sstab(\{\cup_{i\leq k_1}S_i, \cup_{i\leq k_2}\Omega_i\} )$ which by theorem \ref{finstab} is maximal and therefore $Sstab(P)$ is not maximal.\\
Case 3: $P=\{F_0,F_1 \ldots F_{k_1},\Omega_0, \Omega_1 \ldots \Omega_{k_2}\}$ where $F_1,F_2 \ldots F_{k_1}$ are finite sets at least one of which has at least $2$ elements, $k_1,k_2 \in \mathbb{N}$ and at least one of $k_1,k_2$ is not $1$.\\
Similarly to case 2 if $k_2 \neq 1$ then  $Sstab(P) < Sstab(\{\cup_{i\leq k_1}F_i, \cup_{i\leq k_2}\Omega_i\} )$ which by theorem \ref{finstab} is maximal and therefore $Sstab(P)$ is not maximal. If $k_2 = 1$ then we have wlog that $F_0$ has at least 2 elements and $k_1>1$. Therefore if $x_1 \in F_0$ and $x_2 \in F_1$ then it is clear that $(x_1, x_2) \in Sstab(\{\cup_{i\leq k_1}F_i, \cup_{i\leq k_2}\Omega_i\} )\backslash Sstab(P)$ and so $Sstab(P) < Sstab(\{\cup_{i\leq k_1}F_i, \cup_{i\leq k_2}\Omega_i\} )$ which by theorem \ref{finstab} is maximal and therefore $Sstab(P)$ is not maximal.\\
Case 4: $P=\{F,F^c\}$ where $F$ is finite.\\
It follows immediately by theorem \ref{finstab} that $Sstab(P)$ is maximal.\\
Case 5: $P=\{I,\Omega_0,\Omega_1 \ldots \Omega_k\}$ where $k \in \mathbb{N}$ and $\aleph_0 \leq \vert I \vert < \vert \Omega \vert$.\\
It is clear that $Sstab(P)\leq Sstab(\{\cup\{s \in P:\vert s \vert = \vert I \vert\},\{\cup\{s \in P:\vert s \vert \neq \vert I \vert\}\}\})$ which is not maximal by theorem \ref{infstab} and therefore $Sstab(P)$ is not maximal.\\
Case 6: $P=\{M_1,M_2 \ldots M_k\}$ a partition into moieties of $\Omega$.\\
It follows immediately from theorem \ref{moistab} that $Sstab(P)$ is not maximal. $\qed$
\begin{thebibliography}{3}
  \bibitem{Gdelta}
    Properties of $G_\delta$ Subsets of $\mathbb{R}$: \url{http://www.artsci.kyushu-u.ac.jp/~ssaito/eng/maths/Gdelta.pdf}
  \bibitem{4004} MT4004 Lecture notes
  \bibitem{4526} MT4526 Lecture notes
  \bibitem{}
   \url{https://data.math.au.dk/kurser/advanalyse/F06/lecture1.pdf}
        \bibitem{}
   \url{   https://en.wikipedia.org/wiki/Cantor%27s_intersection_theorem}
   \bibitem{finitegen}
   \url{http://citeseerx.ist.psu.edu/viewdoc/download?doi=10.1.1.842.5218&rep=rep1&type=pdf}
        \bibitem{bergman property}
        The Bergman property for semigroups: \url{https://research-repository.st-andrews.ac.uk/bitstream/handle/10023/2145/bergman7.pdf?sequence=1&isAllowed=y}
        \bibitem{filter defn}
        \url{https://en.wikipedia.org/wiki/Filter_(mathematics)}
        \bibitem{ultrafilter defn}
        \url{https://en.wikipedia.org/wiki/Ultrafilter}
        \bibitem{ultrafiltermax}
        \url{http://citeseerx.ist.psu.edu/viewdoc/download?doi=10.1.1.109.6104&rep=rep1&type=pdf}
        \bibitem{notes}
        \url{http://sci-hub.io/10.1112/blms/20.4.305}
        \bibitem{ordinals}
        \url{http://www.math.uchicago.edu/~may/VIGRE/VIGRE2009/REUPapers/Murphy.pdf}
        \bibitem{no ultrafilters}
        \url{https://www.math.leidenuniv.nl/scripties/NeveBach.pdf}
        \bibitem{zorn's lemma}
        \url{https://math.stackexchange.com/questions/97308/zorns-lemma-and-axiom-of-choice}
        \bibitem{symgroupintergen}
        \url{https://pdfs.semanticscholar.org/8f32/93409850c37d3a71969546c508063351bfd0.pdf}
        \bibitem{well orderable}
        \url{https://en.wikipedia.org/wiki/Well-ordering_theorem#Proof}
        \bibitem{good zorn}
        \url{http://people.reed.edu/~jerry/332/23zorn.pdf}
        \bibitem{shuffle1}
        \url{http://www.ams.org/journals/proc/2002-130-02/S0002-9939-01-06344-4/S0002-9939-01-06344-4.pdf}
\bibitem{shuffle2}
\url{http://www.ams.org/journals/proc/2002-130-08/S0002-9939-02-06309-8/S0002-9939-02-06309-8.pdf}
\bibitem{abelian prod}
\url{http://www.ams.org/journals/proc/2003-131-06/S0002-9939-02-06720-5/S0002-9939-02-06720-5.pdf}
\end{thebibliography}
\end{document}
